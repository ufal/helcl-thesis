
% 20 stran:

% intro          1  (1)
% NAT background 5  (6)
% CTC            8 (14)
% experiments    7 (21)
% concl.         1 (22)


% %%%%%%%%%%%%%%%%%%%%%%%%%%%%%%%%%%%%%%%%%%%%%%%%%%%%%%%%%%%%%%%%%%%%%%%%%%%%%
\chapter{Introduction}
% %%%%%%%%%%%%%%%%%%%%%%%%%%%%%%%%%%%%%%%%%%%%%%%%%%%%%%%%%%%%%%%%%%%%%%%%%%%%%

In real-world applications of \ac{mt}, efficiency is often crucial.  Most
commercial \ac{nmt} models are available through a cloud-based service, such as
Microsoft Translator\footnote{\url{https://microsoft.com/translator/}} or
Google Translate.\footnote{\url{https://translate.google.com/}} Scaling
cloud-based solutions for large user bases is simple but costly. Even with a
large pool of computational resources, it is worthwhile to implement
optimizations that decrease model latency and improve user experience.

Locally deployed \ac{nmt} models provide a number of advantages over
cloud-based solutions. First, the service does not rely on the internet
connection. Second, the data is not being sent to a third party server and,
therefore, it is suitable for translating private or confidential data.
However, without optimization, running state-of-the-art translation models
locally often requires specialized hardware, such as one or more
GPUs. Otherwise, the time to translate a single sentence can easily exceed one
second on a standard CPU.

Higher decoding speeds can be achieved by model optimization. In their 2019
submission to the \ac{wngt} Efficiency Shared Task,
\citet{kim-etal-2019-research} successfully employed knowledge distillation,
quantization, shortlisting \citep{jean-etal-2015-using} and a simpler recurrent
unit design to bring the throughput of a translation model up to 3,600 words
per second on a CPU, with a modest drop in the translation quality. Following
this work, \citet{bogoychev-etal-2020-edinburghs} reported further improvements
with attention head pruning \citep{voita-etal-2019-analyzing}. Their work has
been part of the Bergamot Research Project, which aims to bring offline
translation models to a browser.\footnote{\url{https://browser.mt/}}

\Ac{nar} models present an alternative approach to model optimization, using
different architecture and a different decoding algorithm which has lower time
complexity.  In \ac{nmt}, a non-autoregressive decoding algorithm does not
access previously decoded outputs, imposing conditional independence assumption
on the output token probability distributions. This assumption allows for
parallelization of the decoding, which can significantly reduce the latency of
the translation system. On the other hand, it also presents a challenge to the
language model, which usually leads to poorer translation quality.


% %%%%%%%%%%%%%%%%%%%%%%%%%%%%%%%%%%%%%%%%%%%%%%%%%%%%%%%%%%%%%%%%%%%%%%%%%%%%%
\chapter{Non-Autoregressive Neural Machine Translation}
% %%%%%%%%%%%%%%%%%%%%%%%%%%%%%%%%%%%%%%%%%%%%%%%%%%%%%%%%%%%%%%%%%%%%%%%%%%%%%

The defining feature of a \ac{nar} model is the assumption of
conditional independence between the output distributions across time
steps. The output distribution in autoregressive models is defined as follows:
%
\begin{equation}
  p(y|x) = \prod_{t=1}^{T_y}p(y_t|y_{<t},x,\theta)
  \label{eq:output-distribution}
\end{equation}
%
Unlike Equation \ref{eq:output-distribution}, \Ac{nar} models do not condition
the output token probabilities on previously decoded outputs $y_{<t}$.  The
probability of an output sentence $y$ given an input sequence $x$ can then be
modeled as:
%
\begin{equation}
  p(y|x) = \prod_{t=1}^{T_y}p(y_t|x,\theta)
  \label{eq:nat-output-distribution}
\end{equation}

Although technically possible, making the outputs in \acs{rnn}-based models
conditionally independent does not reduce the time complexity because in
\acsp{rnn}, the value of each hidden state depends on the value of the
preceding state. However, in the Transformer model, hidden states in each layer
depend only on the states from the previous layer. This allows for parallel
computation at the layer level


In the following paragraphs, we discuss the necessary alterations to the
Transformer architecture. Since the outputs are conditionally independent, we
cannot feed the previously decoded outputs into the Transformer decoder. We
need to provide the input to the decoder and estimate the target length. The
causal mask over decoder self-attention is now unnecessary.

% ------------------------------------------------------------------------------
\paragraph{Multimodality Problem.} In one of the first applications of
non-autoregressive models to \ac{nmt}, \citet{gu2017nonautoregressive} describe
the \emph{multimodality problem} which arises when the outputs are
conditionally independent.

When estimating the probability of a word on a given position, there may be
multiple words which get a high probability. These words are the so-called
\emph{modes} of the distribution. In \acl{ar} models, once a word is selected,
other modes are ignored in the following time steps. However, a \acl{nar} model
does not base its decision for a given position on the preceding ones, so when
multiple positions have multiple modes, the model has no means of coordinating
the selection of modes across different time steps.

A well-known example of the multimodality problem is the translation of the
sentence ``thank you'' into German, which has two equally likely translations:
``vielen dank'' and ``danke schön.'' In this case, the pair of German tokens
``danke'' and ``vielen'' create the two modes in the first position, and the
tokens ``dank'' and ``schön'' are the modes in the second position. If an
\acl{ar} model chooses to generate ``danke'' in the first position, the token
``dank'' in the second position will no longer receive high probability from
the model. However, when a \acl{nar} model assigns high probabilities to the
correct translations, it also has to assign high probabilities to the other
(incorrect) two combinations, ``danke dank'' and ``vielen schön''
\citep{gu2017nonautoregressive}.

% ------------------------------------------------------------------------------
\paragraph{Decoder Inputs.} A \acs{nar} Transformer decoder cannot
receive the previously decoded tokens on the input. A solution proposed by
\citet{gu2017nonautoregressive} is to use a simple fertility model, which also
serves as the explicit target length estimator.

Compared to the \acl{ar} Transformer, the model has the following
modifications. First, the inputs to the decoder are made up of the sequence of
encoder inputs, either uniformly stretched to the predicted target sentence
length, or copied using a fertility model. Second, the decoder self-attention
does not use the causal mask, since all states can now attend to all other
states in both directions. Third, a \emph{positional attention} sub-layer is
added to every decoder layer, where the positional encoding is used
as queries and keys, and the decoder states as
values. \citet{gu2017nonautoregressive} argue that providing positional
information directly to the decoder layers could improve the potential of the
decoder to model local reodering.

% In \citet{gu2017nonautoregressive}, the multimodality problem (and length
% estimation) is addressed by introducing latent fertility variables
% $F = f_1, \ldots, f_{T_x}$ sampled from a prior distribution. Each
% $f_i \in \mathbb{N}_0$ denotes the number of times $x_i$ is copied to the
% decoder input (summing up to the target length $T_y$). The output probability
% is then conditioned on the latent vector $F$, which is marginalized out:
% %
% \begin{equation}
%   p(y|x, \theta) = \sum_{F \in \mathcal{F}} p(F|x, \theta) \cdot p(y|x, F, \theta)
% \end{equation}
% %
% where the fertility model $p(F|x, \theta)$ and the translation model
% $p(y|x, F, \theta)$ can be trained jointly using a variational lower bound with
% a candidate distribution $q$:
% \begin{align}
%   \begin{split}
%     \mathcal{L}(\theta)
%     & = \log p(y|x, \theta) = \log \sum_{F \in \mathcal{F}} p(F| x, \theta ) \cdot p(y | x, F, \theta) \\
%     & \geq \mathbb{E}_{F \sim q} \left(\sum_{t=1}^{T_y} \log p(y_t | x, F,
%       \theta) + \sum_{t=1}^{T_x} \log p(f_t | x, \theta) \right) +
%     \mathcal{H}(q)
%   \end{split}
% \end{align}
% %
% where $q$ is an external deterministic fertility model (and, therefore,
% $\mathcal{H}$ is a constant), and the expectation is also deterministic.  The
% fertility model depends on an external module which is not trained together
% with the model. The authors fine-tune the trained translation model using
% reinforcement learning \citep{williams1992simple} to estimate the gradients of
% the fertility model.

% During decoding, marginalizing over all possible fertility values is
% intractable. Therefore, \citet{gu2017nonautoregressive} experiment with three
% approximation methods -- argmax, average decoding, and \ac{npd}. In argmax
% decoding, the fertility with the highest probability is chosen in each step,
% similarly to greedy decoding. The average method chooses the expected fertility
% given the distribution in each position. \ac{npd} is based on sampling and
% rescoring with an \acl{ar} model, as explained below.

\paragraph{Knowledge Distillation.} To tackle the multimodality problem from
another angle, \citet{gu2017nonautoregressive} propose to use sequence-level
knowledge distillation to create artificial training data
\citep{kim-rush-2016-sequence}. Knowledge distillation is based on training an
\ac{ar} teacher model and using it to create artificial training data for a
student model. The main idea is that the outputs of a teacher model will have a
simpler structure than natural language, limiting the number of distribution
modes.

According to an ablation study published by \citet{gu-kong-2021-fully}, using
knowledge distillation is a crucial element in training non-autoregressive
translation models, regardless the actual method used.
\citet{zhou-etal-2020-understanding} further study the effects of knowledge
distillation strategies on the translation quality of \ac{nar} models. They
link the data complexity to the potential of \ac{nar} modeling and find that
knowledge distillation reduces the complexity, which in turn leads to improved
model performance.


% ------------------------------------------------------------------------------
\section{Limitations of Related Work}%
\label{sec:nat:discussion}
% ------------------------------------------------------------------------------

In this section, we take a high-level view of related approaches. We point out
a few aspects that most of the literature has in common, as well as some issues
regarding the evaluation and comparison to meaningful baselines.

% -----------------------------------------------------------------------------
\paragraph{Translation Quality.} Weak \acl{ar} baseline models are a common
element in the \acl{nar} literature. In most cases, the base variant of the
Transformer model is used as baseline (and most of the base hyperparameters are
used for the \ac{nar} model as well), arguing that having similar numbers of
parameters makes the baseline and the \ac{nar} model comparable. A comparable
model size is a valid point; on the other hand, it raises doubts about the
scalability of the proposed approaches; the difference between large \ac{ar}
and \ac{nar} model variants might not be proportional.

Moreover, even the Transformer base model can be trained in a better way than
as reported by \citet{vaswani2017attention} and as referenced by many \ac{nar}
papers as their baseline, achieving a better translation quality
\citep{popel-bojar-2018-training}. A reasonable improvement of the baseline
could be achieved by training a student \ac{ar} model on the
knowledge-distilled data.

% -----------------------------------------------------------------------------
\paragraph{Decoding Speed.} Similarly to the previous paragraph, the baselines
that appear in the literature are usually weak also in terms of decoding speed.
For example, \citet{gu2017nonautoregressive} report a latency of 408 ms for
their \ac{ar} model (a Transformer base), measured on a single Nvidia P100 GPU,
without batching. In contrast, our implementation achieves a latency of around
100 ms with the same model under the same hardware and hyperparameter settings
and no optimizations. Additionally, some articles cite a baseline \ac{ar}
latency of 607 ms \citep{wang-etal-2019-nonautoregressive,
  guo-etal-2020-jointly}, which is the result of
\citet{gu2017nonautoregressive} with beam search, even though beam search is
not used in the presented \ac{nar} models for the most part.

% -----------------------------------------------------------------------------
\paragraph{Evaluation Methodology.} Another aspect to be addressed is the
evaluation methodology itself. Setting aside the fact that automatic quality
evaluation using \acs{bleu} is the only reported metric, perhaps complemented
by a cursory manual evaluation on a small sample, the evaluation of the speed
improvements is inconsistent and sometimes the interpretation of the results is
outright wrong.

The main problem is that the actual decoding speed depends on a lot of factors
that are not easily reproducible. The most obvious factor is perhaps the
hardware on which the decoding speed is measured, followed by the
implementation. The average decoding time per sentence is also heavily
influenced by the batch size in batched decoding. Table
\ref{tab:related:hardware} shows that these conditions vary wildly within the
literature.

\begin{table}
  \centering

  \begin{tabular}{llcr}
    \toprule
    Publication & GPU type & CPU? & Batch \\
    \midrule
    \citet{gu2017nonautoregressive} & P100 & \xmark & 1 \\
    \citet{lee-etal-2018-deterministic} & P100 or P40 & \cmark & 1  \\
    \citet{kaiser2018fast} & GTX 1080 & \xmark  & 1, 64  \\
    \citet{ghazvininejad-etal-2019-mask} & Possibly V100 & \xmark  & 10 \\
    \citet{sun2019fast} & P100 & \xmark & 1    \\
    \citet{wang-etal-2019-nonautoregressive} & P100 & \xmark & 1    \\
    \citet{li-etal-2019-hint} & Possibly M40 & \xmark  & 1   \\
    \citet{ma-etal-2019-flowseq} &  TITAN X & \xmark &  {\it various}    \\
    \citet{ghazvininejad2020aligned} & {\it Not reported} & \xmark  & {\it unknown}  \\
    \citet{shao2020minimizing} &  TITAN X & \xmark  & 1   \\
    \citet{guo-etal-2020-jointly} & GTX 1080 Ti & \xmark & 1    \\
    \citet{kasai2020nonautoregressive} & V100 & \xmark & 1    \\
    \citet{qian-etal-2021-glancing} & GTX 1080 Ti & \xmark  & 1?  \\
    \citet{ran-etal-2021-guiding} & P40 & \xmark & 1   \\
    \citet{gu-kong-2021-fully} & V100 & \cmark  & 1   \\
    \citet{du2021orderagnostic} & {\it Not reported} & \xmark  & 1   \\
    \citet{huang-etal-2021-nonautoregressive} & V100 & \xmark & 1   \\
    \bottomrule
  \end{tabular}

  \caption{The hardware setting and decoding batch size for measuring the
    decoding speed as reported in a sample of papers described in this
    section.}%
  \label{tab:related:hardware}
\end{table}

A popular solution that takes the varying conditions into account for
evaluating decoding speed is to report the relative speed-up measured between
experiments within a single study. However, comparing these relative speed-up
ratios between different papers disregards the actual decoding times -- it is
easier to achieve 20 times speed-up over a slow baseline than over a baseline
which is faster.

Due to these reasons, we believe that we cannot show a fair comparison of the
methods presented here on a graph showing the pareto frontier between the
decoding speed and translation quality.

However, it is challenging to find a way to compare the contributions of
different research groups objectively. One such effort is made by the
organizers of the Efficiency Shared Task, now organized yearly at the \acl{wmt}
(\acs{wmt}\glsunset{wmt}; \citealp{heafield-etal-2020-findings,
  heafield-etal-2021-findings}).

In the Efficiency Shared Task, submissions are evaluated both in terms of
translation quality and decoding speed under various settings. All submissions
are evaluated on the same hardware, hosted by Amazon Web Services.  To amortize
the effects of data and model loading, the decoding time is measured on a
dataset that contains one million sentences. The speed is measured in five
different scenarios: GPU latency (decoding without batching) and throughput
(decoding with the optimal batch size), latency and throughput on a single CPU,
and throughput on a multi-core CPU. The main findings of the shared task are
that \acl{ar} models can be optimized to achieve latency of 5--20 ms even with
inexpensive hardware, without sacrificing much in terms of translation quality
\citep{heafield-etal-2021-findings}.

Our final observation is that measuring latency on a single GPU without
batching is a setup that favors \ac{nar} models, even though other scenarios
should also be considered for real-world applications. In batched GPU decoding,
the differences between \ac{ar} and \ac{nar} models fade, because
batch-parallelization of \ac{ar} models makes up for time-parallelization of
\ac{nar} models. On the other hand, in an online decoding scenario on a CPU,
only limited parallelization is possible, so the advantage of \ac{nar} models
is also diminished. We present detailed analysis of this behavior in
Chapter~\ref{chap:experiments}.


% %%%%%%%%%%%%%%%%%%%%%%%%%%%%%%%%%%%%%%%%%%%%%%%%%%%%%%%%%%%%%%%%%%%%%%%%%%%%%
\chapter{Connectionist Temporal Classification}%
\label{chap:nar-nmt-ctc}
% %%%%%%%%%%%%%%%%%%%%%%%%%%%%%%%%%%%%%%%%%%%%%%%%%%%%%%%%%%%%%%%%%%%%%%%%%%%%%

\paperdisclaim{This chapter is based on the paper \emph{``End-to-End
    Non-Autoregressive Neural Machine Translation with Connectionist Temporal
    Classification''}, joint work with Jindřich Libovický, published at EMNLP
  2018.}

\noindent
In this chapter, we lay the foundations for the \ac{nar} approaches to \ac{nmt}
studied in this thesis. We describe our experiments with an architecture based
on the \ac{ctc} loss \citep{libovicky-helcl-2018-end}.


\Ac{ctc} \citep{graves2006connectionist} is a method for training neural
networks on sequential data. Originally applied to the phonetic labelling task,
but later successfully adapted in related areas, including \ac{asr} or
handwriting recognition \citep{liwicki2007novel, eyben2009speech,
  graves2014towards}.

The main strength of \ac{ctc} becomes evident in tasks where the input and
output labels are weakly or not at all aligned, for example, in situations
where the observed input sequence is considerably longer than the target output
sequence -- hence the application to \ac{asr}, where the number of extracted
features per second is higher than the number of phonemes uttered per second.

Training neural networks with \ac{ctc} is independent of the actual neural
network architecture. The \ac{ctc} loss function can be applied on any network
with sequential outputs. Thus, this method is applicable to both \acp{rnn} and
the Transformer model.

Models trained with the \ac{ctc} assume that the alignment between the input
states (e.g. a group of frames in an audio signal) and the output states
(e.g. a phoneme) is unknown. A variable number of frames in a row can encode a
single phoneme. Similarly, in translation, multiple words in the source
language may correspond to any number of (even non-consecutive) words in the
target language.

The idea behind \ac{ctc} is to allow some states to produce no output. This is
achieved by introducing a special blank token into the target vocabulary.
Optionally, identical outputs produced by multiple consecutive states may be
merged and considered a single output. Because of these properties, there are
groups of equivalent output sequences, which all represent the same target, as
illustrated in Figure~\ref{fig:ctc-equivalent-sequences}.

\begin{figure}
  \centering
  \begin{minipage}{\textwidth}
    \begin{equation*}
        \text{a cat sat on a mat} =
        \begin{cases}
          & \text{a <blank> cat sat on a <blank> mat} \\
          & \text{a a cat cat sat on a mat} \\
          & \text{a <blank> cat cat sat on a mat} \\
          & \ldots
        \end{cases}
    \end{equation*}
  \end{minipage}
  \caption{A group of output sequences of equal length which all represent the
    same target in CTC.} %
  \label{fig:ctc-equivalent-sequences}
\end{figure}

In standard sequence-to-sequence architectures, the value of the loss function
is defined as the sum of cross entropies of the output distributions with
respect to the target sequence.% (see Equation \ref{eq:loss}).
In \ac{ctc}, the
loss is defined as the sum of cross-entropy losses of all output sequences
equivalent to the given target sequence:
%
\begin{equation}
  \mathcal{L}_{\text{CTC}}({\theta}) = - \sum_{(x, y) \in D} \sum_{y' \sim y}  \log p(y' | x, \theta)
  \label{eq:ctc-loss}
\end{equation}
%
where $\sim$ denotes the equivalence relation.

The inner summation in Equation \ref{eq:ctc-loss} is computed over all possible
sequences equivalent to the label sequence. For technical purposes, the label
sequences are limited to a fixed length, which greatly reduces the number of
acceptable hypotheses. However, the number of equivalent hypotheses of a given
length still grows exponentially with the sequence length -- in \ac{ctc}, the
fixed length is always set to be longer than the label sequence.

% The main difference between our approach and \ac{axe} described in Section
% \ref{sec:nat:alignment} is that in our models, we take into account all
% possible alignments, whereas \ac{axe} considers only the alignment with the
% minimum cross entropy. Moreover, we do not allow skipping positions in the
% ground truth.

The summation over the large set of equivalent sequences can be implemented
using dynamic programming. When both the length of the output and the length of
the target sequences are known, there is a constant number of blank tokens to
be generated. The process of computing the loss of the whole output sequence is
divided into computing the partial losses with respect to the possible label
prefixes.

\begin{figure}
  \centering

  \includegraphics[width=10cm]{img/ctc_schema.png}

  \caption{An illustration of the algorihm for the CTC loss computation. Each
    node denotes producing either a token from the label sequence, or the blank
    token. Each path from one of the two top-left nodes to one of the two
    bottom-right nodes corresponds to one of the equivalent sequences.  }
  \label{fig:ctc-dynamic-programming}
\end{figure}

The \ac{ctc} loss computation is illustrated in Figure
\ref{fig:ctc-dynamic-programming}. The rows represent tokens from the label
sequence plus the optional blank tokens. The columns represent the output
sequence.  Each node in the graph denotes generating a label from an output
distribution. The arrows show valid transitions between the generation
steps. An arrow can only go down one or two rows, or horizontally.  The
horizontal arrows denote repeated generation of the same label. These labels
are later merged to form a single output. An arrow can only go two rows down
when the skipped row corresponds to the blank token, so no target tokens are
left out. Therefore, each path in the diagram shows one of the equivalent
sequences that lead to generating the given label sequence.

Using the idea that many of the paths from left to right in the diagram share
segments, we can apply dynamic programming to compute the sum of losses across
all paths without the need to enumerate each of them. A node on coordinates
$(i,j)$ stores the accumulated losses for all path prefixes that lead to the
node, summed with the negative log-likelihood of the label on the $i$-th row
being generated by the $j$-th output state. The two bottom-right nodes then
store the sum of losses of all the paths.

Since the \ac{ctc} loss function is differentiable with respect to the model
parameters, the training of the network can be done in the standard way using
the backpropagation algorithm.


% -----------------------------------------------------------------------------
\section{Model Architecture}
\label{sec:ctc:arch}
% -----------------------------------------------------------------------------

As mentioned in the previous section, training models with \ac{ctc} does not
impose any requirements on the model architecture. In our experiments, we aim
to make a reasonable comparison between our proposed approach and the
state-of-the-art \acl{ar} models. We adapt the Transformer model and use
similar hyperparameters where applicable.

\Acl{nar} models generate the outputs in parallel, which requires that the
output length is known beforehand. In \acl{ar} models, the end of the sequence
is indicated by a special end symbol, and the constraint on maximum length is
merely a technical aspect.

To maximize the ability to output empty tokens to the full extent, the output
length should be set to a higher number than the length of the target sequence.
Since the length estimation does not need to be accurate, we select a number
$k$, and we set the target sequence length to be $k$ times longer than the
source length. Note that in case the selected length is shorter than the label
sequence, the model will not be able to generate the whole target sequence.

\begin{figure}
  \centering
  \scalebox{0.7}{%
  \def\inputsize{7}

\begin{tikzpicture}[]

\draw (\inputsize / 2 + 0.1, -0.1) node {Input token embeddings};

\foreach \i in {0,...,\inputsize} {
	\draw (\i,-0.5) rectangle (\i+0.2,-1);
    \draw [->] (\i+0.1,-1) -- (\i+0.1, -1.25);
};

\draw (0, -1.25) rectangle (\inputsize + 0.2, -2.25);
\draw (\inputsize / 2 + 0.1, -1.75) node {Encoder};

\foreach \i in {0,...,\inputsize} {
	\draw [->] (\i+0.1,-2.25) -- (\i+0.1, -2.5);
    \draw[fill=yellow!40] (\i,-2.5) rectangle (\i+0.2,-3);

    \draw [->] (\i+0.1,-3) -- (\i+0.1, -3.25);
	\draw[fill=blue!40] (\i,-3.25) rectangle (\i+0.2,-3.75);
	\draw[fill=red!40] (\i,-3.75) rectangle (\i+0.2,-4.25);

    \draw [dashed,->] (\i+0.1,-4.25) -  - (\i+0.1, -4.75);
    \draw [dashed,->] (\i+0.2,-3.5) .. controls (\i + 0.6, -3.65) .. (\i+0.6, -4.75);

	\draw[fill=red!40] (\i,-4.75) rectangle (\i+0.2,-5.25);
	\draw[fill=blue!40] (\i + 0.5,-4.75) rectangle (\i+0.7,-5.25);

    \draw [->] (\i+0.1,-5.25) - - (\i+0.1, -5.5);
    \draw [->] (\i+0.6,-5.25) - - (\i+0.6, -5.5);
};

\draw (\inputsize + 1.2, -2.75) node {$\mathbf{h}$};
\draw (\inputsize + 1.2, -3.75) node {$W_{\text{spl}}\mathbf{h}$};
\draw (\inputsize + 1.2, -5.00) node {$\mathbf{s}$};

\draw (0, -5.5) rectangle (\inputsize + 0.7, -6.5);
\draw (\inputsize / 2 + 0.5 + 0.1, -6.0) node {Decoder};

\draw [fill=green!80!black!60] (\inputsize / 2 + 0.4,-7.2) circle [x radius=\inputsize / 2 + 0.4, y radius=0.5];
\draw (\inputsize / 2 + 0.6, -7.2) node {Connectionist Temporal Classification};

\foreach \i in {0,...,\inputsize} {
   \draw [->] (\i+0.1,-7.9) - - (\i+0.1, -8.15);
   \draw [->] (\i+0.6,-7.9) - - (\i+0.6, -8.15);
}

\draw  (0+0.1,-8.4) node {$w_1$};
\draw  (0+0.6,-8.4) node {$w_2$};
\draw  (1+0.1,-8.4) node {$w_3$};
\draw  (1+0.6,-8.4) node {$\varnothing$};
\draw  (2+0.1,-8.4) node {$w_4$};
\draw  (2+0.6,-8.4) node {$\varnothing$};
\draw  (3+0.1,-8.4) node {$w_5$};
\draw  (3+0.6,-8.4) node {$w_6$};
\draw  (4+0.1,-8.4) node {$\varnothing$};
\draw  (4+0.6,-8.4) node {$\varnothing$};
\draw  (5+0.1,-8.4) node {$\varnothing$};
\draw  (5+0.6,-8.4) node {$w_7$};
\draw  (6+0.1,-8.4) node {$w_8$};
\draw  (6+0.6,-8.4) node {$\varnothing$};
\draw  (7+0.1,-8.4) node {$w_9$};
\draw  (7+0.6,-8.4) node {$\varnothing$};

\draw (\inputsize / 2 + 0.3, -8.95) node {Output tokens / null symbols};

\end{tikzpicture}
}
  \caption{The scheme of the non-autoregressive architecture with
    state splitting and CTC. Image source: \citet{libovicky-helcl-2018-end},
    Figure 1.}%
  \label{fig:state-splitting}
\end{figure}


We implement the source-to-target length expansion by linear projections and
state splitting. This mechanism is illustrated in Figure
\ref{fig:state-splitting}. After a given Transformer layer completes its
computation, we linearly project the states
$h_1, \ldots, h_{T_x} \in \mathbb{R}^d$ into $\mathbb{R}^{kd}$. Then, we split
each of these projections into $k$ parts, which results in a $k$-times longer
sequence of states $s_1, \ldots, s_{k \cdot T_x}$ of the original dimension
$d$:
%
\begin{equation}
  s_{ck+b} = \left( W_{\text{spl}} h_c + b_{\text{spl}} \right)_{bd:(b+1)d}
\end{equation}
%
for $b=0 \ldots k-1$ and $c=1 \ldots T_x$ where
$W_{\text{spl}} \in \mathbb{R}^{d \times kd}$ and
$b_{\text{spl}} \in \mathbb{R}^{kd}$ are the trainable projection parameters.

We experiment with two placement options for the state-splitting layer. First,
we try placing the state splitting at the end of the Transformer layer
stack. In this scenario, there are 12 Transformer encoder layers, followed by
the state-splitting layer, whose outputs are used in the output projection. In
Table \ref{tab:end-to-end:bleu}, this variant is called ``Deep
encoder''. Second, we place the state-splitting layer in the middle of the
Transformer layer stack, mimicking the 6-layer encoder-decoder architecture of
the \ac{ar} Transformer model. We use the name ``Encoder-decoder'' for this
variant in Table \ref{tab:end-to-end:bleu}. In the second variant,
cross-attention can be included in the second half of the layers, which attends
to the states right after state splitting. We call this model ``Encoder-decoder
with pos. enc.'' in the results table.


% -----------------------------------------------------------------------------
\section{Preliminary Experiments}%
\label{sec:ctc:experiments}
% -----------------------------------------------------------------------------

We conducted experiments with the \ac{ctc}-based \ac{nat} models described
above on English--German and English--Romanian translation in both directions.

\paragraph{Data.}
In our experiments, we use the parallel data provided by the \ac{wmt}
organizers. For English--German, the training data consist of the Europarl
corpus \citep{koehn-2005-europarl}, News commentary
\citep{tiedemann-2012-parallel}, and Common
Crawl.\footnote{\url{https://commoncrawl.org/}} For validation, we use the
WMT~13 test set \citep{bojar-etal-2013-findings}, and we evaluate translation
quality on the WMT~14 test set \citep{bojar-etal-2014-findings}. For
English--Romanian, the data consist of the Europarl corpus and the SETIMES
corpus distributed by OPUS \citep{tiedemann-2012-parallel}. We use the
development and test set from the WMT~16 \citep{bojar-etal-2016-findings}.

We preprocess the data with scripts from the \texttt{mosesdecoder}
repository,\footnote{\url{https://github.com/moses-smt/mosesdecoder/}} a part
of the Moses translation toolkit \citep{koehn-etal-2007-moses}. Namely, we
normalize the punctuation in the data, then we use the tokenizer and a
truecaser. We segment the data using the wordpiece algorithm, creating a
vocabulary of approximately 41k wordpieces for English--German, and 38k
wordpieces for English--Romanian \citep{wu2016google}.

\paragraph{Models and Training.}
We implement and train our models in the Neural Monkey toolkit
\citep{helcl-libovicky-2017-neural,helcl-etal-2018-neural}. Neural Monkey is a
higher-level deep learning toolkit implemented in TensorFlow
\citep{tensorflow2015-whitepaper}, aimed at fast prototyping using
simple-format configuration files. We train all models for 10 epochs and select
the best-scoring model based on the validation \acs{bleu} score. Training of
the En--De models on a single Nvidia GeForce GTX 1080 GPU took approximately 4
weeks. Since the En--Ro parallel data was much smaller, training of the En--Ro
models took 4 days.

% We summarize the model and training settings in Table
% \ref{tab:end-to-end:hparams}. Note that the architecture resembles the
% Transformer big settings, but uses a smaller model dimension. We use the same
% settings for training models in both directions in both language pairs.

% \begin{table}
%   \centering
%   \begin{tabular}{lr}
%     \toprule
%     Parameter & Value \\
%     \midrule
%     No. of encoder layers & 6 \\
%     No. of decoder layers & 6 \\
%     Model dimension & 512 \\
%     Attention heads & 16 \\
%     Dropout probability & 0.1 \\
%     Feed-forward hidden size & 4,096 \\
%     State splitting factor & 3 \\
%     State splitting projection size & 1,536 \\
%     Vocabulary size & See Table \ref{tab:end-to-end:data} \\
%     \midrule
%     Optimizer method & adam \\
%     $\beta_1$ & 0.9 \\
%     $\beta_2$ & 0.997 \\
%     $\epsilon$ & 10\textsuperscript{-9} \\
%     Learning rate & 10\textsuperscript{-4} \\
%     Fixed batch size & 20 \\
%     Gradient clipping & 1 \\
%     \bottomrule
%   \end{tabular}

%   \caption{Experimental settings for the Neural Monkey experiments.}%
%   \label{tab:end-to-end:hparams}
% \end{table}


\paragraph{Results.} Table \ref{tab:end-to-end:bleu} compares the quantitative
results of our \ac{nar} models with the methods proposed by
\citet{gu2017nonautoregressive} and \citet{lee-etal-2018-deterministic}. We use
Sacrebleu \citep{post-2018-call} to compute the \acs{bleu} scores of the model
outputs.

\begin{table}
  \centering
  \begin{tabular}{lcccc}
    \toprule
     & \multicolumn{2}{c}{WMT~16} & \multicolumn{2}{c}{WMT~14} \\
     & En $\rightarrow$ Ro & Ro $\rightarrow$ En & En $\rightarrow$ De & De $\rightarrow$ En \\
    \midrule
    \citet{gu2017nonautoregressive} & & & & \\
    Autoregressive baseline & 31.35 & 31.03 & 22.71 & 26.39 \\
    \addlinespace
    NAT + FT & 27.29 & 29.06 & 17.69 & 21.47 \\
    NAT + FT + NPD (100 s) & 29.79 & 31.44 & 19.17 & 23.20 \\
    \midrule
    \citet{lee-etal-2018-deterministic} & & & & \\
    Autoregressive baseline & 31.93 & 31.55  & 23.77 & 28.15 \\
    \addlinespace
    1 iteration & 24.45 & 25.73 & 13.91 & 16.77 \\
    10 iterations & 29.32 & 30.19 & 21.61 & 25.48 \\
    \midrule
    \citet{libovicky-helcl-2018-end} & & & & \\
    Autoregressive baseline & 21.19 & 29.64 & 22.94 & 28.58 \\
    \addlinespace
    Deep encoder & 17.33 & 22.85 & 12.21 & 12.53 \\
    \quad + weight averaging & 18.47 & 24.68 & 14.65 & 16.72 \\
    \quad + beam search & 18.70 & 25.28 & 15.19 & 17.58 \\
    \addlinespace
    Encoder-decoder  & 18.51 & 22.37 & 13.29 & 17.98 \\
    \quad + weight averaging & 19.54 & 24.67 & 16.56 & 18.64 \\
    \quad + beam search & 19.81 & 25.21 & 17.09 & 18.80  \\
    \addlinespace
    Encoder-decoder with pos. enc. & 18.13 & 22.75 & 12.51 & 11.35 \\
    \quad + weight averaging & 19.31 & 24.21 & 17.37 & 18.07 \\
    \quad + beam search & 19.93 & 24.71 & 17.68 & 19.80 \\
    \bottomrule
  \end{tabular}

  \caption{Automatic evaluation of our \acs{ctc}-based approach, compared to
    the two of the first \acl{nar} methods, along with \acl{ar} greedy-decoding
    baseline scores. }%
  \label{tab:end-to-end:bleu}
\end{table}

The first model of \citet{gu2017nonautoregressive} represents the \ac{nat}
model with fine-tuning, which minimizes the KL divergence between the output
distributions of the teacher and student models. In the second row, \ac{npd}
with 100 samples is used. Using \ac{npd} can slow down the decoding because an
additional scoring step is needed using an \ac{ar} model. In theory, if enough
parallelism is available, this doubles the latency.

The approach of \citet{lee-etal-2018-deterministic} is iterative. We show the
performance of this method with 1 iteration and with 10 iterations. Note that
using more iterations slows the decoding speed of the iterative model.

The performance of the English--German \acs{ctc}-based models is similar to
single-pass models. However, the performance of the English--Romanian models is
poor. This is probably due to issues with Romanian diacritics in the data, as
suggested also by the poorer performance of our autoregressive baseline. Also,
the enhanced techniques (\ac{npd} and iterative refiniment) outperform our
method in all scenarios, but at a computational cost.

We also evaluate the effect of weight averaging and beam search. As opposed to
decoding from an \ac{ar} model, beam search in \acs{ctc}-based \ac{nar} models
operates only on the output distributions from the single decoding
step. Although in \ac{nar} models we are able to find the most likely sequence
by taking the argmax of the output distribution in each position, this way we
only find a single alignment. Using beam search takes into account more
alignments, and the sum of the alignment probabilities can be higher than the
probability of the sentence generated by argmax decoding. We see from Table
\ref{tab:end-to-end:bleu} that especially weight averaging brings a substantial
improvement of the translation quality.

The decoding speed for \ac{ar} and \ac{nar} models under different conditions
is shown in Table~\ref{tab:end-to-end:speed}. We report the average time to
decode a single sentence on the \ac{wmt}~15 test set
\citep{bojar-etal-2015-findings}, which consists of 2,169 sentence pairs. In
the CPU setting, we are using a CPU machine with the TensorFlow session
configured to use 12 CPU threads. GPU results are measured on a single Nvidia
GeForce GTX 1080 GPU. We show the average latency as well as the average time
to decode a sentence with batch decoding.

\begin{table}
  \centering

  \begin{tabular}{lrr}
    \toprule
     & \mc{CPU} & \mc{GPU} \\
    \midrule
    \acs{ar}, batch=1 & 2,247 ms & 1,129 ms \\
    \acs{ar}, batch=100 & & 127 ms\\
    % AR, beam=?
    \addlinespace
    \acs{nar}, batch=1 & 397 ms & 353 ms  \\
    \acs{nar}, batch=100 &  & 41 ms \\
    \bottomrule
  \end{tabular}

  \caption{The average times to decode one sentence under different conditions.}%
  \label{tab:end-to-end:speed}

\end{table}

Figure \ref{fig:end-to-end:speed} shows the decoding latencies on sentences
from the \ac{wmt}~15 test set \citep{bojar-etal-2015-findings}. As we can see
from the relationship between the source sentence length and the decoding time,
the \ac{nar} model exhibits a lower time complexity than the \ac{ar}
model. The latencies in the plot were measured on a CPU server using 12
threads. We use greedy decoding with a single sentence in a batch.

\begin{figure}
  \centering
  \begin{tikzpicture}[gnuplot]
%% generated with GNUPLOT 5.2p8 (Lua 5.3; terminal rev. Nov 2018, script rev. 108)
%% Thu 14 Oct 2021 08:30:26 PM BST
\path (0.000,0.000) rectangle (9.000,7.000);
\gpcolor{color=gp lt color border}
\gpsetlinetype{gp lt border}
\gpsetdashtype{gp dt solid}
\gpsetlinewidth{1.00}
\draw[gp path] (0.920,0.986)--(1.100,0.986);
\draw[gp path] (8.815,0.986)--(8.635,0.986);
\node[gp node right] at (0.736,0.986) {$0$};
\draw[gp path] (0.920,1.988)--(1.100,1.988);
\draw[gp path] (8.815,1.988)--(8.635,1.988);
\node[gp node right] at (0.736,1.988) {$1$};
\draw[gp path] (0.920,2.990)--(1.100,2.990);
\draw[gp path] (8.815,2.990)--(8.635,2.990);
\node[gp node right] at (0.736,2.990) {$2$};
\draw[gp path] (0.920,3.993)--(1.100,3.993);
\draw[gp path] (8.815,3.993)--(8.635,3.993);
\node[gp node right] at (0.736,3.993) {$3$};
\draw[gp path] (0.920,4.995)--(1.100,4.995);
\draw[gp path] (8.815,4.995)--(8.635,4.995);
\node[gp node right] at (0.736,4.995) {$4$};
\draw[gp path] (0.920,5.997)--(1.100,5.997);
\draw[gp path] (8.815,5.997)--(8.635,5.997);
\node[gp node right] at (0.736,5.997) {$5$};
\draw[gp path] (0.920,6.999)--(1.100,6.999);
\draw[gp path] (8.815,6.999)--(8.635,6.999);
\node[gp node right] at (0.736,6.999) {$6$};
\draw[gp path] (0.920,0.986)--(0.920,1.166);
\draw[gp path] (0.920,6.999)--(0.920,6.819);
\node[gp node center] at (0.920,0.678) {$0$};
\draw[gp path] (2.499,0.986)--(2.499,1.166);
\draw[gp path] (2.499,6.999)--(2.499,6.819);
\node[gp node center] at (2.499,0.678) {$20$};
\draw[gp path] (4.078,0.986)--(4.078,1.166);
\draw[gp path] (4.078,6.999)--(4.078,6.819);
\node[gp node center] at (4.078,0.678) {$40$};
\draw[gp path] (5.657,0.986)--(5.657,1.166);
\draw[gp path] (5.657,6.999)--(5.657,6.819);
\node[gp node center] at (5.657,0.678) {$60$};
\draw[gp path] (7.236,0.986)--(7.236,1.166);
\draw[gp path] (7.236,6.999)--(7.236,6.819);
\node[gp node center] at (7.236,0.678) {$80$};
\draw[gp path] (8.815,0.986)--(8.815,1.166);
\draw[gp path] (8.815,6.999)--(8.815,6.819);
\node[gp node center] at (8.815,0.678) {$100$};
\draw[gp path] (0.920,6.999)--(0.920,0.986)--(8.815,0.986)--(8.815,6.999)--cycle;
\node[gp node center,rotate=-270] at (-0.016,3.992) {Latency (seconds)};
\node[gp node center] at (4.867,0.216) {Source tokens};
\gpcolor{rgb color={0.800,0.000,0.000}}
\gpsetlinewidth{5.50}
\draw[gp path] (0.920,1.263)--(1.000,1.336)--(1.079,1.410)--(1.159,1.484)--(1.239,1.558)%
  --(1.319,1.632)--(1.398,1.706)--(1.478,1.780)--(1.558,1.853)--(1.638,1.927)--(1.717,2.001)%
  --(1.797,2.075)--(1.877,2.149)--(1.957,2.223)--(2.036,2.297)--(2.116,2.371)--(2.196,2.444)%
  --(2.276,2.518)--(2.355,2.592)--(2.435,2.666)--(2.515,2.740)--(2.595,2.814)--(2.674,2.888)%
  --(2.754,2.961)--(2.834,3.035)--(2.914,3.109)--(2.993,3.183)--(3.073,3.257)--(3.153,3.331)%
  --(3.233,3.405)--(3.312,3.478)--(3.392,3.552)--(3.472,3.626)--(3.552,3.700)--(3.631,3.774)%
  --(3.711,3.848)--(3.791,3.922)--(3.871,3.996)--(3.950,4.069)--(4.030,4.143)--(4.110,4.217)%
  --(4.190,4.291)--(4.269,4.365)--(4.349,4.439)--(4.429,4.513)--(4.509,4.586)--(4.588,4.660)%
  --(4.668,4.734)--(4.748,4.808)--(4.828,4.882)--(4.907,4.956)--(4.987,5.030)--(5.067,5.103)%
  --(5.147,5.177)--(5.226,5.251)--(5.306,5.325)--(5.386,5.399)--(5.466,5.473)--(5.545,5.547)%
  --(5.625,5.621)--(5.705,5.694)--(5.785,5.768)--(5.864,5.842)--(5.944,5.916)--(6.024,5.990)%
  --(6.104,6.064)--(6.183,6.138)--(6.263,6.211)--(6.343,6.285)--(6.423,6.359)--(6.502,6.433)%
  --(6.582,6.507)--(6.662,6.581)--(6.742,6.655)--(6.821,6.729)--(6.901,6.802)--(6.981,6.876)%
  --(7.061,6.950)--(7.113,6.999);
\gpcolor{rgb color={0.000,0.000,0.800}}
\draw[gp path] (0.920,1.215)--(1.000,1.221)--(1.079,1.227)--(1.159,1.233)--(1.239,1.239)%
  --(1.319,1.244)--(1.398,1.250)--(1.478,1.256)--(1.558,1.262)--(1.638,1.268)--(1.717,1.274)%
  --(1.797,1.280)--(1.877,1.286)--(1.957,1.292)--(2.036,1.298)--(2.116,1.304)--(2.196,1.310)%
  --(2.276,1.316)--(2.355,1.322)--(2.435,1.328)--(2.515,1.333)--(2.595,1.339)--(2.674,1.345)%
  --(2.754,1.351)--(2.834,1.357)--(2.914,1.363)--(2.993,1.369)--(3.073,1.375)--(3.153,1.381)%
  --(3.233,1.387)--(3.312,1.393)--(3.392,1.399)--(3.472,1.405)--(3.552,1.411)--(3.631,1.417)%
  --(3.711,1.422)--(3.791,1.428)--(3.871,1.434)--(3.950,1.440)--(4.030,1.446)--(4.110,1.452)%
  --(4.190,1.458)--(4.269,1.464)--(4.349,1.470)--(4.429,1.476)--(4.509,1.482)--(4.588,1.488)%
  --(4.668,1.494)--(4.748,1.500)--(4.828,1.506)--(4.907,1.511)--(4.987,1.517)--(5.067,1.523)%
  --(5.147,1.529)--(5.226,1.535)--(5.306,1.541)--(5.386,1.547)--(5.466,1.553)--(5.545,1.559)%
  --(5.625,1.565)--(5.705,1.571)--(5.785,1.577)--(5.864,1.583)--(5.944,1.589)--(6.024,1.594)%
  --(6.104,1.600)--(6.183,1.606)--(6.263,1.612)--(6.343,1.618)--(6.423,1.624)--(6.502,1.630)%
  --(6.582,1.636)--(6.662,1.642)--(6.742,1.648)--(6.821,1.654)--(6.901,1.660)--(6.981,1.666)%
  --(7.061,1.672)--(7.140,1.678)--(7.220,1.683)--(7.300,1.689)--(7.380,1.695)--(7.459,1.701)%
  --(7.539,1.707)--(7.619,1.713)--(7.699,1.719)--(7.778,1.725)--(7.858,1.731)--(7.938,1.737)%
  --(8.018,1.743)--(8.097,1.749)--(8.177,1.755)--(8.257,1.761)--(8.337,1.767)--(8.416,1.772)%
  --(8.496,1.778)--(8.576,1.784)--(8.656,1.790)--(8.735,1.796)--(8.815,1.802);
\gpcolor{color=gp lt color border}
\node[gp node right] at (8.083,4.223) {\acs{ar}};
\gpcolor{rgb color={0.867,0.000,0.000}}
\gpsetlinewidth{0.80}
\gpsetpointsize{4.00}
\gppoint{gp mark 3}{(1.552,2.944)}
\gppoint{gp mark 3}{(5.104,5.179)}
\gppoint{gp mark 3}{(3.131,3.343)}
\gppoint{gp mark 3}{(2.183,2.426)}
\gppoint{gp mark 3}{(2.262,2.860)}
\gppoint{gp mark 3}{(2.736,2.862)}
\gppoint{gp mark 3}{(4.078,4.814)}
\gppoint{gp mark 3}{(4.473,5.662)}
\gppoint{gp mark 3}{(3.446,3.635)}
\gppoint{gp mark 3}{(3.367,3.611)}
\gppoint{gp mark 3}{(3.131,3.330)}
\gppoint{gp mark 3}{(2.262,2.376)}
\gppoint{gp mark 3}{(2.025,2.338)}
\gppoint{gp mark 3}{(2.104,2.468)}
\gppoint{gp mark 3}{(2.104,2.427)}
\gppoint{gp mark 3}{(2.499,2.610)}
\gppoint{gp mark 3}{(3.525,3.064)}
\gppoint{gp mark 3}{(3.289,3.378)}
\gppoint{gp mark 3}{(3.210,3.444)}
\gppoint{gp mark 3}{(2.341,2.666)}
\gppoint{gp mark 3}{(2.183,2.268)}
\gppoint{gp mark 3}{(2.736,3.253)}
\gppoint{gp mark 3}{(4.710,6.120)}
\gppoint{gp mark 3}{(3.525,3.045)}
\gppoint{gp mark 3}{(3.841,4.057)}
\gppoint{gp mark 3}{(4.473,5.371)}
\gppoint{gp mark 3}{(2.104,2.466)}
\gppoint{gp mark 3}{(3.762,3.850)}
\gppoint{gp mark 3}{(4.315,5.573)}
\gppoint{gp mark 3}{(4.078,4.322)}
\gppoint{gp mark 3}{(5.183,5.537)}
\gppoint{gp mark 3}{(4.631,5.557)}
\gppoint{gp mark 3}{(5.025,5.763)}
\gppoint{gp mark 3}{(5.420,6.596)}
\gppoint{gp mark 3}{(3.052,3.635)}
\gppoint{gp mark 3}{(7.157,6.654)}
\gppoint{gp mark 3}{(2.183,2.611)}
\gppoint{gp mark 3}{(5.183,5.542)}
\gppoint{gp mark 3}{(4.552,5.229)}
\gppoint{gp mark 3}{(2.341,2.600)}
\gppoint{gp mark 3}{(1.394,1.760)}
\gppoint{gp mark 3}{(2.499,2.632)}
\gppoint{gp mark 3}{(3.446,3.726)}
\gppoint{gp mark 3}{(2.183,2.684)}
\gppoint{gp mark 3}{(1.552,1.850)}
\gppoint{gp mark 3}{(3.289,3.640)}
\gppoint{gp mark 3}{(2.973,3.125)}
\gppoint{gp mark 3}{(2.025,2.745)}
\gppoint{gp mark 3}{(3.604,4.848)}
\gppoint{gp mark 3}{(4.946,6.315)}
\gppoint{gp mark 3}{(3.210,3.118)}
\gppoint{gp mark 3}{(3.762,4.093)}
\gppoint{gp mark 3}{(2.578,2.908)}
\gppoint{gp mark 3}{(3.446,4.833)}
\gppoint{gp mark 3}{(6.999,6.565)}
\gppoint{gp mark 3}{(3.604,3.556)}
\gppoint{gp mark 3}{(3.525,3.849)}
\gppoint{gp mark 3}{(3.052,3.918)}
\gppoint{gp mark 3}{(4.078,5.196)}
\gppoint{gp mark 3}{(3.920,3.840)}
\gppoint{gp mark 3}{(2.973,3.290)}
\gppoint{gp mark 3}{(3.683,3.995)}
\gppoint{gp mark 3}{(3.367,4.688)}
\gppoint{gp mark 3}{(2.341,3.394)}
\gppoint{gp mark 3}{(3.052,3.897)}
\gppoint{gp mark 3}{(2.578,3.684)}
\gppoint{gp mark 3}{(3.525,4.390)}
\gppoint{gp mark 3}{(4.710,4.821)}
\gppoint{gp mark 3}{(2.104,2.527)}
\gppoint{gp mark 3}{(2.341,2.127)}
\gppoint{gp mark 3}{(4.157,4.015)}
\gppoint{gp mark 3}{(2.499,2.791)}
\gppoint{gp mark 3}{(3.131,3.106)}
\gppoint{gp mark 3}{(3.999,5.824)}
\gppoint{gp mark 3}{(3.683,4.417)}
\gppoint{gp mark 3}{(5.657,6.625)}
\gppoint{gp mark 3}{(3.920,4.043)}
\gppoint{gp mark 3}{(3.052,2.649)}
\gppoint{gp mark 3}{(3.525,3.645)}
\gppoint{gp mark 3}{(1.552,2.005)}
\gppoint{gp mark 3}{(6.999,6.345)}
\gppoint{gp mark 3}{(3.289,3.319)}
\gppoint{gp mark 3}{(5.104,6.225)}
\gppoint{gp mark 3}{(3.762,3.807)}
\gppoint{gp mark 3}{(3.210,3.795)}
\gppoint{gp mark 3}{(4.710,5.241)}
\gppoint{gp mark 3}{(5.025,5.915)}
\gppoint{gp mark 3}{(2.894,3.333)}
\gppoint{gp mark 3}{(3.367,3.564)}
\gppoint{gp mark 3}{(3.131,3.156)}
\gppoint{gp mark 3}{(3.210,3.442)}
\gppoint{gp mark 3}{(2.341,2.593)}
\gppoint{gp mark 3}{(2.894,2.737)}
\gppoint{gp mark 3}{(2.578,2.972)}
\gppoint{gp mark 3}{(1.473,1.879)}
\gppoint{gp mark 3}{(2.341,2.355)}
\gppoint{gp mark 3}{(3.052,2.832)}
\gppoint{gp mark 3}{(3.683,4.284)}
\gppoint{gp mark 3}{(2.499,2.711)}
\gppoint{gp mark 3}{(3.367,3.440)}
\gppoint{gp mark 3}{(3.525,3.704)}
\gppoint{gp mark 3}{(1.394,1.684)}
\gppoint{gp mark 3}{(1.867,2.397)}
\gppoint{gp mark 3}{(3.683,3.708)}
\gppoint{gp mark 3}{(3.446,3.659)}
\gppoint{gp mark 3}{(4.868,5.945)}
\gppoint{gp mark 3}{(3.683,4.462)}
\gppoint{gp mark 3}{(3.131,2.981)}
\gppoint{gp mark 3}{(2.104,2.743)}
\gppoint{gp mark 3}{(2.973,3.581)}
\gppoint{gp mark 3}{(3.841,4.105)}
\gppoint{gp mark 3}{(1.946,2.099)}
\gppoint{gp mark 3}{(1.788,1.928)}
\gppoint{gp mark 3}{(3.367,2.711)}
\gppoint{gp mark 3}{(2.025,1.893)}
\gppoint{gp mark 3}{(1.552,1.713)}
\gppoint{gp mark 3}{(1.946,1.975)}
\gppoint{gp mark 3}{(2.815,2.784)}
\gppoint{gp mark 3}{(3.052,3.539)}
\gppoint{gp mark 3}{(1.552,1.757)}
\gppoint{gp mark 3}{(4.157,4.178)}
\gppoint{gp mark 3}{(4.236,5.400)}
\gppoint{gp mark 3}{(5.025,5.146)}
\gppoint{gp mark 3}{(5.025,5.332)}
\gppoint{gp mark 3}{(3.525,4.143)}
\gppoint{gp mark 3}{(3.289,4.024)}
\gppoint{gp mark 3}{(3.841,3.512)}
\gppoint{gp mark 3}{(2.104,2.071)}
\gppoint{gp mark 3}{(1.394,1.708)}
\gppoint{gp mark 3}{(4.946,5.767)}
\gppoint{gp mark 3}{(4.473,5.009)}
\gppoint{gp mark 3}{(5.341,6.369)}
\gppoint{gp mark 3}{(5.736,6.023)}
\gppoint{gp mark 3}{(2.420,3.003)}
\gppoint{gp mark 3}{(3.210,2.205)}
\gppoint{gp mark 3}{(3.289,3.305)}
\gppoint{gp mark 3}{(3.920,4.366)}
\gppoint{gp mark 3}{(3.210,3.337)}
\gppoint{gp mark 3}{(3.052,3.570)}
\gppoint{gp mark 3}{(4.078,4.673)}
\gppoint{gp mark 3}{(2.736,2.428)}
\gppoint{gp mark 3}{(2.973,3.309)}
\gppoint{gp mark 3}{(3.604,3.451)}
\gppoint{gp mark 3}{(2.736,3.101)}
\gppoint{gp mark 3}{(1.946,2.072)}
\gppoint{gp mark 3}{(4.789,5.801)}
\gppoint{gp mark 3}{(4.315,4.257)}
\gppoint{gp mark 3}{(4.157,4.017)}
\gppoint{gp mark 3}{(2.104,2.221)}
\gppoint{gp mark 3}{(4.157,3.982)}
\gppoint{gp mark 3}{(4.078,3.748)}
\gppoint{gp mark 3}{(1.315,1.661)}
\gppoint{gp mark 3}{(3.604,3.847)}
\gppoint{gp mark 3}{(7.552,6.636)}
\gppoint{gp mark 3}{(1.236,1.564)}
\gppoint{gp mark 3}{(2.262,2.336)}
\gppoint{gp mark 3}{(3.131,2.492)}
\gppoint{gp mark 3}{(5.578,4.543)}
\gppoint{gp mark 3}{(3.762,3.266)}
\gppoint{gp mark 3}{(2.815,2.305)}
\gppoint{gp mark 3}{(4.946,5.147)}
\gppoint{gp mark 3}{(1.078,1.511)}
\gppoint{gp mark 3}{(3.210,3.651)}
\gppoint{gp mark 3}{(3.131,3.103)}
\gppoint{gp mark 3}{(1.631,2.152)}
\gppoint{gp mark 3}{(6.210,6.525)}
\gppoint{gp mark 3}{(3.683,3.502)}
\gppoint{gp mark 3}{(5.104,5.203)}
\gppoint{gp mark 3}{(4.157,4.412)}
\gppoint{gp mark 3}{(2.183,2.170)}
\gppoint{gp mark 3}{(4.157,4.219)}
\gppoint{gp mark 3}{(3.131,2.982)}
\gppoint{gp mark 3}{(3.052,3.685)}
\gppoint{gp mark 3}{(4.631,5.922)}
\gppoint{gp mark 3}{(4.552,5.776)}
\gppoint{gp mark 3}{(2.025,2.377)}
\gppoint{gp mark 3}{(2.025,1.886)}
\gppoint{gp mark 3}{(3.131,3.850)}
\gppoint{gp mark 3}{(3.289,3.827)}
\gppoint{gp mark 3}{(2.262,2.231)}
\gppoint{gp mark 3}{(5.657,6.239)}
\gppoint{gp mark 3}{(2.499,2.837)}
\gppoint{gp mark 3}{(2.183,2.276)}
\gppoint{gp mark 3}{(3.841,3.226)}
\gppoint{gp mark 3}{(3.210,3.582)}
\gppoint{gp mark 3}{(2.420,2.730)}
\gppoint{gp mark 3}{(3.210,4.165)}
\gppoint{gp mark 3}{(2.499,2.606)}
\gppoint{gp mark 3}{(3.841,4.587)}
\gppoint{gp mark 3}{(3.210,4.011)}
\gppoint{gp mark 3}{(5.499,5.924)}
\gppoint{gp mark 3}{(3.210,3.835)}
\gppoint{gp mark 3}{(7.315,4.959)}
\gppoint{gp mark 3}{(4.946,4.239)}
\gppoint{gp mark 3}{(4.946,5.662)}
\gppoint{gp mark 3}{(4.078,4.082)}
\gppoint{gp mark 3}{(2.815,3.296)}
\gppoint{gp mark 3}{(4.631,4.627)}
\gppoint{gp mark 3}{(3.525,3.580)}
\gppoint{gp mark 3}{(2.183,2.463)}
\gppoint{gp mark 3}{(1.631,1.818)}
\gppoint{gp mark 3}{(2.025,2.048)}
\gppoint{gp mark 3}{(2.657,2.088)}
\gppoint{gp mark 3}{(3.052,2.864)}
\gppoint{gp mark 3}{(2.499,2.425)}
\gppoint{gp mark 3}{(6.999,6.966)}
\gppoint{gp mark 3}{(2.499,2.818)}
\gppoint{gp mark 3}{(1.867,1.966)}
\gppoint{gp mark 3}{(4.473,4.515)}
\gppoint{gp mark 3}{(1.788,2.192)}
\gppoint{gp mark 3}{(2.894,3.169)}
\gppoint{gp mark 3}{(3.841,4.028)}
\gppoint{gp mark 3}{(3.289,3.483)}
\gppoint{gp mark 3}{(3.999,4.159)}
\gppoint{gp mark 3}{(2.657,2.903)}
\gppoint{gp mark 3}{(3.999,4.427)}
\gppoint{gp mark 3}{(3.131,3.302)}
\gppoint{gp mark 3}{(3.525,3.764)}
\gppoint{gp mark 3}{(4.236,4.758)}
\gppoint{gp mark 3}{(5.262,4.952)}
\gppoint{gp mark 3}{(5.183,5.952)}
\gppoint{gp mark 3}{(1.867,1.981)}
\gppoint{gp mark 3}{(2.815,2.990)}
\gppoint{gp mark 3}{(2.420,2.904)}
\gppoint{gp mark 3}{(3.762,4.119)}
\gppoint{gp mark 3}{(3.289,3.267)}
\gppoint{gp mark 3}{(2.657,2.335)}
\gppoint{gp mark 3}{(3.210,3.205)}
\gppoint{gp mark 3}{(4.315,4.664)}
\gppoint{gp mark 3}{(3.920,3.858)}
\gppoint{gp mark 3}{(1.236,1.626)}
\gppoint{gp mark 3}{(5.578,5.122)}
\gppoint{gp mark 3}{(6.525,5.554)}
\gppoint{gp mark 3}{(3.052,2.977)}
\gppoint{gp mark 3}{(6.131,6.022)}
\gppoint{gp mark 3}{(5.025,5.077)}
\gppoint{gp mark 3}{(2.736,2.482)}
\gppoint{gp mark 3}{(4.631,4.383)}
\gppoint{gp mark 3}{(2.657,2.851)}
\gppoint{gp mark 3}{(3.052,2.935)}
\gppoint{gp mark 3}{(1.631,1.846)}
\gppoint{gp mark 3}{(1.631,2.233)}
\gppoint{gp mark 3}{(1.710,2.295)}
\gppoint{gp mark 3}{(2.025,2.147)}
\gppoint{gp mark 3}{(4.078,4.263)}
\gppoint{gp mark 3}{(2.973,2.352)}
\gppoint{gp mark 3}{(3.920,4.356)}
\gppoint{gp mark 3}{(2.578,2.801)}
\gppoint{gp mark 3}{(6.052,6.778)}
\gppoint{gp mark 3}{(3.683,3.924)}
\gppoint{gp mark 3}{(3.920,3.508)}
\gppoint{gp mark 3}{(5.262,5.262)}
\gppoint{gp mark 3}{(2.815,2.798)}
\gppoint{gp mark 3}{(3.683,4.368)}
\gppoint{gp mark 3}{(1.946,2.098)}
\gppoint{gp mark 3}{(2.499,2.731)}
\gppoint{gp mark 3}{(2.499,2.523)}
\gppoint{gp mark 3}{(2.341,2.252)}
\gppoint{gp mark 3}{(3.604,3.145)}
\gppoint{gp mark 3}{(2.420,2.521)}
\gppoint{gp mark 3}{(4.631,4.419)}
\gppoint{gp mark 3}{(1.788,2.238)}
\gppoint{gp mark 3}{(3.446,3.371)}
\gppoint{gp mark 3}{(2.499,2.504)}
\gppoint{gp mark 3}{(3.210,3.436)}
\gppoint{gp mark 3}{(2.262,2.226)}
\gppoint{gp mark 3}{(4.394,4.599)}
\gppoint{gp mark 3}{(2.104,2.766)}
\gppoint{gp mark 3}{(1.946,2.086)}
\gppoint{gp mark 3}{(2.499,2.609)}
\gppoint{gp mark 3}{(3.131,2.841)}
\gppoint{gp mark 3}{(2.341,2.386)}
\gppoint{gp mark 3}{(2.657,2.579)}
\gppoint{gp mark 3}{(1.631,1.964)}
\gppoint{gp mark 3}{(3.446,3.307)}
\gppoint{gp mark 3}{(3.762,3.128)}
\gppoint{gp mark 3}{(2.815,3.420)}
\gppoint{gp mark 3}{(3.999,3.188)}
\gppoint{gp mark 3}{(3.131,3.060)}
\gppoint{gp mark 3}{(2.894,2.780)}
\gppoint{gp mark 3}{(2.262,2.481)}
\gppoint{gp mark 3}{(2.657,2.952)}
\gppoint{gp mark 3}{(4.473,4.237)}
\gppoint{gp mark 3}{(2.420,2.333)}
\gppoint{gp mark 3}{(3.683,3.326)}
\gppoint{gp mark 3}{(5.025,4.768)}
\gppoint{gp mark 3}{(4.789,4.646)}
\gppoint{gp mark 3}{(2.499,2.709)}
\gppoint{gp mark 3}{(3.604,3.797)}
\gppoint{gp mark 3}{(2.025,2.166)}
\gppoint{gp mark 3}{(3.525,3.882)}
\gppoint{gp mark 3}{(3.367,3.355)}
\gppoint{gp mark 3}{(5.341,6.181)}
\gppoint{gp mark 3}{(1.867,2.201)}
\gppoint{gp mark 3}{(5.420,5.768)}
\gppoint{gp mark 3}{(4.868,4.606)}
\gppoint{gp mark 3}{(4.631,4.492)}
\gppoint{gp mark 3}{(2.736,2.879)}
\gppoint{gp mark 3}{(6.447,6.654)}
\gppoint{gp mark 3}{(6.289,6.428)}
\gppoint{gp mark 3}{(5.104,5.737)}
\gppoint{gp mark 3}{(6.762,4.954)}
\gppoint{gp mark 3}{(2.341,1.977)}
\gppoint{gp mark 3}{(1.946,2.089)}
\gppoint{gp mark 3}{(1.788,2.008)}
\gppoint{gp mark 3}{(2.736,3.223)}
\gppoint{gp mark 3}{(2.499,2.504)}
\gppoint{gp mark 3}{(4.315,4.318)}
\gppoint{gp mark 3}{(2.578,2.675)}
\gppoint{gp mark 3}{(1.946,2.108)}
\gppoint{gp mark 3}{(3.131,2.728)}
\gppoint{gp mark 3}{(1.473,1.723)}
\gppoint{gp mark 3}{(2.578,2.936)}
\gppoint{gp mark 3}{(1.315,1.688)}
\gppoint{gp mark 3}{(1.078,1.569)}
\gppoint{gp mark 3}{(1.631,1.790)}
\gppoint{gp mark 3}{(5.183,5.202)}
\gppoint{gp mark 3}{(1.710,1.912)}
\gppoint{gp mark 3}{(2.657,2.640)}
\gppoint{gp mark 3}{(4.236,4.328)}
\gppoint{gp mark 3}{(4.552,3.600)}
\gppoint{gp mark 3}{(3.920,3.600)}
\gppoint{gp mark 3}{(3.683,3.990)}
\gppoint{gp mark 3}{(2.262,3.120)}
\gppoint{gp mark 3}{(2.736,2.849)}
\gppoint{gp mark 3}{(3.999,3.972)}
\gppoint{gp mark 3}{(3.052,3.200)}
\gppoint{gp mark 3}{(2.104,2.318)}
\gppoint{gp mark 3}{(1.473,1.875)}
\gppoint{gp mark 3}{(1.710,1.771)}
\gppoint{gp mark 3}{(1.315,1.621)}
\gppoint{gp mark 3}{(2.025,1.764)}
\gppoint{gp mark 3}{(1.710,1.724)}
\gppoint{gp mark 3}{(2.657,2.508)}
\gppoint{gp mark 3}{(1.788,1.736)}
\gppoint{gp mark 3}{(2.104,1.834)}
\gppoint{gp mark 3}{(1.788,1.735)}
\gppoint{gp mark 3}{(1.867,2.098)}
\gppoint{gp mark 3}{(1.710,1.873)}
\gppoint{gp mark 3}{(3.683,3.847)}
\gppoint{gp mark 3}{(2.894,3.139)}
\gppoint{gp mark 3}{(1.631,1.621)}
\gppoint{gp mark 3}{(1.710,1.788)}
\gppoint{gp mark 3}{(1.473,1.919)}
\gppoint{gp mark 3}{(3.367,3.142)}
\gppoint{gp mark 3}{(2.736,2.632)}
\gppoint{gp mark 3}{(2.183,2.215)}
\gppoint{gp mark 3}{(3.052,3.064)}
\gppoint{gp mark 3}{(2.025,2.246)}
\gppoint{gp mark 3}{(2.420,2.690)}
\gppoint{gp mark 3}{(4.315,4.222)}
\gppoint{gp mark 3}{(1.473,1.657)}
\gppoint{gp mark 3}{(4.789,5.428)}
\gppoint{gp mark 3}{(1.552,1.832)}
\gppoint{gp mark 3}{(4.631,5.146)}
\gppoint{gp mark 3}{(1.710,1.839)}
\gppoint{gp mark 3}{(2.736,2.206)}
\gppoint{gp mark 3}{(3.289,3.458)}
\gppoint{gp mark 3}{(4.552,4.536)}
\gppoint{gp mark 3}{(3.052,3.433)}
\gppoint{gp mark 3}{(4.157,5.036)}
\gppoint{gp mark 3}{(7.236,6.663)}
\gppoint{gp mark 3}{(2.657,2.704)}
\gppoint{gp mark 3}{(1.552,1.801)}
\gppoint{gp mark 3}{(1.788,2.041)}
\gppoint{gp mark 3}{(4.236,4.192)}
\gppoint{gp mark 3}{(1.710,1.833)}
\gppoint{gp mark 3}{(3.841,4.528)}
\gppoint{gp mark 3}{(3.210,2.418)}
\gppoint{gp mark 3}{(3.052,3.020)}
\gppoint{gp mark 3}{(2.815,2.658)}
\gppoint{gp mark 3}{(1.788,2.108)}
\gppoint{gp mark 3}{(2.341,2.164)}
\gppoint{gp mark 3}{(3.841,3.237)}
\gppoint{gp mark 3}{(2.736,3.041)}
\gppoint{gp mark 3}{(1.394,1.702)}
\gppoint{gp mark 3}{(2.341,2.340)}
\gppoint{gp mark 3}{(1.867,6.019)}
\gppoint{gp mark 3}{(2.815,2.193)}
\gppoint{gp mark 3}{(3.131,3.128)}
\gppoint{gp mark 3}{(2.815,2.975)}
\gppoint{gp mark 3}{(3.999,3.498)}
\gppoint{gp mark 3}{(2.341,2.845)}
\gppoint{gp mark 3}{(2.736,2.888)}
\gppoint{gp mark 3}{(1.710,1.680)}
\gppoint{gp mark 3}{(1.631,2.220)}
\gppoint{gp mark 3}{(2.815,2.445)}
\gppoint{gp mark 3}{(1.788,1.840)}
\gppoint{gp mark 3}{(1.631,2.064)}
\gppoint{gp mark 3}{(2.341,2.356)}
\gppoint{gp mark 3}{(4.236,4.345)}
\gppoint{gp mark 3}{(1.946,1.936)}
\gppoint{gp mark 3}{(1.867,1.728)}
\gppoint{gp mark 3}{(1.788,1.679)}
\gppoint{gp mark 3}{(1.710,1.960)}
\gppoint{gp mark 3}{(4.078,4.109)}
\gppoint{gp mark 3}{(4.078,4.168)}
\gppoint{gp mark 3}{(2.420,2.476)}
\gppoint{gp mark 3}{(3.367,2.806)}
\gppoint{gp mark 3}{(2.499,2.214)}
\gppoint{gp mark 3}{(4.552,4.421)}
\gppoint{gp mark 3}{(1.631,1.837)}
\gppoint{gp mark 3}{(1.236,1.519)}
\gppoint{gp mark 3}{(1.394,1.645)}
\gppoint{gp mark 3}{(1.631,1.617)}
\gppoint{gp mark 3}{(2.736,2.180)}
\gppoint{gp mark 3}{(1.236,1.592)}
\gppoint{gp mark 3}{(1.236,1.597)}
\gppoint{gp mark 3}{(1.867,2.342)}
\gppoint{gp mark 3}{(3.841,4.091)}
\gppoint{gp mark 3}{(3.762,3.999)}
\gppoint{gp mark 3}{(1.631,1.836)}
\gppoint{gp mark 3}{(2.025,2.236)}
\gppoint{gp mark 3}{(4.710,5.104)}
\gppoint{gp mark 3}{(2.578,2.760)}
\gppoint{gp mark 3}{(3.446,3.631)}
\gppoint{gp mark 3}{(2.815,2.943)}
\gppoint{gp mark 3}{(2.578,2.543)}
\gppoint{gp mark 3}{(2.736,2.739)}
\gppoint{gp mark 3}{(2.420,2.854)}
\gppoint{gp mark 3}{(3.683,3.378)}
\gppoint{gp mark 3}{(2.578,2.985)}
\gppoint{gp mark 3}{(2.341,2.664)}
\gppoint{gp mark 3}{(2.736,2.966)}
\gppoint{gp mark 3}{(4.473,4.889)}
\gppoint{gp mark 3}{(4.789,4.928)}
\gppoint{gp mark 3}{(3.604,4.396)}
\gppoint{gp mark 3}{(2.578,2.601)}
\gppoint{gp mark 3}{(7.789,6.582)}
\gppoint{gp mark 3}{(2.973,2.998)}
\gppoint{gp mark 3}{(6.052,6.723)}
\gppoint{gp mark 3}{(1.946,2.222)}
\gppoint{gp mark 3}{(3.999,3.913)}
\gppoint{gp mark 3}{(2.025,2.042)}
\gppoint{gp mark 3}{(2.104,2.357)}
\gppoint{gp mark 3}{(1.394,1.670)}
\gppoint{gp mark 3}{(2.262,2.495)}
\gppoint{gp mark 3}{(3.762,3.202)}
\gppoint{gp mark 3}{(2.815,2.907)}
\gppoint{gp mark 3}{(2.736,2.451)}
\gppoint{gp mark 3}{(2.183,2.040)}
\gppoint{gp mark 3}{(2.894,2.618)}
\gppoint{gp mark 3}{(3.210,3.198)}
\gppoint{gp mark 3}{(2.262,2.247)}
\gppoint{gp mark 3}{(1.946,1.957)}
\gppoint{gp mark 3}{(2.736,2.444)}
\gppoint{gp mark 3}{(2.262,2.079)}
\gppoint{gp mark 3}{(1.631,1.900)}
\gppoint{gp mark 3}{(5.262,5.281)}
\gppoint{gp mark 3}{(3.683,3.998)}
\gppoint{gp mark 3}{(3.131,3.247)}
\gppoint{gp mark 3}{(3.525,3.914)}
\gppoint{gp mark 3}{(3.446,3.416)}
\gppoint{gp mark 3}{(3.446,2.653)}
\gppoint{gp mark 3}{(2.578,2.698)}
\gppoint{gp mark 3}{(3.131,3.149)}
\gppoint{gp mark 3}{(3.210,2.605)}
\gppoint{gp mark 3}{(2.815,3.035)}
\gppoint{gp mark 3}{(4.078,4.210)}
\gppoint{gp mark 3}{(4.473,4.083)}
\gppoint{gp mark 3}{(7.315,6.465)}
\gppoint{gp mark 3}{(2.815,2.876)}
\gppoint{gp mark 3}{(6.289,6.453)}
\gppoint{gp mark 3}{(3.289,3.405)}
\gppoint{gp mark 3}{(4.236,4.188)}
\gppoint{gp mark 3}{(2.341,2.149)}
\gppoint{gp mark 3}{(3.052,3.083)}
\gppoint{gp mark 3}{(4.394,4.029)}
\gppoint{gp mark 3}{(3.210,2.841)}
\gppoint{gp mark 3}{(3.841,4.031)}
\gppoint{gp mark 3}{(1.867,2.389)}
\gppoint{gp mark 3}{(1.315,1.725)}
\gppoint{gp mark 3}{(3.367,3.290)}
\gppoint{gp mark 3}{(2.894,2.942)}
\gppoint{gp mark 3}{(6.525,6.575)}
\gppoint{gp mark 3}{(5.499,5.913)}
\gppoint{gp mark 3}{(3.920,3.810)}
\gppoint{gp mark 3}{(7.078,4.934)}
\gppoint{gp mark 3}{(6.604,6.797)}
\gppoint{gp mark 3}{(3.920,4.271)}
\gppoint{gp mark 3}{(3.131,2.906)}
\gppoint{gp mark 3}{(3.920,3.779)}
\gppoint{gp mark 3}{(1.788,1.926)}
\gppoint{gp mark 3}{(4.868,4.413)}
\gppoint{gp mark 3}{(6.525,6.332)}
\gppoint{gp mark 3}{(4.473,4.493)}
\gppoint{gp mark 3}{(4.946,5.780)}
\gppoint{gp mark 3}{(6.052,6.616)}
\gppoint{gp mark 3}{(2.262,2.724)}
\gppoint{gp mark 3}{(3.289,3.307)}
\gppoint{gp mark 3}{(3.525,3.189)}
\gppoint{gp mark 3}{(3.052,3.218)}
\gppoint{gp mark 3}{(2.025,2.151)}
\gppoint{gp mark 3}{(2.578,2.351)}
\gppoint{gp mark 3}{(5.973,5.578)}
\gppoint{gp mark 3}{(6.131,6.818)}
\gppoint{gp mark 3}{(2.578,2.897)}
\gppoint{gp mark 3}{(1.710,1.994)}
\gppoint{gp mark 3}{(3.683,3.318)}
\gppoint{gp mark 3}{(3.289,2.965)}
\gppoint{gp mark 3}{(4.236,4.137)}
\gppoint{gp mark 3}{(3.289,2.548)}
\gppoint{gp mark 3}{(3.131,2.597)}
\gppoint{gp mark 3}{(3.762,4.050)}
\gppoint{gp mark 3}{(1.552,1.948)}
\gppoint{gp mark 3}{(2.736,2.403)}
\gppoint{gp mark 3}{(2.973,3.553)}
\gppoint{gp mark 3}{(3.131,3.349)}
\gppoint{gp mark 3}{(4.473,4.871)}
\gppoint{gp mark 3}{(1.552,1.859)}
\gppoint{gp mark 3}{(2.736,2.919)}
\gppoint{gp mark 3}{(3.446,3.447)}
\gppoint{gp mark 3}{(3.210,3.716)}
\gppoint{gp mark 3}{(1.867,2.289)}
\gppoint{gp mark 3}{(1.867,1.965)}
\gppoint{gp mark 3}{(1.552,1.769)}
\gppoint{gp mark 3}{(3.210,2.528)}
\gppoint{gp mark 3}{(3.604,3.408)}
\gppoint{gp mark 3}{(2.815,2.830)}
\gppoint{gp mark 3}{(3.131,3.394)}
\gppoint{gp mark 3}{(2.341,2.753)}
\gppoint{gp mark 3}{(3.289,3.204)}
\gppoint{gp mark 3}{(1.867,1.835)}
\gppoint{gp mark 3}{(3.604,3.982)}
\gppoint{gp mark 3}{(3.762,3.929)}
\gppoint{gp mark 3}{(2.341,2.106)}
\gppoint{gp mark 3}{(1.710,1.727)}
\gppoint{gp mark 3}{(4.157,4.342)}
\gppoint{gp mark 3}{(1.788,2.266)}
\gppoint{gp mark 3}{(2.578,2.308)}
\gppoint{gp mark 3}{(3.131,3.230)}
\gppoint{gp mark 3}{(4.552,4.720)}
\gppoint{gp mark 3}{(2.973,2.855)}
\gppoint{gp mark 3}{(3.920,4.729)}
\gppoint{gp mark 3}{(1.710,1.719)}
\gppoint{gp mark 3}{(3.920,3.431)}
\gppoint{gp mark 3}{(3.604,4.038)}
\gppoint{gp mark 3}{(3.289,3.186)}
\gppoint{gp mark 3}{(3.446,3.389)}
\gppoint{gp mark 3}{(3.762,2.905)}
\gppoint{gp mark 3}{(3.525,3.753)}
\gppoint{gp mark 3}{(2.183,2.088)}
\gppoint{gp mark 3}{(2.104,2.207)}
\gppoint{gp mark 3}{(2.341,2.405)}
\gppoint{gp mark 3}{(5.183,5.642)}
\gppoint{gp mark 3}{(2.973,2.843)}
\gppoint{gp mark 3}{(2.025,2.195)}
\gppoint{gp mark 3}{(2.973,2.813)}
\gppoint{gp mark 3}{(1.867,2.374)}
\gppoint{gp mark 3}{(3.210,3.212)}
\gppoint{gp mark 3}{(3.999,4.172)}
\gppoint{gp mark 3}{(4.078,5.556)}
\gppoint{gp mark 3}{(5.420,5.516)}
\gppoint{gp mark 3}{(1.710,1.997)}
\gppoint{gp mark 3}{(2.499,2.380)}
\gppoint{gp mark 3}{(3.131,3.274)}
\gppoint{gp mark 3}{(2.183,2.447)}
\gppoint{gp mark 3}{(5.104,5.372)}
\gppoint{gp mark 3}{(4.946,5.357)}
\gppoint{gp mark 3}{(2.578,2.490)}
\gppoint{gp mark 3}{(2.341,2.219)}
\gppoint{gp mark 3}{(5.973,6.802)}
\gppoint{gp mark 3}{(2.104,2.382)}
\gppoint{gp mark 3}{(3.841,3.838)}
\gppoint{gp mark 3}{(4.552,5.553)}
\gppoint{gp mark 3}{(6.368,6.766)}
\gppoint{gp mark 3}{(2.894,2.943)}
\gppoint{gp mark 3}{(4.473,4.311)}
\gppoint{gp mark 3}{(4.236,3.739)}
\gppoint{gp mark 3}{(5.499,6.268)}
\gppoint{gp mark 3}{(5.104,6.725)}
\gppoint{gp mark 3}{(3.446,3.903)}
\gppoint{gp mark 3}{(3.762,4.414)}
\gppoint{gp mark 3}{(4.394,4.877)}
\gppoint{gp mark 3}{(3.841,4.068)}
\gppoint{gp mark 3}{(4.157,4.824)}
\gppoint{gp mark 3}{(3.999,4.048)}
\gppoint{gp mark 3}{(3.683,3.714)}
\gppoint{gp mark 3}{(4.315,3.635)}
\gppoint{gp mark 3}{(2.815,2.437)}
\gppoint{gp mark 3}{(2.262,2.818)}
\gppoint{gp mark 3}{(2.104,2.586)}
\gppoint{gp mark 3}{(2.104,2.140)}
\gppoint{gp mark 3}{(4.631,5.305)}
\gppoint{gp mark 3}{(4.315,4.752)}
\gppoint{gp mark 3}{(2.815,3.004)}
\gppoint{gp mark 3}{(4.315,4.377)}
\gppoint{gp mark 3}{(4.315,4.597)}
\gppoint{gp mark 3}{(5.499,6.574)}
\gppoint{gp mark 3}{(2.499,2.544)}
\gppoint{gp mark 3}{(3.131,3.799)}
\gppoint{gp mark 3}{(3.367,3.370)}
\gppoint{gp mark 3}{(3.367,3.520)}
\gppoint{gp mark 3}{(3.367,3.071)}
\gppoint{gp mark 3}{(1.473,1.790)}
\gppoint{gp mark 3}{(1.394,1.845)}
\gppoint{gp mark 3}{(1.946,2.070)}
\gppoint{gp mark 3}{(4.394,4.593)}
\gppoint{gp mark 3}{(4.315,4.377)}
\gppoint{gp mark 3}{(4.473,4.938)}
\gppoint{gp mark 3}{(1.788,1.845)}
\gppoint{gp mark 3}{(5.025,5.979)}
\gppoint{gp mark 3}{(3.210,2.704)}
\gppoint{gp mark 3}{(3.604,3.411)}
\gppoint{gp mark 3}{(1.473,1.815)}
\gppoint{gp mark 3}{(1.710,1.716)}
\gppoint{gp mark 3}{(2.578,2.608)}
\gppoint{gp mark 3}{(1.867,1.979)}
\gppoint{gp mark 3}{(2.262,2.342)}
\gppoint{gp mark 3}{(5.973,6.811)}
\gppoint{gp mark 3}{(3.920,4.071)}
\gppoint{gp mark 3}{(1.788,1.720)}
\gppoint{gp mark 3}{(3.762,4.051)}
\gppoint{gp mark 3}{(2.420,2.581)}
\gppoint{gp mark 3}{(1.946,2.200)}
\gppoint{gp mark 3}{(2.420,2.474)}
\gppoint{gp mark 3}{(2.894,3.313)}
\gppoint{gp mark 3}{(2.104,2.153)}
\gppoint{gp mark 3}{(1.867,2.184)}
\gppoint{gp mark 3}{(1.552,1.860)}
\gppoint{gp mark 3}{(1.552,1.841)}
\gppoint{gp mark 3}{(2.420,2.208)}
\gppoint{gp mark 3}{(1.631,1.831)}
\gppoint{gp mark 3}{(2.183,2.396)}
\gppoint{gp mark 3}{(2.420,2.911)}
\gppoint{gp mark 3}{(1.394,1.553)}
\gppoint{gp mark 3}{(2.262,2.151)}
\gppoint{gp mark 3}{(1.631,1.870)}
\gppoint{gp mark 3}{(4.157,3.809)}
\gppoint{gp mark 3}{(2.815,2.326)}
\gppoint{gp mark 3}{(2.894,3.026)}
\gppoint{gp mark 3}{(1.552,1.926)}
\gppoint{gp mark 3}{(3.604,4.184)}
\gppoint{gp mark 3}{(3.683,3.301)}
\gppoint{gp mark 3}{(3.367,4.131)}
\gppoint{gp mark 3}{(2.578,3.139)}
\gppoint{gp mark 3}{(1.473,1.739)}
\gppoint{gp mark 3}{(3.604,4.229)}
\gppoint{gp mark 3}{(1.473,1.757)}
\gppoint{gp mark 3}{(2.578,2.363)}
\gppoint{gp mark 3}{(1.710,2.019)}
\gppoint{gp mark 3}{(2.025,2.289)}
\gppoint{gp mark 3}{(2.736,2.553)}
\gppoint{gp mark 3}{(2.736,2.575)}
\gppoint{gp mark 3}{(2.025,1.802)}
\gppoint{gp mark 3}{(1.710,1.848)}
\gppoint{gp mark 3}{(3.289,3.111)}
\gppoint{gp mark 3}{(1.473,1.819)}
\gppoint{gp mark 3}{(2.262,2.341)}
\gppoint{gp mark 3}{(2.578,2.522)}
\gppoint{gp mark 3}{(2.736,2.852)}
\gppoint{gp mark 3}{(1.473,1.678)}
\gppoint{gp mark 3}{(2.341,2.464)}
\gppoint{gp mark 3}{(2.657,2.052)}
\gppoint{gp mark 3}{(1.473,1.856)}
\gppoint{gp mark 3}{(1.394,1.828)}
\gppoint{gp mark 3}{(1.236,1.719)}
\gppoint{gp mark 3}{(1.394,1.864)}
\gppoint{gp mark 3}{(1.631,2.049)}
\gppoint{gp mark 3}{(2.104,2.610)}
\gppoint{gp mark 3}{(1.236,1.587)}
\gppoint{gp mark 3}{(1.946,1.956)}
\gppoint{gp mark 3}{(1.867,1.832)}
\gppoint{gp mark 3}{(2.736,2.921)}
\gppoint{gp mark 3}{(3.131,3.167)}
\gppoint{gp mark 3}{(1.788,1.765)}
\gppoint{gp mark 3}{(1.315,1.616)}
\gppoint{gp mark 3}{(1.631,1.753)}
\gppoint{gp mark 3}{(4.157,4.875)}
\gppoint{gp mark 3}{(1.631,1.854)}
\gppoint{gp mark 3}{(1.552,1.822)}
\gppoint{gp mark 3}{(1.315,1.645)}
\gppoint{gp mark 3}{(3.604,3.746)}
\gppoint{gp mark 3}{(2.104,2.276)}
\gppoint{gp mark 3}{(2.736,3.141)}
\gppoint{gp mark 3}{(3.525,3.410)}
\gppoint{gp mark 3}{(1.710,1.744)}
\gppoint{gp mark 3}{(3.762,4.008)}
\gppoint{gp mark 3}{(1.867,1.883)}
\gppoint{gp mark 3}{(4.552,4.496)}
\gppoint{gp mark 3}{(2.025,2.097)}
\gppoint{gp mark 3}{(6.762,6.926)}
\gppoint{gp mark 3}{(1.157,1.540)}
\gppoint{gp mark 3}{(2.736,2.255)}
\gppoint{gp mark 3}{(5.025,6.271)}
\gppoint{gp mark 3}{(3.052,3.022)}
\gppoint{gp mark 3}{(2.262,2.695)}
\gppoint{gp mark 3}{(2.025,2.105)}
\gppoint{gp mark 3}{(2.183,2.260)}
\gppoint{gp mark 3}{(1.946,1.814)}
\gppoint{gp mark 3}{(1.394,1.685)}
\gppoint{gp mark 3}{(3.841,3.671)}
\gppoint{gp mark 3}{(1.552,1.850)}
\gppoint{gp mark 3}{(2.815,2.847)}
\gppoint{gp mark 3}{(5.578,6.811)}
\gppoint{gp mark 3}{(2.025,2.318)}
\gppoint{gp mark 3}{(8.262,6.643)}
\gppoint{gp mark 3}{(1.867,2.115)}
\gppoint{gp mark 3}{(2.657,2.733)}
\gppoint{gp mark 3}{(8.657,6.500)}
\gppoint{gp mark 3}{(3.210,3.305)}
\gppoint{gp mark 3}{(1.631,2.023)}
\gppoint{gp mark 3}{(1.788,2.041)}
\gppoint{gp mark 3}{(1.946,2.186)}
\gppoint{gp mark 3}{(2.815,2.889)}
\gppoint{gp mark 3}{(1.394,1.712)}
\gppoint{gp mark 3}{(1.867,2.132)}
\gppoint{gp mark 3}{(1.236,1.696)}
\gppoint{gp mark 3}{(1.236,1.614)}
\gppoint{gp mark 3}{(4.473,4.255)}
\gppoint{gp mark 3}{(1.867,1.857)}
\gppoint{gp mark 3}{(2.104,2.232)}
\gppoint{gp mark 3}{(3.210,3.187)}
\gppoint{gp mark 3}{(4.789,5.011)}
\gppoint{gp mark 3}{(4.394,4.405)}
\gppoint{gp mark 3}{(2.183,2.518)}
\gppoint{gp mark 3}{(3.683,3.432)}
\gppoint{gp mark 3}{(3.999,4.447)}
\gppoint{gp mark 3}{(2.657,2.917)}
\gppoint{gp mark 3}{(6.131,6.565)}
\gppoint{gp mark 3}{(4.236,4.310)}
\gppoint{gp mark 3}{(2.420,2.687)}
\gppoint{gp mark 3}{(2.104,2.207)}
\gppoint{gp mark 3}{(1.631,1.715)}
\gppoint{gp mark 3}{(4.078,4.743)}
\gppoint{gp mark 3}{(4.157,4.524)}
\gppoint{gp mark 3}{(4.394,4.913)}
\gppoint{gp mark 3}{(2.657,2.719)}
\gppoint{gp mark 3}{(2.183,2.079)}
\gppoint{gp mark 3}{(4.236,4.307)}
\gppoint{gp mark 3}{(1.394,1.843)}
\gppoint{gp mark 3}{(2.183,2.295)}
\gppoint{gp mark 3}{(1.236,1.731)}
\gppoint{gp mark 3}{(1.867,2.089)}
\gppoint{gp mark 3}{(3.841,3.964)}
\gppoint{gp mark 3}{(1.946,2.216)}
\gppoint{gp mark 3}{(1.710,1.861)}
\gppoint{gp mark 3}{(1.631,1.894)}
\gppoint{gp mark 3}{(1.867,2.176)}
\gppoint{gp mark 3}{(2.420,2.463)}
\gppoint{gp mark 3}{(4.789,5.302)}
\gppoint{gp mark 3}{(1.236,1.622)}
\gppoint{gp mark 3}{(2.025,2.244)}
\gppoint{gp mark 3}{(3.525,4.983)}
\gppoint{gp mark 3}{(3.920,3.961)}
\gppoint{gp mark 3}{(2.499,3.306)}
\gppoint{gp mark 3}{(4.710,5.392)}
\gppoint{gp mark 3}{(6.762,6.680)}
\gppoint{gp mark 3}{(3.289,3.038)}
\gppoint{gp mark 3}{(5.183,5.987)}
\gppoint{gp mark 3}{(2.657,3.108)}
\gppoint{gp mark 3}{(3.210,3.014)}
\gppoint{gp mark 3}{(2.578,3.037)}
\gppoint{gp mark 3}{(2.973,3.442)}
\gppoint{gp mark 3}{(2.183,2.587)}
\gppoint{gp mark 3}{(1.710,1.781)}
\gppoint{gp mark 3}{(3.367,3.086)}
\gppoint{gp mark 3}{(4.552,4.488)}
\gppoint{gp mark 3}{(1.473,1.949)}
\gppoint{gp mark 3}{(2.420,2.669)}
\gppoint{gp mark 3}{(2.341,2.272)}
\gppoint{gp mark 3}{(3.999,3.716)}
\gppoint{gp mark 3}{(4.394,5.194)}
\gppoint{gp mark 3}{(5.025,5.219)}
\gppoint{gp mark 3}{(3.683,3.236)}
\gppoint{gp mark 3}{(3.289,2.650)}
\gppoint{gp mark 3}{(2.736,2.876)}
\gppoint{gp mark 3}{(3.920,4.291)}
\gppoint{gp mark 3}{(3.052,3.141)}
\gppoint{gp mark 3}{(3.289,2.836)}
\gppoint{gp mark 3}{(5.025,6.139)}
\gppoint{gp mark 3}{(4.394,5.593)}
\gppoint{gp mark 3}{(3.920,4.052)}
\gppoint{gp mark 3}{(4.315,5.087)}
\gppoint{gp mark 3}{(3.604,3.988)}
\gppoint{gp mark 3}{(2.973,3.416)}
\gppoint{gp mark 3}{(3.289,3.172)}
\gppoint{gp mark 3}{(2.657,3.189)}
\gppoint{gp mark 3}{(4.157,5.057)}
\gppoint{gp mark 3}{(3.762,4.198)}
\gppoint{gp mark 3}{(3.210,3.956)}
\gppoint{gp mark 3}{(3.289,3.956)}
\gppoint{gp mark 3}{(2.973,3.543)}
\gppoint{gp mark 3}{(1.236,1.581)}
\gppoint{gp mark 3}{(2.499,2.843)}
\gppoint{gp mark 3}{(2.104,2.174)}
\gppoint{gp mark 3}{(3.999,4.624)}
\gppoint{gp mark 3}{(2.578,2.702)}
\gppoint{gp mark 3}{(1.946,2.249)}
\gppoint{gp mark 3}{(3.525,3.480)}
\gppoint{gp mark 3}{(1.552,1.847)}
\gppoint{gp mark 3}{(2.104,2.667)}
\gppoint{gp mark 3}{(1.710,2.001)}
\gppoint{gp mark 3}{(3.446,3.841)}
\gppoint{gp mark 3}{(1.552,1.939)}
\gppoint{gp mark 3}{(1.867,2.320)}
\gppoint{gp mark 3}{(2.736,3.015)}
\gppoint{gp mark 3}{(2.973,3.086)}
\gppoint{gp mark 3}{(2.499,2.266)}
\gppoint{gp mark 3}{(2.183,2.571)}
\gppoint{gp mark 3}{(2.025,2.120)}
\gppoint{gp mark 3}{(1.394,1.711)}
\gppoint{gp mark 3}{(3.131,2.522)}
\gppoint{gp mark 3}{(2.183,2.604)}
\gppoint{gp mark 3}{(2.894,2.721)}
\gppoint{gp mark 3}{(1.552,1.790)}
\gppoint{gp mark 3}{(2.262,2.449)}
\gppoint{gp mark 3}{(3.367,3.930)}
\gppoint{gp mark 3}{(4.157,4.903)}
\gppoint{gp mark 3}{(2.420,2.824)}
\gppoint{gp mark 3}{(3.604,4.049)}
\gppoint{gp mark 3}{(3.367,3.713)}
\gppoint{gp mark 3}{(1.631,2.123)}
\gppoint{gp mark 3}{(2.104,2.225)}
\gppoint{gp mark 3}{(4.157,3.774)}
\gppoint{gp mark 3}{(3.210,3.506)}
\gppoint{gp mark 3}{(2.973,2.381)}
\gppoint{gp mark 3}{(3.289,3.274)}
\gppoint{gp mark 3}{(2.341,2.420)}
\gppoint{gp mark 3}{(1.710,1.689)}
\gppoint{gp mark 3}{(1.631,1.812)}
\gppoint{gp mark 3}{(5.341,5.604)}
\gppoint{gp mark 3}{(3.446,3.419)}
\gppoint{gp mark 3}{(1.946,2.130)}
\gppoint{gp mark 3}{(2.183,2.412)}
\gppoint{gp mark 3}{(2.341,2.512)}
\gppoint{gp mark 3}{(3.367,2.960)}
\gppoint{gp mark 3}{(2.262,2.393)}
\gppoint{gp mark 3}{(1.631,1.935)}
\gppoint{gp mark 3}{(1.552,1.802)}
\gppoint{gp mark 3}{(3.446,4.354)}
\gppoint{gp mark 3}{(2.657,3.190)}
\gppoint{gp mark 3}{(2.420,2.502)}
\gppoint{gp mark 3}{(3.999,4.838)}
\gppoint{gp mark 3}{(1.473,1.821)}
\gppoint{gp mark 3}{(2.736,3.217)}
\gppoint{gp mark 3}{(2.973,2.969)}
\gppoint{gp mark 3}{(3.367,3.801)}
\gppoint{gp mark 3}{(2.025,2.166)}
\gppoint{gp mark 3}{(1.867,2.296)}
\gppoint{gp mark 3}{(2.657,2.845)}
\gppoint{gp mark 3}{(3.131,3.204)}
\gppoint{gp mark 3}{(3.920,5.002)}
\gppoint{gp mark 3}{(2.578,2.768)}
\gppoint{gp mark 3}{(3.525,3.689)}
\gppoint{gp mark 3}{(1.946,2.071)}
\gppoint{gp mark 3}{(2.499,2.602)}
\gppoint{gp mark 3}{(2.657,2.716)}
\gppoint{gp mark 3}{(2.973,3.144)}
\gppoint{gp mark 3}{(1.710,1.982)}
\gppoint{gp mark 3}{(2.894,3.359)}
\gppoint{gp mark 3}{(2.025,2.284)}
\gppoint{gp mark 3}{(1.867,1.926)}
\gppoint{gp mark 3}{(2.262,2.683)}
\gppoint{gp mark 3}{(2.736,3.157)}
\gppoint{gp mark 3}{(2.499,2.366)}
\gppoint{gp mark 3}{(3.762,3.869)}
\gppoint{gp mark 3}{(3.525,4.007)}
\gppoint{gp mark 3}{(1.552,1.798)}
\gppoint{gp mark 3}{(4.236,4.385)}
\gppoint{gp mark 3}{(3.052,2.929)}
\gppoint{gp mark 3}{(3.367,3.911)}
\gppoint{gp mark 3}{(2.499,2.916)}
\gppoint{gp mark 3}{(4.078,4.535)}
\gppoint{gp mark 3}{(1.788,2.017)}
\gppoint{gp mark 3}{(1.788,2.073)}
\gppoint{gp mark 3}{(1.473,1.761)}
\gppoint{gp mark 3}{(2.420,2.931)}
\gppoint{gp mark 3}{(3.525,4.139)}
\gppoint{gp mark 3}{(3.131,3.680)}
\gppoint{gp mark 3}{(2.341,2.316)}
\gppoint{gp mark 3}{(3.525,3.662)}
\gppoint{gp mark 3}{(4.078,4.466)}
\gppoint{gp mark 3}{(2.341,2.391)}
\gppoint{gp mark 3}{(1.552,1.750)}
\gppoint{gp mark 3}{(2.262,2.415)}
\gppoint{gp mark 3}{(1.946,2.299)}
\gppoint{gp mark 3}{(3.131,3.644)}
\gppoint{gp mark 3}{(1.631,2.041)}
\gppoint{gp mark 3}{(4.157,6.675)}
\gppoint{gp mark 3}{(2.894,3.217)}
\gppoint{gp mark 3}{(2.025,2.314)}
\gppoint{gp mark 3}{(2.341,2.553)}
\gppoint{gp mark 3}{(2.736,2.627)}
\gppoint{gp mark 3}{(2.657,2.759)}
\gppoint{gp mark 3}{(2.736,2.645)}
\gppoint{gp mark 3}{(2.262,2.275)}
\gppoint{gp mark 3}{(1.946,2.423)}
\gppoint{gp mark 3}{(1.867,2.129)}
\gppoint{gp mark 3}{(1.552,2.050)}
\gppoint{gp mark 3}{(1.631,2.108)}
\gppoint{gp mark 3}{(3.920,3.440)}
\gppoint{gp mark 3}{(1.552,1.966)}
\gppoint{gp mark 3}{(2.104,1.956)}
\gppoint{gp mark 3}{(1.710,2.275)}
\gppoint{gp mark 3}{(2.736,2.771)}
\gppoint{gp mark 3}{(2.262,2.562)}
\gppoint{gp mark 3}{(2.025,2.278)}
\gppoint{gp mark 3}{(2.104,2.733)}
\gppoint{gp mark 3}{(1.788,2.235)}
\gppoint{gp mark 3}{(2.183,2.451)}
\gppoint{gp mark 3}{(2.341,2.382)}
\gppoint{gp mark 3}{(2.025,2.490)}
\gppoint{gp mark 3}{(1.315,1.651)}
\gppoint{gp mark 3}{(2.183,2.528)}
\gppoint{gp mark 3}{(1.710,2.076)}
\gppoint{gp mark 3}{(1.710,1.952)}
\gppoint{gp mark 3}{(2.183,2.198)}
\gppoint{gp mark 3}{(1.946,2.093)}
\gppoint{gp mark 3}{(1.552,1.743)}
\gppoint{gp mark 3}{(2.341,2.751)}
\gppoint{gp mark 3}{(3.367,3.745)}
\gppoint{gp mark 3}{(3.446,3.577)}
\gppoint{gp mark 3}{(1.788,2.159)}
\gppoint{gp mark 3}{(4.789,5.422)}
\gppoint{gp mark 3}{(2.025,2.302)}
\gppoint{gp mark 3}{(2.815,2.920)}
\gppoint{gp mark 3}{(3.289,3.307)}
\gppoint{gp mark 3}{(1.473,1.687)}
\gppoint{gp mark 3}{(2.578,2.675)}
\gppoint{gp mark 3}{(1.236,1.649)}
\gppoint{gp mark 3}{(3.446,2.993)}
\gppoint{gp mark 3}{(3.289,3.096)}
\gppoint{gp mark 3}{(1.946,2.296)}
\gppoint{gp mark 3}{(2.657,2.711)}
\gppoint{gp mark 3}{(1.631,2.050)}
\gppoint{gp mark 3}{(2.262,2.491)}
\gppoint{gp mark 3}{(2.815,3.363)}
\gppoint{gp mark 3}{(2.341,2.969)}
\gppoint{gp mark 3}{(2.104,2.308)}
\gppoint{gp mark 3}{(1.473,1.734)}
\gppoint{gp mark 3}{(2.657,2.705)}
\gppoint{gp mark 3}{(2.499,2.641)}
\gppoint{gp mark 3}{(2.341,2.277)}
\gppoint{gp mark 3}{(2.420,2.592)}
\gppoint{gp mark 3}{(1.946,2.207)}
\gppoint{gp mark 3}{(2.025,2.563)}
\gppoint{gp mark 3}{(1.867,1.939)}
\gppoint{gp mark 3}{(2.183,2.145)}
\gppoint{gp mark 3}{(1.946,2.025)}
\gppoint{gp mark 3}{(2.894,2.895)}
\gppoint{gp mark 3}{(1.315,1.624)}
\gppoint{gp mark 3}{(2.104,2.373)}
\gppoint{gp mark 3}{(2.183,2.480)}
\gppoint{gp mark 3}{(2.815,2.514)}
\gppoint{gp mark 3}{(1.867,2.072)}
\gppoint{gp mark 3}{(2.736,3.323)}
\gppoint{gp mark 3}{(1.631,1.907)}
\gppoint{gp mark 3}{(2.420,2.918)}
\gppoint{gp mark 3}{(2.025,2.298)}
\gppoint{gp mark 3}{(2.894,3.181)}
\gppoint{gp mark 3}{(2.341,2.809)}
\gppoint{gp mark 3}{(2.341,2.474)}
\gppoint{gp mark 3}{(2.657,2.715)}
\gppoint{gp mark 3}{(2.262,2.426)}
\gppoint{gp mark 3}{(1.473,1.622)}
\gppoint{gp mark 3}{(1.473,1.720)}
\gppoint{gp mark 3}{(2.183,2.490)}
\gppoint{gp mark 3}{(2.578,2.820)}
\gppoint{gp mark 3}{(2.657,2.959)}
\gppoint{gp mark 3}{(2.578,2.803)}
\gppoint{gp mark 3}{(1.631,1.820)}
\gppoint{gp mark 3}{(2.183,2.519)}
\gppoint{gp mark 3}{(3.052,3.086)}
\gppoint{gp mark 3}{(2.420,2.576)}
\gppoint{gp mark 3}{(3.525,3.354)}
\gppoint{gp mark 3}{(3.446,3.722)}
\gppoint{gp mark 3}{(3.289,3.179)}
\gppoint{gp mark 3}{(3.683,4.197)}
\gppoint{gp mark 3}{(3.210,3.377)}
\gppoint{gp mark 3}{(3.525,3.683)}
\gppoint{gp mark 3}{(2.815,2.876)}
\gppoint{gp mark 3}{(1.315,1.668)}
\gppoint{gp mark 3}{(2.025,2.847)}
\gppoint{gp mark 3}{(4.394,4.133)}
\gppoint{gp mark 3}{(2.973,3.212)}
\gppoint{gp mark 3}{(3.210,3.658)}
\gppoint{gp mark 3}{(2.657,3.155)}
\gppoint{gp mark 3}{(2.104,2.600)}
\gppoint{gp mark 3}{(1.631,2.299)}
\gppoint{gp mark 3}{(2.815,2.618)}
\gppoint{gp mark 3}{(5.341,5.137)}
\gppoint{gp mark 3}{(2.420,2.759)}
\gppoint{gp mark 3}{(2.262,2.452)}
\gppoint{gp mark 3}{(1.631,1.714)}
\gppoint{gp mark 3}{(3.289,3.178)}
\gppoint{gp mark 3}{(2.657,2.550)}
\gppoint{gp mark 3}{(2.736,3.440)}
\gppoint{gp mark 3}{(2.025,2.621)}
\gppoint{gp mark 3}{(2.183,2.400)}
\gppoint{gp mark 3}{(2.420,2.888)}
\gppoint{gp mark 3}{(2.183,2.012)}
\gppoint{gp mark 3}{(3.683,3.011)}
\gppoint{gp mark 3}{(4.394,4.156)}
\gppoint{gp mark 3}{(1.394,1.793)}
\gppoint{gp mark 3}{(3.289,3.435)}
\gppoint{gp mark 3}{(1.315,1.760)}
\gppoint{gp mark 3}{(1.788,2.096)}
\gppoint{gp mark 3}{(2.657,2.949)}
\gppoint{gp mark 3}{(3.210,3.132)}
\gppoint{gp mark 3}{(1.394,1.714)}
\gppoint{gp mark 3}{(2.973,3.267)}
\gppoint{gp mark 3}{(5.104,4.914)}
\gppoint{gp mark 3}{(3.367,3.087)}
\gppoint{gp mark 3}{(3.367,3.854)}
\gppoint{gp mark 3}{(3.683,4.020)}
\gppoint{gp mark 3}{(3.367,3.414)}
\gppoint{gp mark 3}{(2.183,2.290)}
\gppoint{gp mark 3}{(2.420,2.544)}
\gppoint{gp mark 3}{(2.025,1.988)}
\gppoint{gp mark 3}{(2.973,2.873)}
\gppoint{gp mark 3}{(2.420,2.516)}
\gppoint{gp mark 3}{(3.131,3.066)}
\gppoint{gp mark 3}{(1.236,1.656)}
\gppoint{gp mark 3}{(2.183,2.356)}
\gppoint{gp mark 3}{(1.946,2.361)}
\gppoint{gp mark 3}{(1.946,2.136)}
\gppoint{gp mark 3}{(2.815,3.035)}
\gppoint{gp mark 3}{(2.104,2.303)}
\gppoint{gp mark 3}{(2.578,2.611)}
\gppoint{gp mark 3}{(1.552,1.754)}
\gppoint{gp mark 3}{(4.315,5.289)}
\gppoint{gp mark 3}{(2.262,2.452)}
\gppoint{gp mark 3}{(2.025,1.799)}
\gppoint{gp mark 3}{(3.052,3.407)}
\gppoint{gp mark 3}{(3.604,3.641)}
\gppoint{gp mark 3}{(2.183,2.481)}
\gppoint{gp mark 3}{(1.631,2.109)}
\gppoint{gp mark 3}{(2.499,2.663)}
\gppoint{gp mark 3}{(2.341,2.567)}
\gppoint{gp mark 3}{(3.131,3.522)}
\gppoint{gp mark 3}{(2.183,2.500)}
\gppoint{gp mark 3}{(5.183,6.543)}
\gppoint{gp mark 3}{(2.499,2.955)}
\gppoint{gp mark 3}{(4.157,4.181)}
\gppoint{gp mark 3}{(2.657,2.594)}
\gppoint{gp mark 3}{(2.499,3.238)}
\gppoint{gp mark 3}{(2.025,1.967)}
\gppoint{gp mark 3}{(2.736,3.256)}
\gppoint{gp mark 3}{(1.473,1.782)}
\gppoint{gp mark 3}{(1.552,1.729)}
\gppoint{gp mark 3}{(2.499,2.510)}
\gppoint{gp mark 3}{(1.552,1.810)}
\gppoint{gp mark 3}{(4.236,4.252)}
\gppoint{gp mark 3}{(2.262,2.686)}
\gppoint{gp mark 3}{(2.183,2.353)}
\gppoint{gp mark 3}{(2.341,2.218)}
\gppoint{gp mark 3}{(5.341,6.459)}
\gppoint{gp mark 3}{(1.631,1.936)}
\gppoint{gp mark 3}{(2.183,2.480)}
\gppoint{gp mark 3}{(2.025,2.352)}
\gppoint{gp mark 3}{(1.473,1.892)}
\gppoint{gp mark 3}{(2.262,2.943)}
\gppoint{gp mark 3}{(1.394,1.674)}
\gppoint{gp mark 3}{(3.446,3.166)}
\gppoint{gp mark 3}{(5.420,5.744)}
\gppoint{gp mark 3}{(1.867,2.405)}
\gppoint{gp mark 3}{(1.867,2.334)}
\gppoint{gp mark 3}{(3.525,4.035)}
\gppoint{gp mark 3}{(3.210,3.216)}
\gppoint{gp mark 3}{(3.683,4.311)}
\gppoint{gp mark 3}{(2.262,2.524)}
\gppoint{gp mark 3}{(1.710,1.942)}
\gppoint{gp mark 3}{(3.604,3.563)}
\gppoint{gp mark 3}{(2.025,1.988)}
\gppoint{gp mark 3}{(1.710,1.868)}
\gppoint{gp mark 3}{(2.341,2.405)}
\gppoint{gp mark 3}{(3.683,4.017)}
\gppoint{gp mark 3}{(2.025,2.293)}
\gppoint{gp mark 3}{(2.815,2.898)}
\gppoint{gp mark 3}{(3.210,3.223)}
\gppoint{gp mark 3}{(4.236,4.383)}
\gppoint{gp mark 3}{(3.289,3.129)}
\gppoint{gp mark 3}{(5.894,6.531)}
\gppoint{gp mark 3}{(3.131,3.516)}
\gppoint{gp mark 3}{(4.157,4.634)}
\gppoint{gp mark 3}{(4.315,5.419)}
\gppoint{gp mark 3}{(3.052,3.130)}
\gppoint{gp mark 3}{(2.025,2.182)}
\gppoint{gp mark 3}{(3.841,4.106)}
\gppoint{gp mark 3}{(5.262,5.947)}
\gppoint{gp mark 3}{(2.894,3.406)}
\gppoint{gp mark 3}{(8.420,6.782)}
\gppoint{gp mark 3}{(3.289,3.763)}
\gppoint{gp mark 3}{(2.025,2.090)}
\gppoint{gp mark 3}{(1.552,2.001)}
\gppoint{gp mark 3}{(2.973,3.274)}
\gppoint{gp mark 3}{(2.973,3.202)}
\gppoint{gp mark 3}{(4.236,4.641)}
\gppoint{gp mark 3}{(2.657,2.927)}
\gppoint{gp mark 3}{(2.894,3.096)}
\gppoint{gp mark 3}{(2.025,2.177)}
\gppoint{gp mark 3}{(1.473,1.966)}
\gppoint{gp mark 3}{(1.394,1.649)}
\gppoint{gp mark 3}{(1.394,1.776)}
\gppoint{gp mark 3}{(2.578,2.238)}
\gppoint{gp mark 3}{(2.973,3.041)}
\gppoint{gp mark 3}{(2.420,2.683)}
\gppoint{gp mark 3}{(1.631,1.950)}
\gppoint{gp mark 3}{(3.999,4.194)}
\gppoint{gp mark 3}{(1.631,1.736)}
\gppoint{gp mark 3}{(1.788,2.203)}
\gppoint{gp mark 3}{(2.578,3.129)}
\gppoint{gp mark 3}{(4.157,4.212)}
\gppoint{gp mark 3}{(2.499,2.295)}
\gppoint{gp mark 3}{(2.262,2.289)}
\gppoint{gp mark 3}{(1.552,1.796)}
\gppoint{gp mark 3}{(4.078,4.173)}
\gppoint{gp mark 3}{(4.552,5.671)}
\gppoint{gp mark 3}{(1.710,1.768)}
\gppoint{gp mark 3}{(2.025,2.264)}
\gppoint{gp mark 3}{(2.262,2.501)}
\gppoint{gp mark 3}{(1.710,2.036)}
\gppoint{gp mark 3}{(3.604,3.672)}
\gppoint{gp mark 3}{(2.025,2.557)}
\gppoint{gp mark 3}{(3.289,4.364)}
\gppoint{gp mark 3}{(1.867,2.163)}
\gppoint{gp mark 3}{(3.999,4.379)}
\gppoint{gp mark 3}{(2.420,2.564)}
\gppoint{gp mark 3}{(2.262,2.384)}
\gppoint{gp mark 3}{(3.289,2.999)}
\gppoint{gp mark 3}{(5.341,5.413)}
\gppoint{gp mark 3}{(3.999,3.756)}
\gppoint{gp mark 3}{(1.473,1.678)}
\gppoint{gp mark 3}{(2.025,2.492)}
\gppoint{gp mark 3}{(1.473,1.781)}
\gppoint{gp mark 3}{(3.525,3.029)}
\gppoint{gp mark 3}{(1.157,1.545)}
\gppoint{gp mark 3}{(2.341,2.343)}
\gppoint{gp mark 3}{(3.289,3.038)}
\gppoint{gp mark 3}{(2.183,2.313)}
\gppoint{gp mark 3}{(3.052,2.994)}
\gppoint{gp mark 3}{(4.789,4.841)}
\gppoint{gp mark 3}{(2.815,3.748)}
\gppoint{gp mark 3}{(1.788,2.167)}
\gppoint{gp mark 3}{(2.183,2.617)}
\gppoint{gp mark 3}{(2.657,3.076)}
\gppoint{gp mark 3}{(3.052,3.198)}
\gppoint{gp mark 3}{(1.631,1.869)}
\gppoint{gp mark 3}{(2.025,2.380)}
\gppoint{gp mark 3}{(1.867,2.061)}
\gppoint{gp mark 3}{(2.341,2.706)}
\gppoint{gp mark 3}{(2.894,2.557)}
\gppoint{gp mark 3}{(2.262,2.447)}
\gppoint{gp mark 3}{(2.499,2.578)}
\gppoint{gp mark 3}{(1.473,1.889)}
\gppoint{gp mark 3}{(2.499,2.651)}
\gppoint{gp mark 3}{(2.025,1.813)}
\gppoint{gp mark 3}{(3.841,3.940)}
\gppoint{gp mark 3}{(2.104,2.667)}
\gppoint{gp mark 3}{(1.788,1.766)}
\gppoint{gp mark 3}{(3.525,3.606)}
\gppoint{gp mark 3}{(2.262,2.541)}
\gppoint{gp mark 3}{(5.420,4.919)}
\gppoint{gp mark 3}{(3.446,3.970)}
\gppoint{gp mark 3}{(2.736,2.869)}
\gppoint{gp mark 3}{(4.157,4.174)}
\gppoint{gp mark 3}{(1.867,2.116)}
\gppoint{gp mark 3}{(2.341,2.449)}
\gppoint{gp mark 3}{(1.946,1.928)}
\gppoint{gp mark 3}{(1.473,1.857)}
\gppoint{gp mark 3}{(1.394,1.661)}
\gppoint{gp mark 3}{(2.657,2.395)}
\gppoint{gp mark 3}{(2.025,1.962)}
\gppoint{gp mark 3}{(3.683,3.990)}
\gppoint{gp mark 3}{(2.499,2.849)}
\gppoint{gp mark 3}{(1.552,1.874)}
\gppoint{gp mark 3}{(1.946,1.977)}
\gppoint{gp mark 3}{(2.025,2.216)}
\gppoint{gp mark 3}{(3.525,2.464)}
\gppoint{gp mark 3}{(4.552,4.802)}
\gppoint{gp mark 3}{(2.657,2.977)}
\gppoint{gp mark 3}{(2.183,2.298)}
\gppoint{gp mark 3}{(2.025,2.241)}
\gppoint{gp mark 3}{(3.446,3.393)}
\gppoint{gp mark 3}{(2.025,2.349)}
\gppoint{gp mark 3}{(1.394,1.850)}
\gppoint{gp mark 3}{(2.657,2.693)}
\gppoint{gp mark 3}{(4.236,4.161)}
\gppoint{gp mark 3}{(4.078,3.751)}
\gppoint{gp mark 3}{(2.183,2.390)}
\gppoint{gp mark 3}{(2.262,2.471)}
\gppoint{gp mark 3}{(3.367,3.556)}
\gppoint{gp mark 3}{(1.867,2.218)}
\gppoint{gp mark 3}{(2.341,2.543)}
\gppoint{gp mark 3}{(3.052,2.781)}
\gppoint{gp mark 3}{(2.736,3.320)}
\gppoint{gp mark 3}{(1.552,1.821)}
\gppoint{gp mark 3}{(6.289,6.708)}
\gppoint{gp mark 3}{(4.710,4.469)}
\gppoint{gp mark 3}{(1.631,1.977)}
\gppoint{gp mark 3}{(1.394,1.664)}
\gppoint{gp mark 3}{(4.078,5.560)}
\gppoint{gp mark 3}{(2.025,2.403)}
\gppoint{gp mark 3}{(1.473,1.778)}
\gppoint{gp mark 3}{(2.736,3.080)}
\gppoint{gp mark 3}{(3.762,4.013)}
\gppoint{gp mark 3}{(3.920,5.011)}
\gppoint{gp mark 3}{(3.367,3.636)}
\gppoint{gp mark 3}{(1.867,2.035)}
\gppoint{gp mark 3}{(3.762,4.336)}
\gppoint{gp mark 3}{(6.683,6.473)}
\gppoint{gp mark 3}{(2.973,3.202)}
\gppoint{gp mark 3}{(2.578,2.942)}
\gppoint{gp mark 3}{(3.999,4.119)}
\gppoint{gp mark 3}{(3.289,3.119)}
\gppoint{gp mark 3}{(3.446,3.484)}
\gppoint{gp mark 3}{(4.473,4.119)}
\gppoint{gp mark 3}{(1.473,1.921)}
\gppoint{gp mark 3}{(2.657,2.512)}
\gppoint{gp mark 3}{(1.631,1.929)}
\gppoint{gp mark 3}{(1.631,2.049)}
\gppoint{gp mark 3}{(1.788,2.040)}
\gppoint{gp mark 3}{(1.946,2.227)}
\gppoint{gp mark 3}{(2.499,2.378)}
\gppoint{gp mark 3}{(2.104,1.971)}
\gppoint{gp mark 3}{(2.262,2.494)}
\gppoint{gp mark 3}{(2.499,2.784)}
\gppoint{gp mark 3}{(1.552,1.870)}
\gppoint{gp mark 3}{(1.710,1.713)}
\gppoint{gp mark 3}{(1.631,1.737)}
\gppoint{gp mark 3}{(2.183,2.710)}
\gppoint{gp mark 3}{(1.710,1.856)}
\gppoint{gp mark 3}{(2.025,2.227)}
\gppoint{gp mark 3}{(3.604,3.590)}
\gppoint{gp mark 3}{(1.710,1.883)}
\gppoint{gp mark 3}{(1.394,1.644)}
\gppoint{gp mark 3}{(2.578,2.422)}
\gppoint{gp mark 3}{(2.499,2.476)}
\gppoint{gp mark 3}{(2.025,2.536)}
\gppoint{gp mark 3}{(3.210,3.790)}
\gppoint{gp mark 3}{(1.394,1.785)}
\gppoint{gp mark 3}{(1.867,2.030)}
\gppoint{gp mark 3}{(1.946,2.206)}
\gppoint{gp mark 3}{(1.394,1.730)}
\gppoint{gp mark 3}{(1.710,1.716)}
\gppoint{gp mark 3}{(1.867,1.874)}
\gppoint{gp mark 3}{(3.210,3.416)}
\gppoint{gp mark 3}{(3.131,3.218)}
\gppoint{gp mark 3}{(2.262,2.525)}
\gppoint{gp mark 3}{(2.341,2.995)}
\gppoint{gp mark 3}{(2.262,2.839)}
\gppoint{gp mark 3}{(1.788,1.959)}
\gppoint{gp mark 3}{(2.025,2.155)}
\gppoint{gp mark 3}{(2.262,2.569)}
\gppoint{gp mark 3}{(3.131,3.638)}
\gppoint{gp mark 3}{(2.736,3.198)}
\gppoint{gp mark 3}{(2.420,2.475)}
\gppoint{gp mark 3}{(2.104,2.193)}
\gppoint{gp mark 3}{(3.052,3.188)}
\gppoint{gp mark 3}{(1.788,2.185)}
\gppoint{gp mark 3}{(3.446,4.022)}
\gppoint{gp mark 3}{(1.631,1.813)}
\gppoint{gp mark 3}{(2.736,2.435)}
\gppoint{gp mark 3}{(1.315,1.730)}
\gppoint{gp mark 3}{(2.657,2.796)}
\gppoint{gp mark 3}{(1.788,2.238)}
\gppoint{gp mark 3}{(2.657,3.024)}
\gppoint{gp mark 3}{(2.341,2.376)}
\gppoint{gp mark 3}{(1.473,1.794)}
\gppoint{gp mark 3}{(2.420,2.773)}
\gppoint{gp mark 3}{(2.183,2.257)}
\gppoint{gp mark 3}{(1.631,1.844)}
\gppoint{gp mark 3}{(1.710,1.838)}
\gppoint{gp mark 3}{(1.710,2.160)}
\gppoint{gp mark 3}{(3.367,3.368)}
\gppoint{gp mark 3}{(1.710,1.981)}
\gppoint{gp mark 3}{(2.420,2.825)}
\gppoint{gp mark 3}{(3.367,3.397)}
\gppoint{gp mark 3}{(3.920,4.199)}
\gppoint{gp mark 3}{(3.525,3.499)}
\gppoint{gp mark 3}{(3.920,3.797)}
\gppoint{gp mark 3}{(1.473,1.701)}
\gppoint{gp mark 3}{(2.104,1.858)}
\gppoint{gp mark 3}{(2.183,2.495)}
\gppoint{gp mark 3}{(5.973,6.154)}
\gppoint{gp mark 3}{(3.920,4.524)}
\gppoint{gp mark 3}{(4.789,4.979)}
\gppoint{gp mark 3}{(3.210,2.471)}
\gppoint{gp mark 3}{(3.131,2.997)}
\gppoint{gp mark 3}{(4.631,4.501)}
\gppoint{gp mark 3}{(3.683,3.845)}
\gppoint{gp mark 3}{(3.604,3.499)}
\gppoint{gp mark 3}{(2.183,2.516)}
\gppoint{gp mark 3}{(4.552,4.817)}
\gppoint{gp mark 3}{(2.894,3.068)}
\gppoint{gp mark 3}{(3.762,3.975)}
\gppoint{gp mark 3}{(2.657,2.739)}
\gppoint{gp mark 3}{(2.183,2.276)}
\gppoint{gp mark 3}{(3.604,3.153)}
\gppoint{gp mark 3}{(3.052,3.880)}
\gppoint{gp mark 3}{(4.710,5.170)}
\gppoint{gp mark 3}{(2.499,2.365)}
\gppoint{gp mark 3}{(1.710,1.984)}
\gppoint{gp mark 3}{(4.078,3.687)}
\gppoint{gp mark 3}{(3.289,4.056)}
\gppoint{gp mark 3}{(3.367,2.870)}
\gppoint{gp mark 3}{(2.104,2.783)}
\gppoint{gp mark 3}{(3.604,3.247)}
\gppoint{gp mark 3}{(4.078,4.320)}
\gppoint{gp mark 3}{(1.946,2.279)}
\gppoint{gp mark 3}{(3.446,3.023)}
\gppoint{gp mark 3}{(3.446,3.326)}
\gppoint{gp mark 3}{(4.236,4.606)}
\gppoint{gp mark 3}{(3.367,3.513)}
\gppoint{gp mark 3}{(5.104,5.299)}
\gppoint{gp mark 3}{(2.183,2.424)}
\gppoint{gp mark 3}{(3.683,3.793)}
\gppoint{gp mark 3}{(5.736,6.477)}
\gppoint{gp mark 3}{(2.183,2.181)}
\gppoint{gp mark 3}{(2.973,3.202)}
\gppoint{gp mark 3}{(4.157,2.614)}
\gppoint{gp mark 3}{(4.552,5.711)}
\gppoint{gp mark 3}{(2.973,3.961)}
\gppoint{gp mark 3}{(3.210,3.929)}
\gppoint{gp mark 3}{(6.131,6.234)}
\gppoint{gp mark 3}{(1.788,2.033)}
\gppoint{gp mark 3}{(4.157,3.578)}
\gppoint{gp mark 3}{(1.710,2.338)}
\gppoint{gp mark 3}{(4.394,4.169)}
\gppoint{gp mark 3}{(5.183,4.995)}
\gppoint{gp mark 3}{(4.315,3.977)}
\gppoint{gp mark 3}{(2.894,3.027)}
\gppoint{gp mark 3}{(3.210,3.637)}
\gppoint{gp mark 3}{(3.683,4.275)}
\gppoint{gp mark 3}{(2.973,2.921)}
\gppoint{gp mark 3}{(4.315,5.034)}
\gppoint{gp mark 3}{(4.236,4.383)}
\gppoint{gp mark 3}{(2.578,2.907)}
\gppoint{gp mark 3}{(5.578,6.126)}
\gppoint{gp mark 3}{(3.683,4.179)}
\gppoint{gp mark 3}{(3.052,3.072)}
\gppoint{gp mark 3}{(1.867,1.869)}
\gppoint{gp mark 3}{(2.341,2.619)}
\gppoint{gp mark 3}{(4.078,4.379)}
\gppoint{gp mark 3}{(2.262,2.428)}
\gppoint{gp mark 3}{(5.420,6.743)}
\gppoint{gp mark 3}{(5.341,6.264)}
\gppoint{gp mark 3}{(3.131,3.924)}
\gppoint{gp mark 3}{(5.183,5.711)}
\gppoint{gp mark 3}{(5.025,5.918)}
\gppoint{gp mark 3}{(3.683,4.520)}
\gppoint{gp mark 3}{(7.552,6.774)}
\gppoint{gp mark 3}{(2.420,2.550)}
\gppoint{gp mark 3}{(4.868,6.184)}
\gppoint{gp mark 3}{(2.657,3.215)}
\gppoint{gp mark 3}{(5.262,4.204)}
\gppoint{gp mark 3}{(5.499,6.237)}
\gppoint{gp mark 3}{(2.104,2.337)}
\gppoint{gp mark 3}{(3.999,5.209)}
\gppoint{gp mark 3}{(2.657,2.622)}
\gppoint{gp mark 3}{(2.341,2.444)}
\gppoint{gp mark 3}{(3.525,3.733)}
\gppoint{gp mark 3}{(3.289,2.981)}
\gppoint{gp mark 3}{(3.210,3.090)}
\gppoint{gp mark 3}{(2.736,3.104)}
\gppoint{gp mark 3}{(3.289,3.239)}
\gppoint{gp mark 3}{(3.367,4.288)}
\gppoint{gp mark 3}{(2.341,2.645)}
\gppoint{gp mark 3}{(2.973,3.346)}
\gppoint{gp mark 3}{(2.104,2.347)}
\gppoint{gp mark 3}{(1.788,2.447)}
\gppoint{gp mark 3}{(4.868,4.832)}
\gppoint{gp mark 3}{(3.367,3.950)}
\gppoint{gp mark 3}{(1.710,1.928)}
\gppoint{gp mark 3}{(3.367,3.365)}
\gppoint{gp mark 3}{(2.341,2.130)}
\gppoint{gp mark 3}{(1.710,2.034)}
\gppoint{gp mark 3}{(3.210,3.441)}
\gppoint{gp mark 3}{(4.631,4.865)}
\gppoint{gp mark 3}{(3.210,3.340)}
\gppoint{gp mark 3}{(3.210,2.775)}
\gppoint{gp mark 3}{(1.867,2.208)}
\gppoint{gp mark 3}{(3.446,2.841)}
\gppoint{gp mark 3}{(3.920,4.350)}
\gppoint{gp mark 3}{(3.289,3.523)}
\gppoint{gp mark 3}{(5.420,5.560)}
\gppoint{gp mark 3}{(5.341,6.114)}
\gppoint{gp mark 3}{(4.710,5.362)}
\gppoint{gp mark 3}{(5.104,5.624)}
\gppoint{gp mark 3}{(6.052,6.270)}
\gppoint{gp mark 3}{(4.394,4.642)}
\gppoint{gp mark 3}{(3.367,3.512)}
\gppoint{gp mark 3}{(5.736,6.059)}
\gppoint{gp mark 3}{(4.789,5.464)}
\gppoint{gp mark 3}{(3.604,3.916)}
\gppoint{gp mark 3}{(6.210,6.501)}
\gppoint{gp mark 3}{(4.868,4.971)}
\gppoint{gp mark 3}{(5.420,4.964)}
\gppoint{gp mark 3}{(4.946,5.649)}
\gppoint{gp mark 3}{(3.762,4.007)}
\gppoint{gp mark 3}{(3.446,3.631)}
\gppoint{gp mark 3}{(2.736,2.898)}
\gppoint{gp mark 3}{(3.446,3.375)}
\gppoint{gp mark 3}{(3.841,4.557)}
\gppoint{gp mark 3}{(3.841,3.846)}
\gppoint{gp mark 3}{(4.868,4.940)}
\gppoint{gp mark 3}{(7.157,6.617)}
\gppoint{gp mark 3}{(5.815,6.117)}
\gppoint{gp mark 3}{(4.710,3.569)}
\gppoint{gp mark 3}{(6.762,6.602)}
\gppoint{gp mark 3}{(6.525,6.562)}
\gppoint{gp mark 3}{(3.052,2.932)}
\gppoint{gp mark 3}{(3.920,3.810)}
\gppoint{gp mark 3}{(2.973,3.422)}
\gppoint{gp mark 3}{(2.499,2.239)}
\gppoint{gp mark 3}{(2.262,2.784)}
\gppoint{gp mark 3}{(1.946,2.221)}
\gppoint{gp mark 3}{(2.262,2.801)}
\gppoint{gp mark 3}{(3.289,3.748)}
\gppoint{gp mark 3}{(1.631,1.928)}
\gppoint{gp mark 3}{(3.999,4.427)}
\gppoint{gp mark 3}{(1.473,1.898)}
\gppoint{gp mark 3}{(5.104,4.440)}
\gppoint{gp mark 3}{(2.736,2.982)}
\gppoint{gp mark 3}{(2.894,3.009)}
\gppoint{gp mark 3}{(2.262,2.439)}
\gppoint{gp mark 3}{(1.946,2.260)}
\gppoint{gp mark 3}{(2.341,2.231)}
\gppoint{gp mark 3}{(1.552,1.784)}
\gppoint{gp mark 3}{(1.946,2.286)}
\gppoint{gp mark 3}{(3.367,3.206)}
\gppoint{gp mark 3}{(3.210,3.216)}
\gppoint{gp mark 3}{(4.552,4.992)}
\gppoint{gp mark 3}{(1.867,2.214)}
\gppoint{gp mark 3}{(1.315,1.679)}
\gppoint{gp mark 3}{(3.525,3.740)}
\gppoint{gp mark 3}{(2.894,2.901)}
\gppoint{gp mark 3}{(3.052,3.167)}
\gppoint{gp mark 3}{(2.262,2.317)}
\gppoint{gp mark 3}{(3.683,2.972)}
\gppoint{gp mark 3}{(3.920,4.322)}
\gppoint{gp mark 3}{(2.183,2.197)}
\gppoint{gp mark 3}{(3.920,3.935)}
\gppoint{gp mark 3}{(2.183,2.231)}
\gppoint{gp mark 3}{(2.341,2.364)}
\gppoint{gp mark 3}{(3.762,3.481)}
\gppoint{gp mark 3}{(3.525,3.648)}
\gppoint{gp mark 3}{(2.894,2.797)}
\gppoint{gp mark 3}{(2.657,2.400)}
\gppoint{gp mark 3}{(2.262,2.656)}
\gppoint{gp mark 3}{(1.946,1.964)}
\gppoint{gp mark 3}{(1.631,1.842)}
\gppoint{gp mark 3}{(3.210,3.344)}
\gppoint{gp mark 3}{(3.289,3.862)}
\gppoint{gp mark 3}{(2.894,3.011)}
\gppoint{gp mark 3}{(2.183,2.423)}
\gppoint{gp mark 3}{(3.841,4.066)}
\gppoint{gp mark 3}{(1.710,2.054)}
\gppoint{gp mark 3}{(3.683,3.878)}
\gppoint{gp mark 3}{(2.420,2.951)}
\gppoint{gp mark 3}{(2.183,2.560)}
\gppoint{gp mark 3}{(1.867,2.046)}
\gppoint{gp mark 3}{(2.341,2.555)}
\gppoint{gp mark 3}{(6.762,5.955)}
\gppoint{gp mark 3}{(1.788,2.041)}
\gppoint{gp mark 3}{(4.868,4.727)}
\gppoint{gp mark 3}{(3.841,4.696)}
\gppoint{gp mark 3}{(3.289,3.812)}
\gppoint{gp mark 3}{(3.604,4.233)}
\gppoint{gp mark 3}{(3.210,3.129)}
\gppoint{gp mark 3}{(2.341,2.472)}
\gppoint{gp mark 3}{(7.078,5.831)}
\gppoint{gp mark 3}{(3.289,3.611)}
\gppoint{gp mark 3}{(3.920,4.339)}
\gppoint{gp mark 3}{(2.262,2.746)}
\gppoint{gp mark 3}{(1.631,2.010)}
\gppoint{gp mark 3}{(2.025,2.265)}
\gppoint{gp mark 3}{(3.604,4.056)}
\gppoint{gp mark 3}{(3.762,4.536)}
\gppoint{gp mark 3}{(1.394,1.662)}
\gppoint{gp mark 3}{(1.788,2.236)}
\gppoint{gp mark 3}{(5.025,4.807)}
\gppoint{gp mark 3}{(6.210,6.087)}
\gppoint{gp mark 3}{(4.236,4.489)}
\gppoint{gp mark 3}{(3.289,3.248)}
\gppoint{gp mark 3}{(1.867,2.110)}
\gppoint{gp mark 3}{(1.710,2.107)}
\gppoint{gp mark 3}{(3.525,3.428)}
\gppoint{gp mark 3}{(3.210,3.348)}
\gppoint{gp mark 3}{(2.657,3.034)}
\gppoint{gp mark 3}{(3.841,4.398)}
\gppoint{gp mark 3}{(3.052,3.376)}
\gppoint{gp mark 3}{(2.420,2.299)}
\gppoint{gp mark 3}{(3.920,3.881)}
\gppoint{gp mark 3}{(4.315,4.457)}
\gppoint{gp mark 3}{(3.525,3.264)}
\gppoint{gp mark 3}{(3.920,3.573)}
\gppoint{gp mark 3}{(3.367,3.609)}
\gppoint{gp mark 3}{(5.025,5.557)}
\gppoint{gp mark 3}{(2.499,3.022)}
\gppoint{gp mark 3}{(2.025,2.174)}
\gppoint{gp mark 3}{(2.578,3.048)}
\gppoint{gp mark 3}{(2.894,3.072)}
\gppoint{gp mark 3}{(2.578,2.399)}
\gppoint{gp mark 3}{(2.815,2.992)}
\gppoint{gp mark 3}{(1.788,2.117)}
\gppoint{gp mark 3}{(1.710,1.859)}
\gppoint{gp mark 3}{(4.315,4.571)}
\gppoint{gp mark 3}{(3.525,3.681)}
\gppoint{gp mark 3}{(2.183,2.300)}
\gppoint{gp mark 3}{(2.894,2.538)}
\gppoint{gp mark 3}{(4.157,4.359)}
\gppoint{gp mark 3}{(3.762,4.125)}
\gppoint{gp mark 3}{(3.841,4.066)}
\gppoint{gp mark 3}{(5.499,3.471)}
\gppoint{gp mark 3}{(2.499,2.761)}
\gppoint{gp mark 3}{(1.631,1.870)}
\gppoint{gp mark 3}{(2.973,3.261)}
\gppoint{gp mark 3}{(2.578,2.816)}
\gppoint{gp mark 3}{(2.262,1.932)}
\gppoint{gp mark 3}{(3.999,3.993)}
\gppoint{gp mark 3}{(2.973,2.953)}
\gppoint{gp mark 3}{(3.052,3.560)}
\gppoint{gp mark 3}{(4.868,2.697)}
\gppoint{gp mark 3}{(1.631,1.861)}
\gppoint{gp mark 3}{(2.341,2.771)}
\gppoint{gp mark 3}{(4.394,4.701)}
\gppoint{gp mark 3}{(2.183,2.487)}
\gppoint{gp mark 3}{(2.657,3.035)}
\gppoint{gp mark 3}{(2.973,2.759)}
\gppoint{gp mark 3}{(2.736,3.502)}
\gppoint{gp mark 3}{(3.052,3.891)}
\gppoint{gp mark 3}{(3.446,3.830)}
\gppoint{gp mark 3}{(3.367,3.143)}
\gppoint{gp mark 3}{(3.446,3.642)}
\gppoint{gp mark 3}{(1.552,2.018)}
\gppoint{gp mark 3}{(2.183,2.925)}
\gppoint{gp mark 3}{(3.762,4.215)}
\gppoint{gp mark 3}{(2.499,2.421)}
\gppoint{gp mark 3}{(4.868,5.733)}
\gppoint{gp mark 3}{(1.867,2.015)}
\gppoint{gp mark 3}{(3.525,3.613)}
\gppoint{gp mark 3}{(3.525,4.247)}
\gppoint{gp mark 3}{(1.315,1.688)}
\gppoint{gp mark 3}{(3.525,3.477)}
\gppoint{gp mark 3}{(2.973,3.320)}
\gppoint{gp mark 3}{(3.131,3.468)}
\gppoint{gp mark 3}{(1.394,1.813)}
\gppoint{gp mark 3}{(1.631,2.274)}
\gppoint{gp mark 3}{(2.973,3.350)}
\gppoint{gp mark 3}{(1.315,1.693)}
\gppoint{gp mark 3}{(2.341,2.324)}
\gppoint{gp mark 3}{(3.052,2.953)}
\gppoint{gp mark 3}{(2.578,2.466)}
\gppoint{gp mark 3}{(1.631,1.797)}
\gppoint{gp mark 3}{(2.894,2.939)}
\gppoint{gp mark 3}{(3.920,3.709)}
\gppoint{gp mark 3}{(2.736,3.098)}
\gppoint{gp mark 3}{(3.210,3.212)}
\gppoint{gp mark 3}{(2.578,2.617)}
\gppoint{gp mark 3}{(1.157,1.627)}
\gppoint{gp mark 3}{(2.341,2.484)}
\gppoint{gp mark 3}{(5.183,6.578)}
\gppoint{gp mark 3}{(1.788,2.305)}
\gppoint{gp mark 3}{(2.420,2.672)}
\gppoint{gp mark 3}{(2.973,2.946)}
\gppoint{gp mark 3}{(2.815,2.884)}
\gppoint{gp mark 3}{(2.420,2.999)}
\gppoint{gp mark 3}{(2.420,2.787)}
\gppoint{gp mark 3}{(1.946,2.337)}
\gppoint{gp mark 3}{(2.657,2.700)}
\gppoint{gp mark 3}{(2.973,3.149)}
\gppoint{gp mark 3}{(1.552,1.915)}
\gppoint{gp mark 3}{(2.025,2.274)}
\gppoint{gp mark 3}{(2.578,2.470)}
\gppoint{gp mark 3}{(3.289,3.101)}
\gppoint{gp mark 3}{(4.789,5.744)}
\gppoint{gp mark 3}{(2.104,2.012)}
\gppoint{gp mark 3}{(2.578,2.573)}
\gppoint{gp mark 3}{(3.604,2.943)}
\gppoint{gp mark 3}{(3.052,2.806)}
\gppoint{gp mark 3}{(3.446,3.318)}
\gppoint{gp mark 3}{(3.210,3.521)}
\gppoint{gp mark 3}{(5.420,5.433)}
\gppoint{gp mark 3}{(3.367,3.593)}
\gppoint{gp mark 3}{(2.341,2.460)}
\gppoint{gp mark 3}{(4.157,4.871)}
\gppoint{gp mark 3}{(2.499,2.932)}
\gppoint{gp mark 3}{(3.683,4.090)}
\gppoint{gp mark 3}{(3.052,2.808)}
\gppoint{gp mark 3}{(2.657,3.450)}
\gppoint{gp mark 3}{(2.815,2.928)}
\gppoint{gp mark 3}{(4.552,5.344)}
\gppoint{gp mark 3}{(3.762,4.353)}
\gppoint{gp mark 3}{(2.578,2.461)}
\gppoint{gp mark 3}{(2.025,2.264)}
\gppoint{gp mark 3}{(2.420,2.848)}
\gppoint{gp mark 3}{(1.473,1.908)}
\gppoint{gp mark 3}{(2.420,2.626)}
\gppoint{gp mark 3}{(3.210,3.718)}
\gppoint{gp mark 3}{(4.315,5.683)}
\gppoint{gp mark 3}{(3.052,4.229)}
\gppoint{gp mark 3}{(3.920,4.577)}
\gppoint{gp mark 3}{(2.973,3.897)}
\gppoint{gp mark 3}{(4.078,4.193)}
\gppoint{gp mark 3}{(3.052,3.579)}
\gppoint{gp mark 3}{(4.315,3.685)}
\gppoint{gp mark 3}{(1.867,2.167)}
\gppoint{gp mark 3}{(3.762,3.964)}
\gppoint{gp mark 3}{(4.868,6.205)}
\gppoint{gp mark 3}{(5.183,5.501)}
\gppoint{gp mark 3}{(2.657,2.627)}
\gppoint{gp mark 3}{(3.446,3.224)}
\gppoint{gp mark 3}{(2.262,2.152)}
\gppoint{gp mark 3}{(3.999,4.121)}
\gppoint{gp mark 3}{(4.315,3.982)}
\gppoint{gp mark 3}{(1.788,2.081)}
\gppoint{gp mark 3}{(1.867,2.363)}
\gppoint{gp mark 3}{(3.367,3.729)}
\gppoint{gp mark 3}{(2.499,3.438)}
\gppoint{gp mark 3}{(3.131,3.422)}
\gppoint{gp mark 3}{(2.657,3.128)}
\gppoint{gp mark 3}{(2.973,3.075)}
\gppoint{gp mark 3}{(2.341,2.430)}
\gppoint{gp mark 3}{(2.578,3.202)}
\gppoint{gp mark 3}{(4.157,4.473)}
\gppoint{gp mark 3}{(4.315,4.368)}
\gppoint{gp mark 3}{(2.499,2.563)}
\gppoint{gp mark 3}{(2.104,2.189)}
\gppoint{gp mark 3}{(1.394,1.681)}
\gppoint{gp mark 3}{(4.394,4.767)}
\gppoint{gp mark 3}{(4.631,5.589)}
\gppoint{gp mark 3}{(6.368,5.549)}
\gppoint{gp mark 3}{(5.183,5.542)}
\gppoint{gp mark 3}{(3.999,3.928)}
\gppoint{gp mark 3}{(2.736,3.132)}
\gppoint{gp mark 3}{(5.104,6.267)}
\gppoint{gp mark 3}{(3.604,3.412)}
\gppoint{gp mark 3}{(5.499,5.591)}
\gppoint{gp mark 3}{(4.946,4.923)}
\gppoint{gp mark 3}{(3.052,3.068)}
\gppoint{gp mark 3}{(6.920,6.618)}
\gppoint{gp mark 3}{(3.367,3.663)}
\gppoint{gp mark 3}{(2.499,2.969)}
\gppoint{gp mark 3}{(2.183,2.324)}
\gppoint{gp mark 3}{(3.131,3.540)}
\gppoint{gp mark 3}{(2.499,3.133)}
\gppoint{gp mark 3}{(3.289,3.414)}
\gppoint{gp mark 3}{(4.236,2.935)}
\gppoint{gp mark 3}{(2.657,2.738)}
\gppoint{gp mark 3}{(3.446,4.047)}
\gppoint{gp mark 3}{(3.683,3.444)}
\gppoint{gp mark 3}{(3.999,4.357)}
\gppoint{gp mark 3}{(1.946,1.890)}
\gppoint{gp mark 3}{(2.420,2.509)}
\gppoint{gp mark 3}{(4.078,4.687)}
\gppoint{gp mark 3}{(3.841,3.423)}
\gppoint{gp mark 3}{(3.210,3.152)}
\gppoint{gp mark 3}{(8.357,4.223)}
\gpcolor{color=gp lt color border}
\node[gp node right] at (8.083,3.761) {\acs{nar}};
\gpcolor{rgb color={0.000,0.000,0.867}}
\gppoint{gp mark 1}{(1.552,1.628)}
\gppoint{gp mark 1}{(5.104,1.525)}
\gppoint{gp mark 1}{(3.131,1.372)}
\gppoint{gp mark 1}{(2.183,1.295)}
\gppoint{gp mark 1}{(2.262,1.296)}
\gppoint{gp mark 1}{(2.736,1.334)}
\gppoint{gp mark 1}{(4.078,1.450)}
\gppoint{gp mark 1}{(4.473,1.486)}
\gppoint{gp mark 1}{(3.446,1.388)}
\gppoint{gp mark 1}{(3.367,1.397)}
\gppoint{gp mark 1}{(3.131,1.371)}
\gppoint{gp mark 1}{(2.262,1.299)}
\gppoint{gp mark 1}{(2.025,1.281)}
\gppoint{gp mark 1}{(2.104,1.293)}
\gppoint{gp mark 1}{(2.104,1.289)}
\gppoint{gp mark 1}{(2.499,1.321)}
\gppoint{gp mark 1}{(3.525,1.402)}
\gppoint{gp mark 1}{(3.289,1.402)}
\gppoint{gp mark 1}{(3.210,1.376)}
\gppoint{gp mark 1}{(2.341,1.304)}
\gppoint{gp mark 1}{(2.183,1.285)}
\gppoint{gp mark 1}{(2.736,1.321)}
\gppoint{gp mark 1}{(4.710,1.481)}
\gppoint{gp mark 1}{(3.525,1.390)}
\gppoint{gp mark 1}{(3.841,1.417)}
\gppoint{gp mark 1}{(4.473,1.480)}
\gppoint{gp mark 1}{(2.104,1.292)}
\gppoint{gp mark 1}{(3.762,1.418)}
\gppoint{gp mark 1}{(4.315,1.459)}
\gppoint{gp mark 1}{(4.078,1.441)}
\gppoint{gp mark 1}{(5.183,1.540)}
\gppoint{gp mark 1}{(4.631,1.478)}
\gppoint{gp mark 1}{(5.025,1.490)}
\gppoint{gp mark 1}{(5.420,1.552)}
\gppoint{gp mark 1}{(3.052,1.366)}
\gppoint{gp mark 1}{(7.157,1.595)}
\gppoint{gp mark 1}{(2.183,1.351)}
\gppoint{gp mark 1}{(5.183,1.517)}
\gppoint{gp mark 1}{(4.552,1.469)}
\gppoint{gp mark 1}{(2.341,1.288)}
\gppoint{gp mark 1}{(1.394,1.209)}
\gppoint{gp mark 1}{(2.499,1.292)}
\gppoint{gp mark 1}{(3.446,1.359)}
\gppoint{gp mark 1}{(2.183,1.266)}
\gppoint{gp mark 1}{(1.552,1.222)}
\gppoint{gp mark 1}{(3.289,1.356)}
\gppoint{gp mark 1}{(2.973,1.344)}
\gppoint{gp mark 1}{(2.025,1.269)}
\gppoint{gp mark 1}{(3.604,1.385)}
\gppoint{gp mark 1}{(4.946,1.505)}
\gppoint{gp mark 1}{(3.210,1.370)}
\gppoint{gp mark 1}{(3.762,1.404)}
\gppoint{gp mark 1}{(2.578,1.311)}
\gppoint{gp mark 1}{(3.446,1.393)}
\gppoint{gp mark 1}{(6.999,1.622)}
\gppoint{gp mark 1}{(3.604,1.415)}
\gppoint{gp mark 1}{(3.525,1.403)}
\gppoint{gp mark 1}{(3.052,1.405)}
\gppoint{gp mark 1}{(4.078,1.433)}
\gppoint{gp mark 1}{(3.920,1.428)}
\gppoint{gp mark 1}{(2.973,1.359)}
\gppoint{gp mark 1}{(3.683,1.420)}
\gppoint{gp mark 1}{(3.367,1.393)}
\gppoint{gp mark 1}{(2.341,1.308)}
\gppoint{gp mark 1}{(3.052,1.368)}
\gppoint{gp mark 1}{(2.578,1.326)}
\gppoint{gp mark 1}{(3.525,1.393)}
\gppoint{gp mark 1}{(4.710,1.482)}
\gppoint{gp mark 1}{(2.104,1.277)}
\gppoint{gp mark 1}{(2.341,1.290)}
\gppoint{gp mark 1}{(4.157,1.451)}
\gppoint{gp mark 1}{(2.499,1.324)}
\gppoint{gp mark 1}{(3.131,1.372)}
\gppoint{gp mark 1}{(3.999,1.442)}
\gppoint{gp mark 1}{(3.683,1.423)}
\gppoint{gp mark 1}{(5.657,1.586)}
\gppoint{gp mark 1}{(3.920,1.456)}
\gppoint{gp mark 1}{(3.052,1.358)}
\gppoint{gp mark 1}{(3.525,1.403)}
\gppoint{gp mark 1}{(1.552,1.244)}
\gppoint{gp mark 1}{(6.999,1.613)}
\gppoint{gp mark 1}{(3.289,1.396)}
\gppoint{gp mark 1}{(5.104,1.549)}
\gppoint{gp mark 1}{(3.762,1.426)}
\gppoint{gp mark 1}{(3.210,1.366)}
\gppoint{gp mark 1}{(4.710,1.487)}
\gppoint{gp mark 1}{(5.025,1.521)}
\gppoint{gp mark 1}{(2.894,1.354)}
\gppoint{gp mark 1}{(3.367,1.382)}
\gppoint{gp mark 1}{(3.131,1.367)}
\gppoint{gp mark 1}{(3.210,1.360)}
\gppoint{gp mark 1}{(2.341,1.290)}
\gppoint{gp mark 1}{(2.894,1.346)}
\gppoint{gp mark 1}{(2.578,1.326)}
\gppoint{gp mark 1}{(1.473,1.240)}
\gppoint{gp mark 1}{(2.341,1.286)}
\gppoint{gp mark 1}{(3.052,1.347)}
\gppoint{gp mark 1}{(3.683,1.416)}
\gppoint{gp mark 1}{(2.499,1.302)}
\gppoint{gp mark 1}{(3.367,1.376)}
\gppoint{gp mark 1}{(3.525,1.407)}
\gppoint{gp mark 1}{(1.394,1.228)}
\gppoint{gp mark 1}{(1.867,1.265)}
\gppoint{gp mark 1}{(3.683,1.427)}
\gppoint{gp mark 1}{(3.446,1.396)}
\gppoint{gp mark 1}{(4.868,1.503)}
\gppoint{gp mark 1}{(3.683,1.393)}
\gppoint{gp mark 1}{(3.131,1.346)}
\gppoint{gp mark 1}{(2.104,1.283)}
\gppoint{gp mark 1}{(2.973,1.368)}
\gppoint{gp mark 1}{(3.841,1.427)}
\gppoint{gp mark 1}{(1.946,1.295)}
\gppoint{gp mark 1}{(1.788,1.279)}
\gppoint{gp mark 1}{(3.367,1.372)}
\gppoint{gp mark 1}{(2.025,1.285)}
\gppoint{gp mark 1}{(1.552,1.237)}
\gppoint{gp mark 1}{(1.946,1.279)}
\gppoint{gp mark 1}{(2.815,1.370)}
\gppoint{gp mark 1}{(3.052,1.404)}
\gppoint{gp mark 1}{(1.552,1.229)}
\gppoint{gp mark 1}{(4.157,1.426)}
\gppoint{gp mark 1}{(4.236,1.447)}
\gppoint{gp mark 1}{(5.025,1.520)}
\gppoint{gp mark 1}{(5.025,1.542)}
\gppoint{gp mark 1}{(3.525,1.401)}
\gppoint{gp mark 1}{(3.289,1.369)}
\gppoint{gp mark 1}{(3.841,1.414)}
\gppoint{gp mark 1}{(2.104,1.277)}
\gppoint{gp mark 1}{(1.394,1.235)}
\gppoint{gp mark 1}{(4.946,1.527)}
\gppoint{gp mark 1}{(4.473,1.469)}
\gppoint{gp mark 1}{(5.341,1.525)}
\gppoint{gp mark 1}{(5.736,1.571)}
\gppoint{gp mark 1}{(2.420,1.300)}
\gppoint{gp mark 1}{(3.210,1.373)}
\gppoint{gp mark 1}{(3.289,1.365)}
\gppoint{gp mark 1}{(3.920,1.417)}
\gppoint{gp mark 1}{(3.210,1.365)}
\gppoint{gp mark 1}{(3.052,1.353)}
\gppoint{gp mark 1}{(4.078,1.437)}
\gppoint{gp mark 1}{(2.736,1.342)}
\gppoint{gp mark 1}{(2.973,1.367)}
\gppoint{gp mark 1}{(3.604,1.403)}
\gppoint{gp mark 1}{(2.736,1.351)}
\gppoint{gp mark 1}{(1.946,1.293)}
\gppoint{gp mark 1}{(4.789,1.519)}
\gppoint{gp mark 1}{(4.315,1.449)}
\gppoint{gp mark 1}{(4.157,1.431)}
\gppoint{gp mark 1}{(2.104,1.279)}
\gppoint{gp mark 1}{(4.157,1.439)}
\gppoint{gp mark 1}{(4.078,1.431)}
\gppoint{gp mark 1}{(1.315,1.208)}
\gppoint{gp mark 1}{(3.604,1.393)}
\gppoint{gp mark 1}{(7.552,1.618)}
\gppoint{gp mark 1}{(1.236,1.242)}
\gppoint{gp mark 1}{(2.262,1.296)}
\gppoint{gp mark 1}{(3.131,1.371)}
\gppoint{gp mark 1}{(5.578,1.577)}
\gppoint{gp mark 1}{(3.762,1.425)}
\gppoint{gp mark 1}{(2.815,1.345)}
\gppoint{gp mark 1}{(4.946,1.515)}
\gppoint{gp mark 1}{(1.078,1.207)}
\gppoint{gp mark 1}{(3.210,1.386)}
\gppoint{gp mark 1}{(3.131,1.369)}
\gppoint{gp mark 1}{(1.631,1.254)}
\gppoint{gp mark 1}{(6.210,1.606)}
\gppoint{gp mark 1}{(3.683,1.416)}
\gppoint{gp mark 1}{(5.104,1.522)}
\gppoint{gp mark 1}{(4.157,1.446)}
\gppoint{gp mark 1}{(2.183,1.287)}
\gppoint{gp mark 1}{(4.157,1.436)}
\gppoint{gp mark 1}{(3.131,1.365)}
\gppoint{gp mark 1}{(3.052,1.353)}
\gppoint{gp mark 1}{(4.631,1.509)}
\gppoint{gp mark 1}{(4.552,1.476)}
\gppoint{gp mark 1}{(2.025,1.269)}
\gppoint{gp mark 1}{(2.025,1.266)}
\gppoint{gp mark 1}{(3.131,1.356)}
\gppoint{gp mark 1}{(3.289,1.393)}
\gppoint{gp mark 1}{(2.262,1.302)}
\gppoint{gp mark 1}{(5.657,1.594)}
\gppoint{gp mark 1}{(2.499,1.328)}
\gppoint{gp mark 1}{(2.183,1.293)}
\gppoint{gp mark 1}{(3.841,1.429)}
\gppoint{gp mark 1}{(3.210,1.387)}
\gppoint{gp mark 1}{(2.420,1.315)}
\gppoint{gp mark 1}{(3.210,1.375)}
\gppoint{gp mark 1}{(2.499,1.322)}
\gppoint{gp mark 1}{(3.841,1.416)}
\gppoint{gp mark 1}{(3.210,1.382)}
\gppoint{gp mark 1}{(5.499,1.582)}
\gppoint{gp mark 1}{(3.210,1.375)}
\gppoint{gp mark 1}{(7.315,1.609)}
\gppoint{gp mark 1}{(4.946,1.509)}
\gppoint{gp mark 1}{(4.946,1.514)}
\gppoint{gp mark 1}{(4.078,1.435)}
\gppoint{gp mark 1}{(2.815,1.336)}
\gppoint{gp mark 1}{(4.631,1.489)}
\gppoint{gp mark 1}{(3.525,1.394)}
\gppoint{gp mark 1}{(2.183,1.301)}
\gppoint{gp mark 1}{(1.631,1.238)}
\gppoint{gp mark 1}{(2.025,1.275)}
\gppoint{gp mark 1}{(2.657,1.327)}
\gppoint{gp mark 1}{(3.052,1.362)}
\gppoint{gp mark 1}{(2.499,1.320)}
\gppoint{gp mark 1}{(6.999,1.604)}
\gppoint{gp mark 1}{(2.499,1.317)}
\gppoint{gp mark 1}{(1.867,1.271)}
\gppoint{gp mark 1}{(4.473,1.481)}
\gppoint{gp mark 1}{(1.788,1.273)}
\gppoint{gp mark 1}{(2.894,1.352)}
\gppoint{gp mark 1}{(3.841,1.432)}
\gppoint{gp mark 1}{(3.289,1.389)}
\gppoint{gp mark 1}{(3.999,1.433)}
\gppoint{gp mark 1}{(2.657,1.330)}
\gppoint{gp mark 1}{(3.999,1.445)}
\gppoint{gp mark 1}{(3.131,1.366)}
\gppoint{gp mark 1}{(3.525,1.425)}
\gppoint{gp mark 1}{(4.236,1.457)}
\gppoint{gp mark 1}{(5.262,1.535)}
\gppoint{gp mark 1}{(5.183,1.528)}
\gppoint{gp mark 1}{(1.867,1.267)}
\gppoint{gp mark 1}{(2.815,1.321)}
\gppoint{gp mark 1}{(2.420,1.299)}
\gppoint{gp mark 1}{(3.762,1.429)}
\gppoint{gp mark 1}{(3.289,1.384)}
\gppoint{gp mark 1}{(2.657,1.323)}
\gppoint{gp mark 1}{(3.210,1.351)}
\gppoint{gp mark 1}{(4.315,1.452)}
\gppoint{gp mark 1}{(3.920,1.429)}
\gppoint{gp mark 1}{(1.236,1.221)}
\gppoint{gp mark 1}{(5.578,1.547)}
\gppoint{gp mark 1}{(6.525,1.619)}
\gppoint{gp mark 1}{(3.052,1.372)}
\gppoint{gp mark 1}{(6.131,1.590)}
\gppoint{gp mark 1}{(5.025,1.513)}
\gppoint{gp mark 1}{(2.736,1.365)}
\gppoint{gp mark 1}{(4.631,1.490)}
\gppoint{gp mark 1}{(2.657,1.328)}
\gppoint{gp mark 1}{(3.052,1.357)}
\gppoint{gp mark 1}{(1.631,1.258)}
\gppoint{gp mark 1}{(1.631,1.248)}
\gppoint{gp mark 1}{(1.710,1.264)}
\gppoint{gp mark 1}{(2.025,1.271)}
\gppoint{gp mark 1}{(4.078,1.441)}
\gppoint{gp mark 1}{(2.973,1.362)}
\gppoint{gp mark 1}{(3.920,1.445)}
\gppoint{gp mark 1}{(2.578,1.332)}
\gppoint{gp mark 1}{(6.052,1.609)}
\gppoint{gp mark 1}{(3.683,1.428)}
\gppoint{gp mark 1}{(3.920,1.450)}
\gppoint{gp mark 1}{(5.262,1.570)}
\gppoint{gp mark 1}{(2.815,1.338)}
\gppoint{gp mark 1}{(3.683,1.419)}
\gppoint{gp mark 1}{(1.946,1.267)}
\gppoint{gp mark 1}{(2.499,1.297)}
\gppoint{gp mark 1}{(2.499,1.304)}
\gppoint{gp mark 1}{(2.341,1.299)}
\gppoint{gp mark 1}{(3.604,1.388)}
\gppoint{gp mark 1}{(2.420,1.297)}
\gppoint{gp mark 1}{(4.631,1.512)}
\gppoint{gp mark 1}{(1.788,1.260)}
\gppoint{gp mark 1}{(3.446,1.380)}
\gppoint{gp mark 1}{(2.499,1.320)}
\gppoint{gp mark 1}{(3.210,1.389)}
\gppoint{gp mark 1}{(2.262,1.333)}
\gppoint{gp mark 1}{(4.394,1.464)}
\gppoint{gp mark 1}{(2.104,1.276)}
\gppoint{gp mark 1}{(1.946,1.271)}
\gppoint{gp mark 1}{(2.499,1.314)}
\gppoint{gp mark 1}{(3.131,1.372)}
\gppoint{gp mark 1}{(2.341,1.310)}
\gppoint{gp mark 1}{(2.657,1.339)}
\gppoint{gp mark 1}{(1.631,1.251)}
\gppoint{gp mark 1}{(3.446,1.405)}
\gppoint{gp mark 1}{(3.762,1.432)}
\gppoint{gp mark 1}{(2.815,1.355)}
\gppoint{gp mark 1}{(3.999,1.449)}
\gppoint{gp mark 1}{(3.131,1.375)}
\gppoint{gp mark 1}{(2.894,1.354)}
\gppoint{gp mark 1}{(2.262,1.306)}
\gppoint{gp mark 1}{(2.657,1.338)}
\gppoint{gp mark 1}{(4.473,1.491)}
\gppoint{gp mark 1}{(2.420,1.321)}
\gppoint{gp mark 1}{(3.683,1.431)}
\gppoint{gp mark 1}{(5.025,1.527)}
\gppoint{gp mark 1}{(4.789,1.509)}
\gppoint{gp mark 1}{(2.499,1.328)}
\gppoint{gp mark 1}{(3.604,1.418)}
\gppoint{gp mark 1}{(2.025,1.293)}
\gppoint{gp mark 1}{(3.525,1.401)}
\gppoint{gp mark 1}{(3.367,1.379)}
\gppoint{gp mark 1}{(5.341,1.560)}
\gppoint{gp mark 1}{(1.867,1.276)}
\gppoint{gp mark 1}{(5.420,1.563)}
\gppoint{gp mark 1}{(4.868,1.510)}
\gppoint{gp mark 1}{(4.631,1.478)}
\gppoint{gp mark 1}{(2.736,1.332)}
\gppoint{gp mark 1}{(6.447,1.628)}
\gppoint{gp mark 1}{(6.289,1.626)}
\gppoint{gp mark 1}{(5.104,1.541)}
\gppoint{gp mark 1}{(6.762,1.656)}
\gppoint{gp mark 1}{(2.341,1.312)}
\gppoint{gp mark 1}{(1.946,1.280)}
\gppoint{gp mark 1}{(1.788,1.252)}
\gppoint{gp mark 1}{(2.736,1.325)}
\gppoint{gp mark 1}{(2.499,1.300)}
\gppoint{gp mark 1}{(4.315,1.456)}
\gppoint{gp mark 1}{(2.578,1.332)}
\gppoint{gp mark 1}{(1.946,1.270)}
\gppoint{gp mark 1}{(3.131,1.365)}
\gppoint{gp mark 1}{(1.473,1.243)}
\gppoint{gp mark 1}{(2.578,1.323)}
\gppoint{gp mark 1}{(1.315,1.210)}
\gppoint{gp mark 1}{(1.078,1.192)}
\gppoint{gp mark 1}{(1.631,1.240)}
\gppoint{gp mark 1}{(5.183,1.540)}
\gppoint{gp mark 1}{(1.710,1.261)}
\gppoint{gp mark 1}{(2.657,1.335)}
\gppoint{gp mark 1}{(4.236,1.474)}
\gppoint{gp mark 1}{(4.552,1.469)}
\gppoint{gp mark 1}{(3.920,1.443)}
\gppoint{gp mark 1}{(3.683,1.426)}
\gppoint{gp mark 1}{(2.262,1.290)}
\gppoint{gp mark 1}{(2.736,1.336)}
\gppoint{gp mark 1}{(3.999,1.435)}
\gppoint{gp mark 1}{(3.052,1.353)}
\gppoint{gp mark 1}{(2.104,1.279)}
\gppoint{gp mark 1}{(1.473,1.237)}
\gppoint{gp mark 1}{(1.710,1.258)}
\gppoint{gp mark 1}{(1.315,1.216)}
\gppoint{gp mark 1}{(2.025,1.284)}
\gppoint{gp mark 1}{(1.710,1.255)}
\gppoint{gp mark 1}{(2.657,1.324)}
\gppoint{gp mark 1}{(1.788,1.272)}
\gppoint{gp mark 1}{(2.104,1.289)}
\gppoint{gp mark 1}{(1.788,1.258)}
\gppoint{gp mark 1}{(1.867,1.263)}
\gppoint{gp mark 1}{(1.710,1.258)}
\gppoint{gp mark 1}{(3.683,1.407)}
\gppoint{gp mark 1}{(2.894,1.337)}
\gppoint{gp mark 1}{(1.631,1.249)}
\gppoint{gp mark 1}{(1.710,1.249)}
\gppoint{gp mark 1}{(1.473,1.231)}
\gppoint{gp mark 1}{(3.367,1.390)}
\gppoint{gp mark 1}{(2.736,1.358)}
\gppoint{gp mark 1}{(2.183,1.304)}
\gppoint{gp mark 1}{(3.052,1.358)}
\gppoint{gp mark 1}{(2.025,1.288)}
\gppoint{gp mark 1}{(2.420,1.298)}
\gppoint{gp mark 1}{(4.315,1.453)}
\gppoint{gp mark 1}{(1.473,1.229)}
\gppoint{gp mark 1}{(4.789,1.503)}
\gppoint{gp mark 1}{(1.552,1.241)}
\gppoint{gp mark 1}{(4.631,1.483)}
\gppoint{gp mark 1}{(1.710,1.257)}
\gppoint{gp mark 1}{(2.736,1.332)}
\gppoint{gp mark 1}{(3.289,1.393)}
\gppoint{gp mark 1}{(4.552,1.489)}
\gppoint{gp mark 1}{(3.052,1.367)}
\gppoint{gp mark 1}{(4.157,1.473)}
\gppoint{gp mark 1}{(7.236,1.610)}
\gppoint{gp mark 1}{(2.657,1.322)}
\gppoint{gp mark 1}{(1.552,1.227)}
\gppoint{gp mark 1}{(1.788,1.257)}
\gppoint{gp mark 1}{(4.236,1.455)}
\gppoint{gp mark 1}{(1.710,1.240)}
\gppoint{gp mark 1}{(3.841,1.412)}
\gppoint{gp mark 1}{(3.210,1.363)}
\gppoint{gp mark 1}{(3.052,1.360)}
\gppoint{gp mark 1}{(2.815,1.325)}
\gppoint{gp mark 1}{(1.788,1.255)}
\gppoint{gp mark 1}{(2.341,1.301)}
\gppoint{gp mark 1}{(3.841,1.439)}
\gppoint{gp mark 1}{(2.736,1.344)}
\gppoint{gp mark 1}{(1.394,1.212)}
\gppoint{gp mark 1}{(2.341,1.301)}
\gppoint{gp mark 1}{(1.867,1.274)}
\gppoint{gp mark 1}{(2.815,1.348)}
\gppoint{gp mark 1}{(3.131,1.370)}
\gppoint{gp mark 1}{(2.815,1.352)}
\gppoint{gp mark 1}{(3.999,1.447)}
\gppoint{gp mark 1}{(2.341,1.313)}
\gppoint{gp mark 1}{(2.736,1.328)}
\gppoint{gp mark 1}{(1.710,1.246)}
\gppoint{gp mark 1}{(1.631,1.247)}
\gppoint{gp mark 1}{(2.815,1.342)}
\gppoint{gp mark 1}{(1.788,1.270)}
\gppoint{gp mark 1}{(1.631,1.261)}
\gppoint{gp mark 1}{(2.341,1.313)}
\gppoint{gp mark 1}{(4.236,1.487)}
\gppoint{gp mark 1}{(1.946,1.306)}
\gppoint{gp mark 1}{(1.867,1.282)}
\gppoint{gp mark 1}{(1.788,1.278)}
\gppoint{gp mark 1}{(1.710,1.264)}
\gppoint{gp mark 1}{(4.078,1.430)}
\gppoint{gp mark 1}{(4.078,1.450)}
\gppoint{gp mark 1}{(2.420,1.323)}
\gppoint{gp mark 1}{(3.367,1.390)}
\gppoint{gp mark 1}{(2.499,1.322)}
\gppoint{gp mark 1}{(4.552,1.502)}
\gppoint{gp mark 1}{(1.631,1.254)}
\gppoint{gp mark 1}{(1.236,1.210)}
\gppoint{gp mark 1}{(1.394,1.228)}
\gppoint{gp mark 1}{(1.631,1.251)}
\gppoint{gp mark 1}{(2.736,1.338)}
\gppoint{gp mark 1}{(1.236,1.220)}
\gppoint{gp mark 1}{(1.236,1.209)}
\gppoint{gp mark 1}{(1.867,1.273)}
\gppoint{gp mark 1}{(3.841,1.431)}
\gppoint{gp mark 1}{(3.762,1.431)}
\gppoint{gp mark 1}{(1.631,1.245)}
\gppoint{gp mark 1}{(2.025,1.282)}
\gppoint{gp mark 1}{(4.710,1.511)}
\gppoint{gp mark 1}{(2.578,1.330)}
\gppoint{gp mark 1}{(3.446,1.409)}
\gppoint{gp mark 1}{(2.815,1.351)}
\gppoint{gp mark 1}{(2.578,1.330)}
\gppoint{gp mark 1}{(2.736,1.347)}
\gppoint{gp mark 1}{(2.420,1.318)}
\gppoint{gp mark 1}{(3.683,1.425)}
\gppoint{gp mark 1}{(2.578,1.319)}
\gppoint{gp mark 1}{(2.341,1.300)}
\gppoint{gp mark 1}{(2.736,1.347)}
\gppoint{gp mark 1}{(4.473,1.474)}
\gppoint{gp mark 1}{(4.789,1.512)}
\gppoint{gp mark 1}{(3.604,1.409)}
\gppoint{gp mark 1}{(2.578,1.333)}
\gppoint{gp mark 1}{(7.789,1.606)}
\gppoint{gp mark 1}{(2.973,1.370)}
\gppoint{gp mark 1}{(6.052,1.628)}
\gppoint{gp mark 1}{(1.946,1.336)}
\gppoint{gp mark 1}{(3.999,1.453)}
\gppoint{gp mark 1}{(2.025,1.312)}
\gppoint{gp mark 1}{(2.104,1.284)}
\gppoint{gp mark 1}{(1.394,1.222)}
\gppoint{gp mark 1}{(2.262,1.297)}
\gppoint{gp mark 1}{(3.762,1.414)}
\gppoint{gp mark 1}{(2.815,1.340)}
\gppoint{gp mark 1}{(2.736,1.346)}
\gppoint{gp mark 1}{(2.183,1.314)}
\gppoint{gp mark 1}{(2.894,1.338)}
\gppoint{gp mark 1}{(3.210,1.388)}
\gppoint{gp mark 1}{(2.262,1.308)}
\gppoint{gp mark 1}{(1.946,1.287)}
\gppoint{gp mark 1}{(2.736,1.344)}
\gppoint{gp mark 1}{(2.262,1.293)}
\gppoint{gp mark 1}{(1.631,1.250)}
\gppoint{gp mark 1}{(5.262,1.549)}
\gppoint{gp mark 1}{(3.683,1.432)}
\gppoint{gp mark 1}{(3.131,1.372)}
\gppoint{gp mark 1}{(3.525,1.397)}
\gppoint{gp mark 1}{(3.446,1.401)}
\gppoint{gp mark 1}{(3.446,1.405)}
\gppoint{gp mark 1}{(2.578,1.338)}
\gppoint{gp mark 1}{(3.131,1.377)}
\gppoint{gp mark 1}{(3.210,1.381)}
\gppoint{gp mark 1}{(2.815,1.365)}
\gppoint{gp mark 1}{(4.078,1.456)}
\gppoint{gp mark 1}{(4.473,1.481)}
\gppoint{gp mark 1}{(7.315,1.597)}
\gppoint{gp mark 1}{(2.815,1.331)}
\gppoint{gp mark 1}{(6.289,1.610)}
\gppoint{gp mark 1}{(3.289,1.382)}
\gppoint{gp mark 1}{(4.236,1.460)}
\gppoint{gp mark 1}{(2.341,1.311)}
\gppoint{gp mark 1}{(3.052,1.360)}
\gppoint{gp mark 1}{(4.394,1.492)}
\gppoint{gp mark 1}{(3.210,1.390)}
\gppoint{gp mark 1}{(3.841,1.457)}
\gppoint{gp mark 1}{(1.867,1.278)}
\gppoint{gp mark 1}{(1.315,1.214)}
\gppoint{gp mark 1}{(3.367,1.385)}
\gppoint{gp mark 1}{(2.894,1.348)}
\gppoint{gp mark 1}{(6.525,1.629)}
\gppoint{gp mark 1}{(5.499,1.588)}
\gppoint{gp mark 1}{(3.920,1.445)}
\gppoint{gp mark 1}{(7.078,1.606)}
\gppoint{gp mark 1}{(6.604,1.613)}
\gppoint{gp mark 1}{(3.920,1.433)}
\gppoint{gp mark 1}{(3.131,1.379)}
\gppoint{gp mark 1}{(3.920,1.435)}
\gppoint{gp mark 1}{(1.788,1.269)}
\gppoint{gp mark 1}{(4.868,1.509)}
\gppoint{gp mark 1}{(6.525,1.615)}
\gppoint{gp mark 1}{(4.473,1.481)}
\gppoint{gp mark 1}{(4.946,1.518)}
\gppoint{gp mark 1}{(6.052,1.640)}
\gppoint{gp mark 1}{(2.262,1.319)}
\gppoint{gp mark 1}{(3.289,1.391)}
\gppoint{gp mark 1}{(3.525,1.396)}
\gppoint{gp mark 1}{(3.052,1.352)}
\gppoint{gp mark 1}{(2.025,1.283)}
\gppoint{gp mark 1}{(2.578,1.325)}
\gppoint{gp mark 1}{(5.973,1.616)}
\gppoint{gp mark 1}{(6.131,1.635)}
\gppoint{gp mark 1}{(2.578,1.331)}
\gppoint{gp mark 1}{(1.710,1.269)}
\gppoint{gp mark 1}{(3.683,1.420)}
\gppoint{gp mark 1}{(3.289,1.397)}
\gppoint{gp mark 1}{(4.236,1.475)}
\gppoint{gp mark 1}{(3.289,1.394)}
\gppoint{gp mark 1}{(3.131,1.361)}
\gppoint{gp mark 1}{(3.762,1.427)}
\gppoint{gp mark 1}{(1.552,1.270)}
\gppoint{gp mark 1}{(2.736,1.366)}
\gppoint{gp mark 1}{(2.973,1.373)}
\gppoint{gp mark 1}{(3.131,1.385)}
\gppoint{gp mark 1}{(4.473,1.497)}
\gppoint{gp mark 1}{(1.552,1.282)}
\gppoint{gp mark 1}{(2.736,1.351)}
\gppoint{gp mark 1}{(3.446,1.386)}
\gppoint{gp mark 1}{(3.210,1.380)}
\gppoint{gp mark 1}{(1.867,1.288)}
\gppoint{gp mark 1}{(1.867,1.284)}
\gppoint{gp mark 1}{(1.552,1.261)}
\gppoint{gp mark 1}{(3.210,1.376)}
\gppoint{gp mark 1}{(3.604,1.421)}
\gppoint{gp mark 1}{(2.815,1.350)}
\gppoint{gp mark 1}{(3.131,1.371)}
\gppoint{gp mark 1}{(2.341,1.300)}
\gppoint{gp mark 1}{(3.289,1.393)}
\gppoint{gp mark 1}{(1.867,1.283)}
\gppoint{gp mark 1}{(3.604,1.428)}
\gppoint{gp mark 1}{(3.762,1.428)}
\gppoint{gp mark 1}{(2.341,1.327)}
\gppoint{gp mark 1}{(1.710,1.270)}
\gppoint{gp mark 1}{(4.157,1.459)}
\gppoint{gp mark 1}{(1.788,1.273)}
\gppoint{gp mark 1}{(2.578,1.320)}
\gppoint{gp mark 1}{(3.131,1.369)}
\gppoint{gp mark 1}{(4.552,1.483)}
\gppoint{gp mark 1}{(2.973,1.364)}
\gppoint{gp mark 1}{(3.920,1.447)}
\gppoint{gp mark 1}{(1.710,1.273)}
\gppoint{gp mark 1}{(3.920,1.434)}
\gppoint{gp mark 1}{(3.604,1.388)}
\gppoint{gp mark 1}{(3.289,1.376)}
\gppoint{gp mark 1}{(3.446,1.408)}
\gppoint{gp mark 1}{(3.762,1.434)}
\gppoint{gp mark 1}{(3.525,1.418)}
\gppoint{gp mark 1}{(2.183,1.300)}
\gppoint{gp mark 1}{(2.104,1.287)}
\gppoint{gp mark 1}{(2.341,1.310)}
\gppoint{gp mark 1}{(5.183,1.583)}
\gppoint{gp mark 1}{(2.973,1.370)}
\gppoint{gp mark 1}{(2.025,1.294)}
\gppoint{gp mark 1}{(2.973,1.352)}
\gppoint{gp mark 1}{(1.867,1.283)}
\gppoint{gp mark 1}{(3.210,1.385)}
\gppoint{gp mark 1}{(3.999,1.448)}
\gppoint{gp mark 1}{(4.078,1.458)}
\gppoint{gp mark 1}{(5.420,1.585)}
\gppoint{gp mark 1}{(1.710,1.264)}
\gppoint{gp mark 1}{(2.499,1.324)}
\gppoint{gp mark 1}{(3.131,1.356)}
\gppoint{gp mark 1}{(2.183,1.287)}
\gppoint{gp mark 1}{(5.104,1.521)}
\gppoint{gp mark 1}{(4.946,1.509)}
\gppoint{gp mark 1}{(2.578,1.322)}
\gppoint{gp mark 1}{(2.341,1.313)}
\gppoint{gp mark 1}{(5.973,1.630)}
\gppoint{gp mark 1}{(2.104,1.302)}
\gppoint{gp mark 1}{(3.841,1.439)}
\gppoint{gp mark 1}{(4.552,1.516)}
\gppoint{gp mark 1}{(6.368,1.632)}
\gppoint{gp mark 1}{(2.894,1.376)}
\gppoint{gp mark 1}{(4.473,1.509)}
\gppoint{gp mark 1}{(4.236,1.480)}
\gppoint{gp mark 1}{(5.499,1.551)}
\gppoint{gp mark 1}{(5.104,1.541)}
\gppoint{gp mark 1}{(3.446,1.397)}
\gppoint{gp mark 1}{(3.762,1.431)}
\gppoint{gp mark 1}{(4.394,1.478)}
\gppoint{gp mark 1}{(3.841,1.451)}
\gppoint{gp mark 1}{(4.157,1.500)}
\gppoint{gp mark 1}{(3.999,1.454)}
\gppoint{gp mark 1}{(3.683,1.423)}
\gppoint{gp mark 1}{(4.315,1.512)}
\gppoint{gp mark 1}{(2.815,1.357)}
\gppoint{gp mark 1}{(2.262,1.308)}
\gppoint{gp mark 1}{(2.104,1.298)}
\gppoint{gp mark 1}{(2.104,1.300)}
\gppoint{gp mark 1}{(4.631,1.491)}
\gppoint{gp mark 1}{(4.315,1.477)}
\gppoint{gp mark 1}{(2.815,1.341)}
\gppoint{gp mark 1}{(4.315,1.475)}
\gppoint{gp mark 1}{(4.315,1.474)}
\gppoint{gp mark 1}{(5.499,1.566)}
\gppoint{gp mark 1}{(2.499,1.328)}
\gppoint{gp mark 1}{(3.131,1.378)}
\gppoint{gp mark 1}{(3.367,1.384)}
\gppoint{gp mark 1}{(3.367,1.365)}
\gppoint{gp mark 1}{(3.367,1.379)}
\gppoint{gp mark 1}{(1.473,1.243)}
\gppoint{gp mark 1}{(1.394,1.218)}
\gppoint{gp mark 1}{(1.946,1.264)}
\gppoint{gp mark 1}{(4.394,1.460)}
\gppoint{gp mark 1}{(4.315,1.462)}
\gppoint{gp mark 1}{(4.473,1.480)}
\gppoint{gp mark 1}{(1.788,1.263)}
\gppoint{gp mark 1}{(5.025,1.513)}
\gppoint{gp mark 1}{(3.210,1.369)}
\gppoint{gp mark 1}{(3.604,1.421)}
\gppoint{gp mark 1}{(1.473,1.248)}
\gppoint{gp mark 1}{(1.710,1.267)}
\gppoint{gp mark 1}{(2.578,1.325)}
\gppoint{gp mark 1}{(1.867,1.270)}
\gppoint{gp mark 1}{(2.262,1.300)}
\gppoint{gp mark 1}{(5.973,1.626)}
\gppoint{gp mark 1}{(3.920,1.449)}
\gppoint{gp mark 1}{(1.788,1.283)}
\gppoint{gp mark 1}{(3.762,1.430)}
\gppoint{gp mark 1}{(2.420,1.318)}
\gppoint{gp mark 1}{(1.946,1.271)}
\gppoint{gp mark 1}{(2.420,1.315)}
\gppoint{gp mark 1}{(2.894,1.357)}
\gppoint{gp mark 1}{(2.104,1.281)}
\gppoint{gp mark 1}{(1.867,1.268)}
\gppoint{gp mark 1}{(1.552,1.242)}
\gppoint{gp mark 1}{(1.552,1.241)}
\gppoint{gp mark 1}{(2.420,1.301)}
\gppoint{gp mark 1}{(1.631,1.240)}
\gppoint{gp mark 1}{(2.183,1.285)}
\gppoint{gp mark 1}{(2.420,1.313)}
\gppoint{gp mark 1}{(1.394,1.235)}
\gppoint{gp mark 1}{(2.262,1.299)}
\gppoint{gp mark 1}{(1.631,1.245)}
\gppoint{gp mark 1}{(4.157,1.440)}
\gppoint{gp mark 1}{(2.815,1.347)}
\gppoint{gp mark 1}{(2.894,1.346)}
\gppoint{gp mark 1}{(1.552,1.242)}
\gppoint{gp mark 1}{(3.604,1.400)}
\gppoint{gp mark 1}{(3.683,1.404)}
\gppoint{gp mark 1}{(3.367,1.383)}
\gppoint{gp mark 1}{(2.578,1.331)}
\gppoint{gp mark 1}{(1.473,1.241)}
\gppoint{gp mark 1}{(3.604,1.421)}
\gppoint{gp mark 1}{(1.473,1.257)}
\gppoint{gp mark 1}{(2.578,1.324)}
\gppoint{gp mark 1}{(1.710,1.246)}
\gppoint{gp mark 1}{(2.025,1.277)}
\gppoint{gp mark 1}{(2.736,1.328)}
\gppoint{gp mark 1}{(2.736,1.356)}
\gppoint{gp mark 1}{(2.025,1.297)}
\gppoint{gp mark 1}{(1.710,1.268)}
\gppoint{gp mark 1}{(3.289,1.395)}
\gppoint{gp mark 1}{(1.473,1.237)}
\gppoint{gp mark 1}{(2.262,1.292)}
\gppoint{gp mark 1}{(2.578,1.318)}
\gppoint{gp mark 1}{(2.736,1.331)}
\gppoint{gp mark 1}{(1.473,1.282)}
\gppoint{gp mark 1}{(2.341,1.333)}
\gppoint{gp mark 1}{(2.657,1.352)}
\gppoint{gp mark 1}{(1.473,1.230)}
\gppoint{gp mark 1}{(1.394,1.227)}
\gppoint{gp mark 1}{(1.236,1.207)}
\gppoint{gp mark 1}{(1.394,1.229)}
\gppoint{gp mark 1}{(1.631,1.250)}
\gppoint{gp mark 1}{(2.104,1.290)}
\gppoint{gp mark 1}{(1.236,1.204)}
\gppoint{gp mark 1}{(1.946,1.278)}
\gppoint{gp mark 1}{(1.867,1.276)}
\gppoint{gp mark 1}{(2.736,1.321)}
\gppoint{gp mark 1}{(3.131,1.363)}
\gppoint{gp mark 1}{(1.788,1.285)}
\gppoint{gp mark 1}{(1.315,1.215)}
\gppoint{gp mark 1}{(1.631,1.232)}
\gppoint{gp mark 1}{(4.157,1.446)}
\gppoint{gp mark 1}{(1.631,1.245)}
\gppoint{gp mark 1}{(1.552,1.230)}
\gppoint{gp mark 1}{(1.315,1.212)}
\gppoint{gp mark 1}{(3.604,1.404)}
\gppoint{gp mark 1}{(2.104,1.286)}
\gppoint{gp mark 1}{(2.736,1.338)}
\gppoint{gp mark 1}{(3.525,1.432)}
\gppoint{gp mark 1}{(1.710,1.249)}
\gppoint{gp mark 1}{(3.762,1.414)}
\gppoint{gp mark 1}{(1.867,1.268)}
\gppoint{gp mark 1}{(4.552,1.495)}
\gppoint{gp mark 1}{(2.025,1.294)}
\gppoint{gp mark 1}{(6.762,1.678)}
\gppoint{gp mark 1}{(1.157,1.219)}
\gppoint{gp mark 1}{(2.736,1.347)}
\gppoint{gp mark 1}{(5.025,1.533)}
\gppoint{gp mark 1}{(3.052,1.370)}
\gppoint{gp mark 1}{(2.262,1.320)}
\gppoint{gp mark 1}{(2.025,1.277)}
\gppoint{gp mark 1}{(2.183,1.289)}
\gppoint{gp mark 1}{(1.946,1.269)}
\gppoint{gp mark 1}{(1.394,1.217)}
\gppoint{gp mark 1}{(3.841,1.421)}
\gppoint{gp mark 1}{(1.552,1.227)}
\gppoint{gp mark 1}{(2.815,1.333)}
\gppoint{gp mark 1}{(5.578,1.563)}
\gppoint{gp mark 1}{(2.025,1.293)}
\gppoint{gp mark 1}{(8.262,1.595)}
\gppoint{gp mark 1}{(1.867,1.273)}
\gppoint{gp mark 1}{(2.657,1.329)}
\gppoint{gp mark 1}{(8.657,1.612)}
\gppoint{gp mark 1}{(3.210,1.391)}
\gppoint{gp mark 1}{(1.631,1.261)}
\gppoint{gp mark 1}{(1.788,1.261)}
\gppoint{gp mark 1}{(1.946,1.283)}
\gppoint{gp mark 1}{(2.815,1.351)}
\gppoint{gp mark 1}{(1.394,1.240)}
\gppoint{gp mark 1}{(1.867,1.279)}
\gppoint{gp mark 1}{(1.236,1.217)}
\gppoint{gp mark 1}{(1.236,1.214)}
\gppoint{gp mark 1}{(4.473,1.488)}
\gppoint{gp mark 1}{(1.867,1.280)}
\gppoint{gp mark 1}{(2.104,1.297)}
\gppoint{gp mark 1}{(3.210,1.393)}
\gppoint{gp mark 1}{(4.789,1.511)}
\gppoint{gp mark 1}{(4.394,1.496)}
\gppoint{gp mark 1}{(2.183,1.301)}
\gppoint{gp mark 1}{(3.683,1.433)}
\gppoint{gp mark 1}{(3.999,1.459)}
\gppoint{gp mark 1}{(2.657,1.327)}
\gppoint{gp mark 1}{(6.131,1.613)}
\gppoint{gp mark 1}{(4.236,1.519)}
\gppoint{gp mark 1}{(2.420,1.334)}
\gppoint{gp mark 1}{(2.104,1.296)}
\gppoint{gp mark 1}{(1.631,1.261)}
\gppoint{gp mark 1}{(4.078,1.473)}
\gppoint{gp mark 1}{(4.157,1.493)}
\gppoint{gp mark 1}{(4.394,1.496)}
\gppoint{gp mark 1}{(2.657,1.325)}
\gppoint{gp mark 1}{(2.183,1.297)}
\gppoint{gp mark 1}{(4.236,1.452)}
\gppoint{gp mark 1}{(1.394,1.229)}
\gppoint{gp mark 1}{(2.183,1.297)}
\gppoint{gp mark 1}{(1.236,1.209)}
\gppoint{gp mark 1}{(1.867,1.272)}
\gppoint{gp mark 1}{(3.841,1.436)}
\gppoint{gp mark 1}{(1.946,1.290)}
\gppoint{gp mark 1}{(1.710,1.273)}
\gppoint{gp mark 1}{(1.631,1.265)}
\gppoint{gp mark 1}{(1.867,1.278)}
\gppoint{gp mark 1}{(2.420,1.322)}
\gppoint{gp mark 1}{(4.789,1.520)}
\gppoint{gp mark 1}{(1.236,1.217)}
\gppoint{gp mark 1}{(2.025,1.284)}
\gppoint{gp mark 1}{(3.525,1.400)}
\gppoint{gp mark 1}{(3.920,1.441)}
\gppoint{gp mark 1}{(2.499,1.338)}
\gppoint{gp mark 1}{(4.710,1.515)}
\gppoint{gp mark 1}{(6.762,1.632)}
\gppoint{gp mark 1}{(3.289,1.402)}
\gppoint{gp mark 1}{(5.183,1.559)}
\gppoint{gp mark 1}{(2.657,1.352)}
\gppoint{gp mark 1}{(3.210,1.385)}
\gppoint{gp mark 1}{(2.578,1.327)}
\gppoint{gp mark 1}{(2.973,1.364)}
\gppoint{gp mark 1}{(2.183,1.308)}
\gppoint{gp mark 1}{(1.710,1.268)}
\gppoint{gp mark 1}{(3.367,1.394)}
\gppoint{gp mark 1}{(4.552,1.502)}
\gppoint{gp mark 1}{(1.473,1.248)}
\gppoint{gp mark 1}{(2.420,1.325)}
\gppoint{gp mark 1}{(2.341,1.319)}
\gppoint{gp mark 1}{(3.999,1.445)}
\gppoint{gp mark 1}{(4.394,1.468)}
\gppoint{gp mark 1}{(5.025,1.538)}
\gppoint{gp mark 1}{(3.683,1.461)}
\gppoint{gp mark 1}{(3.289,1.399)}
\gppoint{gp mark 1}{(2.736,1.354)}
\gppoint{gp mark 1}{(3.920,1.442)}
\gppoint{gp mark 1}{(3.052,1.372)}
\gppoint{gp mark 1}{(3.289,1.396)}
\gppoint{gp mark 1}{(5.025,1.508)}
\gppoint{gp mark 1}{(4.394,1.484)}
\gppoint{gp mark 1}{(3.920,1.462)}
\gppoint{gp mark 1}{(4.315,1.494)}
\gppoint{gp mark 1}{(3.604,1.441)}
\gppoint{gp mark 1}{(2.973,1.360)}
\gppoint{gp mark 1}{(3.289,1.373)}
\gppoint{gp mark 1}{(2.657,1.340)}
\gppoint{gp mark 1}{(4.157,1.488)}
\gppoint{gp mark 1}{(3.762,1.460)}
\gppoint{gp mark 1}{(3.210,1.394)}
\gppoint{gp mark 1}{(3.289,1.405)}
\gppoint{gp mark 1}{(2.973,1.378)}
\gppoint{gp mark 1}{(1.236,1.227)}
\gppoint{gp mark 1}{(2.499,1.329)}
\gppoint{gp mark 1}{(2.104,1.298)}
\gppoint{gp mark 1}{(3.999,1.456)}
\gppoint{gp mark 1}{(2.578,1.342)}
\gppoint{gp mark 1}{(1.946,1.297)}
\gppoint{gp mark 1}{(3.525,1.407)}
\gppoint{gp mark 1}{(1.552,1.263)}
\gppoint{gp mark 1}{(2.104,1.320)}
\gppoint{gp mark 1}{(1.710,1.291)}
\gppoint{gp mark 1}{(3.446,1.393)}
\gppoint{gp mark 1}{(1.552,1.238)}
\gppoint{gp mark 1}{(1.867,1.274)}
\gppoint{gp mark 1}{(2.736,1.354)}
\gppoint{gp mark 1}{(2.973,1.383)}
\gppoint{gp mark 1}{(2.499,1.351)}
\gppoint{gp mark 1}{(2.183,1.311)}
\gppoint{gp mark 1}{(2.025,1.295)}
\gppoint{gp mark 1}{(1.394,1.233)}
\gppoint{gp mark 1}{(3.131,1.376)}
\gppoint{gp mark 1}{(2.183,1.319)}
\gppoint{gp mark 1}{(2.894,1.371)}
\gppoint{gp mark 1}{(1.552,1.252)}
\gppoint{gp mark 1}{(2.262,1.300)}
\gppoint{gp mark 1}{(3.367,1.386)}
\gppoint{gp mark 1}{(4.157,1.455)}
\gppoint{gp mark 1}{(2.420,1.314)}
\gppoint{gp mark 1}{(3.604,1.427)}
\gppoint{gp mark 1}{(3.367,1.414)}
\gppoint{gp mark 1}{(1.631,1.261)}
\gppoint{gp mark 1}{(2.104,1.297)}
\gppoint{gp mark 1}{(4.157,1.481)}
\gppoint{gp mark 1}{(3.210,1.400)}
\gppoint{gp mark 1}{(2.973,1.366)}
\gppoint{gp mark 1}{(3.289,1.394)}
\gppoint{gp mark 1}{(2.341,1.322)}
\gppoint{gp mark 1}{(1.710,1.263)}
\gppoint{gp mark 1}{(1.631,1.248)}
\gppoint{gp mark 1}{(5.341,1.555)}
\gppoint{gp mark 1}{(3.446,1.408)}
\gppoint{gp mark 1}{(1.946,1.278)}
\gppoint{gp mark 1}{(2.183,1.297)}
\gppoint{gp mark 1}{(2.341,1.309)}
\gppoint{gp mark 1}{(3.367,1.384)}
\gppoint{gp mark 1}{(2.262,1.325)}
\gppoint{gp mark 1}{(1.631,1.268)}
\gppoint{gp mark 1}{(1.552,1.259)}
\gppoint{gp mark 1}{(3.446,1.410)}
\gppoint{gp mark 1}{(2.657,1.350)}
\gppoint{gp mark 1}{(2.420,1.326)}
\gppoint{gp mark 1}{(3.999,1.458)}
\gppoint{gp mark 1}{(1.473,1.249)}
\gppoint{gp mark 1}{(2.736,1.353)}
\gppoint{gp mark 1}{(2.973,1.393)}
\gppoint{gp mark 1}{(3.367,1.391)}
\gppoint{gp mark 1}{(2.025,1.297)}
\gppoint{gp mark 1}{(1.867,1.280)}
\gppoint{gp mark 1}{(2.657,1.345)}
\gppoint{gp mark 1}{(3.131,1.395)}
\gppoint{gp mark 1}{(3.920,1.466)}
\gppoint{gp mark 1}{(2.578,1.340)}
\gppoint{gp mark 1}{(3.525,1.431)}
\gppoint{gp mark 1}{(1.946,1.284)}
\gppoint{gp mark 1}{(2.499,1.329)}
\gppoint{gp mark 1}{(2.657,1.332)}
\gppoint{gp mark 1}{(2.973,1.358)}
\gppoint{gp mark 1}{(1.710,1.268)}
\gppoint{gp mark 1}{(2.894,1.360)}
\gppoint{gp mark 1}{(2.025,1.286)}
\gppoint{gp mark 1}{(1.867,1.284)}
\gppoint{gp mark 1}{(2.262,1.312)}
\gppoint{gp mark 1}{(2.736,1.363)}
\gppoint{gp mark 1}{(2.499,1.327)}
\gppoint{gp mark 1}{(3.762,1.448)}
\gppoint{gp mark 1}{(3.525,1.396)}
\gppoint{gp mark 1}{(1.552,1.250)}
\gppoint{gp mark 1}{(4.236,1.478)}
\gppoint{gp mark 1}{(3.052,1.383)}
\gppoint{gp mark 1}{(3.367,1.422)}
\gppoint{gp mark 1}{(2.499,1.344)}
\gppoint{gp mark 1}{(4.078,1.475)}
\gppoint{gp mark 1}{(1.788,1.288)}
\gppoint{gp mark 1}{(1.788,1.286)}
\gppoint{gp mark 1}{(1.473,1.257)}
\gppoint{gp mark 1}{(2.420,1.333)}
\gppoint{gp mark 1}{(3.525,1.432)}
\gppoint{gp mark 1}{(3.131,1.393)}
\gppoint{gp mark 1}{(2.341,1.325)}
\gppoint{gp mark 1}{(3.525,1.430)}
\gppoint{gp mark 1}{(4.078,1.478)}
\gppoint{gp mark 1}{(2.341,1.317)}
\gppoint{gp mark 1}{(1.552,1.261)}
\gppoint{gp mark 1}{(2.262,1.316)}
\gppoint{gp mark 1}{(1.946,1.300)}
\gppoint{gp mark 1}{(3.131,1.390)}
\gppoint{gp mark 1}{(1.631,1.274)}
\gppoint{gp mark 1}{(4.157,1.481)}
\gppoint{gp mark 1}{(2.894,1.375)}
\gppoint{gp mark 1}{(2.025,1.301)}
\gppoint{gp mark 1}{(2.341,1.306)}
\gppoint{gp mark 1}{(2.736,1.362)}
\gppoint{gp mark 1}{(2.657,1.349)}
\gppoint{gp mark 1}{(2.736,1.350)}
\gppoint{gp mark 1}{(2.262,1.319)}
\gppoint{gp mark 1}{(1.946,1.316)}
\gppoint{gp mark 1}{(1.867,1.295)}
\gppoint{gp mark 1}{(1.552,1.251)}
\gppoint{gp mark 1}{(1.631,1.259)}
\gppoint{gp mark 1}{(3.920,1.435)}
\gppoint{gp mark 1}{(1.552,1.252)}
\gppoint{gp mark 1}{(2.104,1.297)}
\gppoint{gp mark 1}{(1.710,1.275)}
\gppoint{gp mark 1}{(2.736,1.349)}
\gppoint{gp mark 1}{(2.262,1.321)}
\gppoint{gp mark 1}{(2.025,1.303)}
\gppoint{gp mark 1}{(2.104,1.307)}
\gppoint{gp mark 1}{(1.788,1.285)}
\gppoint{gp mark 1}{(2.183,1.301)}
\gppoint{gp mark 1}{(2.341,1.327)}
\gppoint{gp mark 1}{(2.025,1.295)}
\gppoint{gp mark 1}{(1.315,1.223)}
\gppoint{gp mark 1}{(2.183,1.305)}
\gppoint{gp mark 1}{(1.710,1.253)}
\gppoint{gp mark 1}{(1.710,1.268)}
\gppoint{gp mark 1}{(2.183,1.302)}
\gppoint{gp mark 1}{(1.946,1.293)}
\gppoint{gp mark 1}{(1.552,1.255)}
\gppoint{gp mark 1}{(2.341,1.319)}
\gppoint{gp mark 1}{(3.367,1.410)}
\gppoint{gp mark 1}{(3.446,1.412)}
\gppoint{gp mark 1}{(1.788,1.276)}
\gppoint{gp mark 1}{(4.789,1.514)}
\gppoint{gp mark 1}{(2.025,1.306)}
\gppoint{gp mark 1}{(2.815,1.351)}
\gppoint{gp mark 1}{(3.289,1.408)}
\gppoint{gp mark 1}{(1.473,1.257)}
\gppoint{gp mark 1}{(2.578,1.339)}
\gppoint{gp mark 1}{(1.236,1.222)}
\gppoint{gp mark 1}{(3.446,1.397)}
\gppoint{gp mark 1}{(3.289,1.376)}
\gppoint{gp mark 1}{(1.946,1.292)}
\gppoint{gp mark 1}{(2.657,1.343)}
\gppoint{gp mark 1}{(1.631,1.271)}
\gppoint{gp mark 1}{(2.262,1.315)}
\gppoint{gp mark 1}{(2.815,1.357)}
\gppoint{gp mark 1}{(2.341,1.324)}
\gppoint{gp mark 1}{(2.104,1.295)}
\gppoint{gp mark 1}{(1.473,1.249)}
\gppoint{gp mark 1}{(2.657,1.341)}
\gppoint{gp mark 1}{(2.499,1.338)}
\gppoint{gp mark 1}{(2.341,1.355)}
\gppoint{gp mark 1}{(2.420,1.326)}
\gppoint{gp mark 1}{(1.946,1.304)}
\gppoint{gp mark 1}{(2.025,1.309)}
\gppoint{gp mark 1}{(1.867,1.293)}
\gppoint{gp mark 1}{(2.183,1.298)}
\gppoint{gp mark 1}{(1.946,1.293)}
\gppoint{gp mark 1}{(2.894,1.364)}
\gppoint{gp mark 1}{(1.315,1.252)}
\gppoint{gp mark 1}{(2.104,1.283)}
\gppoint{gp mark 1}{(2.183,1.307)}
\gppoint{gp mark 1}{(2.815,1.342)}
\gppoint{gp mark 1}{(1.867,1.280)}
\gppoint{gp mark 1}{(2.736,1.341)}
\gppoint{gp mark 1}{(1.631,1.257)}
\gppoint{gp mark 1}{(2.420,1.328)}
\gppoint{gp mark 1}{(2.025,1.310)}
\gppoint{gp mark 1}{(2.894,1.365)}
\gppoint{gp mark 1}{(2.341,1.331)}
\gppoint{gp mark 1}{(2.341,1.321)}
\gppoint{gp mark 1}{(2.657,1.346)}
\gppoint{gp mark 1}{(2.262,1.306)}
\gppoint{gp mark 1}{(1.473,1.260)}
\gppoint{gp mark 1}{(1.473,1.249)}
\gppoint{gp mark 1}{(2.183,1.308)}
\gppoint{gp mark 1}{(2.578,1.330)}
\gppoint{gp mark 1}{(2.657,1.326)}
\gppoint{gp mark 1}{(2.578,1.340)}
\gppoint{gp mark 1}{(1.631,1.268)}
\gppoint{gp mark 1}{(2.183,1.308)}
\gppoint{gp mark 1}{(3.052,1.390)}
\gppoint{gp mark 1}{(2.420,1.330)}
\gppoint{gp mark 1}{(3.525,1.423)}
\gppoint{gp mark 1}{(3.446,1.421)}
\gppoint{gp mark 1}{(3.289,1.391)}
\gppoint{gp mark 1}{(3.683,1.425)}
\gppoint{gp mark 1}{(3.210,1.373)}
\gppoint{gp mark 1}{(3.525,1.430)}
\gppoint{gp mark 1}{(2.815,1.374)}
\gppoint{gp mark 1}{(1.315,1.258)}
\gppoint{gp mark 1}{(2.025,1.293)}
\gppoint{gp mark 1}{(4.394,1.476)}
\gppoint{gp mark 1}{(2.973,1.374)}
\gppoint{gp mark 1}{(3.210,1.426)}
\gppoint{gp mark 1}{(2.657,1.347)}
\gppoint{gp mark 1}{(2.104,1.305)}
\gppoint{gp mark 1}{(1.631,1.273)}
\gppoint{gp mark 1}{(2.815,1.358)}
\gppoint{gp mark 1}{(5.341,1.573)}
\gppoint{gp mark 1}{(2.420,1.328)}
\gppoint{gp mark 1}{(2.262,1.328)}
\gppoint{gp mark 1}{(1.631,1.276)}
\gppoint{gp mark 1}{(3.289,1.403)}
\gppoint{gp mark 1}{(2.657,1.344)}
\gppoint{gp mark 1}{(2.736,1.353)}
\gppoint{gp mark 1}{(2.025,1.293)}
\gppoint{gp mark 1}{(2.183,1.304)}
\gppoint{gp mark 1}{(2.420,1.320)}
\gppoint{gp mark 1}{(2.183,1.308)}
\gppoint{gp mark 1}{(3.683,1.428)}
\gppoint{gp mark 1}{(4.394,1.467)}
\gppoint{gp mark 1}{(1.394,1.245)}
\gppoint{gp mark 1}{(3.289,1.386)}
\gppoint{gp mark 1}{(1.315,1.230)}
\gppoint{gp mark 1}{(1.788,1.277)}
\gppoint{gp mark 1}{(2.657,1.334)}
\gppoint{gp mark 1}{(3.210,1.387)}
\gppoint{gp mark 1}{(1.394,1.253)}
\gppoint{gp mark 1}{(2.973,1.371)}
\gppoint{gp mark 1}{(5.104,1.560)}
\gppoint{gp mark 1}{(3.367,1.418)}
\gppoint{gp mark 1}{(3.367,1.408)}
\gppoint{gp mark 1}{(3.683,1.420)}
\gppoint{gp mark 1}{(3.367,1.389)}
\gppoint{gp mark 1}{(2.183,1.327)}
\gppoint{gp mark 1}{(2.420,1.319)}
\gppoint{gp mark 1}{(2.025,1.296)}
\gppoint{gp mark 1}{(2.973,1.364)}
\gppoint{gp mark 1}{(2.420,1.326)}
\gppoint{gp mark 1}{(3.131,1.378)}
\gppoint{gp mark 1}{(1.236,1.245)}
\gppoint{gp mark 1}{(2.183,1.339)}
\gppoint{gp mark 1}{(1.946,1.310)}
\gppoint{gp mark 1}{(1.946,1.301)}
\gppoint{gp mark 1}{(2.815,1.383)}
\gppoint{gp mark 1}{(2.104,1.348)}
\gppoint{gp mark 1}{(2.578,1.333)}
\gppoint{gp mark 1}{(1.552,1.244)}
\gppoint{gp mark 1}{(4.315,1.466)}
\gppoint{gp mark 1}{(2.262,1.308)}
\gppoint{gp mark 1}{(2.025,1.305)}
\gppoint{gp mark 1}{(3.052,1.379)}
\gppoint{gp mark 1}{(3.604,1.432)}
\gppoint{gp mark 1}{(2.183,1.308)}
\gppoint{gp mark 1}{(1.631,1.273)}
\gppoint{gp mark 1}{(2.499,1.342)}
\gppoint{gp mark 1}{(2.341,1.324)}
\gppoint{gp mark 1}{(3.131,1.393)}
\gppoint{gp mark 1}{(2.183,1.306)}
\gppoint{gp mark 1}{(5.183,1.559)}
\gppoint{gp mark 1}{(2.499,1.327)}
\gppoint{gp mark 1}{(4.157,1.482)}
\gppoint{gp mark 1}{(2.657,1.356)}
\gppoint{gp mark 1}{(2.499,1.337)}
\gppoint{gp mark 1}{(2.025,1.290)}
\gppoint{gp mark 1}{(2.736,1.345)}
\gppoint{gp mark 1}{(1.473,1.257)}
\gppoint{gp mark 1}{(1.552,1.242)}
\gppoint{gp mark 1}{(2.499,1.326)}
\gppoint{gp mark 1}{(1.552,1.265)}
\gppoint{gp mark 1}{(4.236,1.483)}
\gppoint{gp mark 1}{(2.262,1.323)}
\gppoint{gp mark 1}{(2.183,1.320)}
\gppoint{gp mark 1}{(2.341,1.331)}
\gppoint{gp mark 1}{(5.341,1.576)}
\gppoint{gp mark 1}{(1.631,1.274)}
\gppoint{gp mark 1}{(2.183,1.305)}
\gppoint{gp mark 1}{(2.025,1.304)}
\gppoint{gp mark 1}{(1.473,1.261)}
\gppoint{gp mark 1}{(2.262,1.322)}
\gppoint{gp mark 1}{(1.394,1.248)}
\gppoint{gp mark 1}{(3.446,1.413)}
\gppoint{gp mark 1}{(5.420,1.575)}
\gppoint{gp mark 1}{(1.867,1.282)}
\gppoint{gp mark 1}{(1.867,1.281)}
\gppoint{gp mark 1}{(3.525,1.416)}
\gppoint{gp mark 1}{(3.210,1.389)}
\gppoint{gp mark 1}{(3.683,1.430)}
\gppoint{gp mark 1}{(2.262,1.322)}
\gppoint{gp mark 1}{(1.710,1.284)}
\gppoint{gp mark 1}{(3.604,1.429)}
\gppoint{gp mark 1}{(2.025,1.316)}
\gppoint{gp mark 1}{(1.710,1.261)}
\gppoint{gp mark 1}{(2.341,1.305)}
\gppoint{gp mark 1}{(3.683,1.413)}
\gppoint{gp mark 1}{(2.025,1.308)}
\gppoint{gp mark 1}{(2.815,1.350)}
\gppoint{gp mark 1}{(3.210,1.385)}
\gppoint{gp mark 1}{(4.236,1.495)}
\gppoint{gp mark 1}{(3.289,1.403)}
\gppoint{gp mark 1}{(5.894,1.605)}
\gppoint{gp mark 1}{(3.131,1.377)}
\gppoint{gp mark 1}{(4.157,1.445)}
\gppoint{gp mark 1}{(4.315,1.468)}
\gppoint{gp mark 1}{(3.052,1.357)}
\gppoint{gp mark 1}{(2.025,1.282)}
\gppoint{gp mark 1}{(3.841,1.434)}
\gppoint{gp mark 1}{(5.262,1.549)}
\gppoint{gp mark 1}{(2.894,1.364)}
\gppoint{gp mark 1}{(8.420,1.642)}
\gppoint{gp mark 1}{(3.289,1.438)}
\gppoint{gp mark 1}{(2.025,1.290)}
\gppoint{gp mark 1}{(1.552,1.254)}
\gppoint{gp mark 1}{(2.973,1.362)}
\gppoint{gp mark 1}{(2.973,1.363)}
\gppoint{gp mark 1}{(4.236,1.492)}
\gppoint{gp mark 1}{(2.657,1.351)}
\gppoint{gp mark 1}{(2.894,1.371)}
\gppoint{gp mark 1}{(2.025,1.307)}
\gppoint{gp mark 1}{(1.473,1.257)}
\gppoint{gp mark 1}{(1.394,1.231)}
\gppoint{gp mark 1}{(1.394,1.237)}
\gppoint{gp mark 1}{(2.578,1.330)}
\gppoint{gp mark 1}{(2.973,1.363)}
\gppoint{gp mark 1}{(2.420,1.325)}
\gppoint{gp mark 1}{(1.631,1.257)}
\gppoint{gp mark 1}{(3.999,1.443)}
\gppoint{gp mark 1}{(1.631,1.273)}
\gppoint{gp mark 1}{(1.788,1.272)}
\gppoint{gp mark 1}{(2.578,1.329)}
\gppoint{gp mark 1}{(4.157,1.450)}
\gppoint{gp mark 1}{(2.499,1.312)}
\gppoint{gp mark 1}{(2.262,1.297)}
\gppoint{gp mark 1}{(1.552,1.259)}
\gppoint{gp mark 1}{(4.078,1.443)}
\gppoint{gp mark 1}{(4.552,1.499)}
\gppoint{gp mark 1}{(1.710,1.268)}
\gppoint{gp mark 1}{(2.025,1.290)}
\gppoint{gp mark 1}{(2.262,1.308)}
\gppoint{gp mark 1}{(1.710,1.278)}
\gppoint{gp mark 1}{(3.604,1.416)}
\gppoint{gp mark 1}{(2.025,1.283)}
\gppoint{gp mark 1}{(3.289,1.379)}
\gppoint{gp mark 1}{(1.867,1.268)}
\gppoint{gp mark 1}{(3.999,1.445)}
\gppoint{gp mark 1}{(2.420,1.312)}
\gppoint{gp mark 1}{(2.262,1.302)}
\gppoint{gp mark 1}{(3.289,1.381)}
\gppoint{gp mark 1}{(5.341,1.564)}
\gppoint{gp mark 1}{(3.999,1.447)}
\gppoint{gp mark 1}{(1.473,1.233)}
\gppoint{gp mark 1}{(2.025,1.290)}
\gppoint{gp mark 1}{(1.473,1.272)}
\gppoint{gp mark 1}{(3.525,1.411)}
\gppoint{gp mark 1}{(1.157,1.212)}
\gppoint{gp mark 1}{(2.341,1.315)}
\gppoint{gp mark 1}{(3.289,1.389)}
\gppoint{gp mark 1}{(2.183,1.292)}
\gppoint{gp mark 1}{(3.052,1.362)}
\gppoint{gp mark 1}{(4.789,1.518)}
\gppoint{gp mark 1}{(2.815,1.348)}
\gppoint{gp mark 1}{(1.788,1.268)}
\gppoint{gp mark 1}{(2.183,1.295)}
\gppoint{gp mark 1}{(2.657,1.327)}
\gppoint{gp mark 1}{(3.052,1.368)}
\gppoint{gp mark 1}{(1.631,1.254)}
\gppoint{gp mark 1}{(2.025,1.290)}
\gppoint{gp mark 1}{(1.867,1.271)}
\gppoint{gp mark 1}{(2.341,1.304)}
\gppoint{gp mark 1}{(2.894,1.348)}
\gppoint{gp mark 1}{(2.262,1.303)}
\gppoint{gp mark 1}{(2.499,1.332)}
\gppoint{gp mark 1}{(1.473,1.248)}
\gppoint{gp mark 1}{(2.499,1.339)}
\gppoint{gp mark 1}{(2.025,1.309)}
\gppoint{gp mark 1}{(3.841,1.451)}
\gppoint{gp mark 1}{(2.104,1.295)}
\gppoint{gp mark 1}{(1.788,1.271)}
\gppoint{gp mark 1}{(3.525,1.407)}
\gppoint{gp mark 1}{(2.262,1.322)}
\gppoint{gp mark 1}{(5.420,1.581)}
\gppoint{gp mark 1}{(3.446,1.414)}
\gppoint{gp mark 1}{(2.736,1.361)}
\gppoint{gp mark 1}{(4.157,1.476)}
\gppoint{gp mark 1}{(1.867,1.286)}
\gppoint{gp mark 1}{(2.341,1.322)}
\gppoint{gp mark 1}{(1.946,1.299)}
\gppoint{gp mark 1}{(1.473,1.241)}
\gppoint{gp mark 1}{(1.394,1.238)}
\gppoint{gp mark 1}{(2.657,1.344)}
\gppoint{gp mark 1}{(2.025,1.292)}
\gppoint{gp mark 1}{(3.683,1.423)}
\gppoint{gp mark 1}{(2.499,1.340)}
\gppoint{gp mark 1}{(1.552,1.264)}
\gppoint{gp mark 1}{(1.946,1.298)}
\gppoint{gp mark 1}{(2.025,1.322)}
\gppoint{gp mark 1}{(3.525,1.451)}
\gppoint{gp mark 1}{(4.552,1.518)}
\gppoint{gp mark 1}{(2.657,1.357)}
\gppoint{gp mark 1}{(2.183,1.315)}
\gppoint{gp mark 1}{(2.025,1.308)}
\gppoint{gp mark 1}{(3.446,1.423)}
\gppoint{gp mark 1}{(2.025,1.297)}
\gppoint{gp mark 1}{(1.394,1.256)}
\gppoint{gp mark 1}{(2.657,1.353)}
\gppoint{gp mark 1}{(4.236,1.495)}
\gppoint{gp mark 1}{(4.078,1.463)}
\gppoint{gp mark 1}{(2.183,1.330)}
\gppoint{gp mark 1}{(2.262,1.305)}
\gppoint{gp mark 1}{(3.367,1.406)}
\gppoint{gp mark 1}{(1.867,1.274)}
\gppoint{gp mark 1}{(2.341,1.319)}
\gppoint{gp mark 1}{(3.052,1.383)}
\gppoint{gp mark 1}{(2.736,1.355)}
\gppoint{gp mark 1}{(1.552,1.243)}
\gppoint{gp mark 1}{(6.289,1.599)}
\gppoint{gp mark 1}{(4.710,1.521)}
\gppoint{gp mark 1}{(1.631,1.252)}
\gppoint{gp mark 1}{(1.394,1.252)}
\gppoint{gp mark 1}{(4.078,1.476)}
\gppoint{gp mark 1}{(2.025,1.335)}
\gppoint{gp mark 1}{(1.473,1.292)}
\gppoint{gp mark 1}{(2.736,1.382)}
\gppoint{gp mark 1}{(3.762,1.471)}
\gppoint{gp mark 1}{(3.920,1.484)}
\gppoint{gp mark 1}{(3.367,1.422)}
\gppoint{gp mark 1}{(1.867,1.283)}
\gppoint{gp mark 1}{(3.762,1.438)}
\gppoint{gp mark 1}{(6.683,1.653)}
\gppoint{gp mark 1}{(2.973,1.384)}
\gppoint{gp mark 1}{(2.578,1.365)}
\gppoint{gp mark 1}{(3.999,1.453)}
\gppoint{gp mark 1}{(3.289,1.410)}
\gppoint{gp mark 1}{(3.446,1.407)}
\gppoint{gp mark 1}{(4.473,1.484)}
\gppoint{gp mark 1}{(1.473,1.265)}
\gppoint{gp mark 1}{(2.657,1.366)}
\gppoint{gp mark 1}{(1.631,1.252)}
\gppoint{gp mark 1}{(1.631,1.262)}
\gppoint{gp mark 1}{(1.788,1.281)}
\gppoint{gp mark 1}{(1.946,1.294)}
\gppoint{gp mark 1}{(2.499,1.340)}
\gppoint{gp mark 1}{(2.104,1.318)}
\gppoint{gp mark 1}{(2.262,1.310)}
\gppoint{gp mark 1}{(2.499,1.347)}
\gppoint{gp mark 1}{(1.552,1.268)}
\gppoint{gp mark 1}{(1.710,1.282)}
\gppoint{gp mark 1}{(1.631,1.272)}
\gppoint{gp mark 1}{(2.183,1.310)}
\gppoint{gp mark 1}{(1.710,1.286)}
\gppoint{gp mark 1}{(2.025,1.313)}
\gppoint{gp mark 1}{(3.604,1.431)}
\gppoint{gp mark 1}{(1.710,1.272)}
\gppoint{gp mark 1}{(1.394,1.246)}
\gppoint{gp mark 1}{(2.578,1.343)}
\gppoint{gp mark 1}{(2.499,1.349)}
\gppoint{gp mark 1}{(2.025,1.297)}
\gppoint{gp mark 1}{(3.210,1.387)}
\gppoint{gp mark 1}{(1.394,1.258)}
\gppoint{gp mark 1}{(1.867,1.292)}
\gppoint{gp mark 1}{(1.946,1.299)}
\gppoint{gp mark 1}{(1.394,1.239)}
\gppoint{gp mark 1}{(1.710,1.272)}
\gppoint{gp mark 1}{(1.867,1.295)}
\gppoint{gp mark 1}{(3.210,1.388)}
\gppoint{gp mark 1}{(3.131,1.380)}
\gppoint{gp mark 1}{(2.262,1.328)}
\gppoint{gp mark 1}{(2.341,1.330)}
\gppoint{gp mark 1}{(2.262,1.309)}
\gppoint{gp mark 1}{(1.788,1.276)}
\gppoint{gp mark 1}{(2.025,1.305)}
\gppoint{gp mark 1}{(2.262,1.299)}
\gppoint{gp mark 1}{(3.131,1.380)}
\gppoint{gp mark 1}{(2.736,1.344)}
\gppoint{gp mark 1}{(2.420,1.324)}
\gppoint{gp mark 1}{(2.104,1.302)}
\gppoint{gp mark 1}{(3.052,1.352)}
\gppoint{gp mark 1}{(1.788,1.266)}
\gppoint{gp mark 1}{(3.446,1.407)}
\gppoint{gp mark 1}{(1.631,1.289)}
\gppoint{gp mark 1}{(2.736,1.340)}
\gppoint{gp mark 1}{(1.315,1.224)}
\gppoint{gp mark 1}{(2.657,1.347)}
\gppoint{gp mark 1}{(1.788,1.296)}
\gppoint{gp mark 1}{(2.657,1.359)}
\gppoint{gp mark 1}{(2.341,1.333)}
\gppoint{gp mark 1}{(1.473,1.266)}
\gppoint{gp mark 1}{(2.420,1.343)}
\gppoint{gp mark 1}{(2.183,1.320)}
\gppoint{gp mark 1}{(1.631,1.263)}
\gppoint{gp mark 1}{(1.710,1.275)}
\gppoint{gp mark 1}{(1.710,1.275)}
\gppoint{gp mark 1}{(3.367,1.422)}
\gppoint{gp mark 1}{(1.710,1.287)}
\gppoint{gp mark 1}{(2.420,1.339)}
\gppoint{gp mark 1}{(3.367,1.402)}
\gppoint{gp mark 1}{(3.920,1.456)}
\gppoint{gp mark 1}{(3.525,1.445)}
\gppoint{gp mark 1}{(3.920,1.506)}
\gppoint{gp mark 1}{(1.473,1.299)}
\gppoint{gp mark 1}{(2.104,1.299)}
\gppoint{gp mark 1}{(2.183,1.305)}
\gppoint{gp mark 1}{(5.973,1.624)}
\gppoint{gp mark 1}{(3.920,1.469)}
\gppoint{gp mark 1}{(4.789,1.496)}
\gppoint{gp mark 1}{(3.210,1.384)}
\gppoint{gp mark 1}{(3.131,1.394)}
\gppoint{gp mark 1}{(4.631,1.509)}
\gppoint{gp mark 1}{(3.683,1.434)}
\gppoint{gp mark 1}{(3.604,1.450)}
\gppoint{gp mark 1}{(2.183,1.318)}
\gppoint{gp mark 1}{(4.552,1.507)}
\gppoint{gp mark 1}{(2.894,1.380)}
\gppoint{gp mark 1}{(3.762,1.454)}
\gppoint{gp mark 1}{(2.657,1.331)}
\gppoint{gp mark 1}{(2.183,1.307)}
\gppoint{gp mark 1}{(3.604,1.416)}
\gppoint{gp mark 1}{(3.052,1.394)}
\gppoint{gp mark 1}{(4.710,1.554)}
\gppoint{gp mark 1}{(2.499,1.333)}
\gppoint{gp mark 1}{(1.710,1.258)}
\gppoint{gp mark 1}{(4.078,1.455)}
\gppoint{gp mark 1}{(3.289,1.407)}
\gppoint{gp mark 1}{(3.367,1.392)}
\gppoint{gp mark 1}{(2.104,1.304)}
\gppoint{gp mark 1}{(3.604,1.412)}
\gppoint{gp mark 1}{(4.078,1.463)}
\gppoint{gp mark 1}{(1.946,1.309)}
\gppoint{gp mark 1}{(3.446,1.424)}
\gppoint{gp mark 1}{(3.446,1.418)}
\gppoint{gp mark 1}{(4.236,1.481)}
\gppoint{gp mark 1}{(3.367,1.406)}
\gppoint{gp mark 1}{(5.104,1.553)}
\gppoint{gp mark 1}{(2.183,1.304)}
\gppoint{gp mark 1}{(3.683,1.445)}
\gppoint{gp mark 1}{(5.736,1.625)}
\gppoint{gp mark 1}{(2.183,1.327)}
\gppoint{gp mark 1}{(2.973,1.393)}
\gppoint{gp mark 1}{(4.157,1.479)}
\gppoint{gp mark 1}{(4.552,1.509)}
\gppoint{gp mark 1}{(2.973,1.369)}
\gppoint{gp mark 1}{(3.210,1.383)}
\gppoint{gp mark 1}{(6.131,1.620)}
\gppoint{gp mark 1}{(1.788,1.307)}
\gppoint{gp mark 1}{(4.157,1.469)}
\gppoint{gp mark 1}{(1.710,1.298)}
\gppoint{gp mark 1}{(4.394,1.511)}
\gppoint{gp mark 1}{(5.183,1.586)}
\gppoint{gp mark 1}{(4.315,1.515)}
\gppoint{gp mark 1}{(2.894,1.388)}
\gppoint{gp mark 1}{(3.210,1.423)}
\gppoint{gp mark 1}{(3.683,1.456)}
\gppoint{gp mark 1}{(2.973,1.427)}
\gppoint{gp mark 1}{(4.315,1.472)}
\gppoint{gp mark 1}{(4.236,1.481)}
\gppoint{gp mark 1}{(2.578,1.352)}
\gppoint{gp mark 1}{(5.578,1.589)}
\gppoint{gp mark 1}{(3.683,1.451)}
\gppoint{gp mark 1}{(3.052,1.385)}
\gppoint{gp mark 1}{(1.867,1.284)}
\gppoint{gp mark 1}{(2.341,1.316)}
\gppoint{gp mark 1}{(4.078,1.461)}
\gppoint{gp mark 1}{(2.262,1.321)}
\gppoint{gp mark 1}{(5.420,1.591)}
\gppoint{gp mark 1}{(5.341,1.608)}
\gppoint{gp mark 1}{(3.131,1.411)}
\gppoint{gp mark 1}{(5.183,1.594)}
\gppoint{gp mark 1}{(5.025,1.563)}
\gppoint{gp mark 1}{(3.683,1.436)}
\gppoint{gp mark 1}{(7.552,1.630)}
\gppoint{gp mark 1}{(2.420,1.326)}
\gppoint{gp mark 1}{(4.868,1.544)}
\gppoint{gp mark 1}{(2.657,1.341)}
\gppoint{gp mark 1}{(5.262,1.533)}
\gppoint{gp mark 1}{(5.499,1.574)}
\gppoint{gp mark 1}{(2.104,1.316)}
\gppoint{gp mark 1}{(3.999,1.447)}
\gppoint{gp mark 1}{(2.657,1.336)}
\gppoint{gp mark 1}{(2.341,1.324)}
\gppoint{gp mark 1}{(3.525,1.409)}
\gppoint{gp mark 1}{(3.289,1.394)}
\gppoint{gp mark 1}{(3.210,1.392)}
\gppoint{gp mark 1}{(2.736,1.351)}
\gppoint{gp mark 1}{(3.289,1.416)}
\gppoint{gp mark 1}{(3.367,1.427)}
\gppoint{gp mark 1}{(2.341,1.375)}
\gppoint{gp mark 1}{(2.973,1.373)}
\gppoint{gp mark 1}{(2.104,1.318)}
\gppoint{gp mark 1}{(1.788,1.298)}
\gppoint{gp mark 1}{(4.868,1.527)}
\gppoint{gp mark 1}{(3.367,1.422)}
\gppoint{gp mark 1}{(1.710,1.303)}
\gppoint{gp mark 1}{(3.367,1.398)}
\gppoint{gp mark 1}{(2.341,1.316)}
\gppoint{gp mark 1}{(1.710,1.273)}
\gppoint{gp mark 1}{(3.210,1.387)}
\gppoint{gp mark 1}{(4.631,1.514)}
\gppoint{gp mark 1}{(3.210,1.406)}
\gppoint{gp mark 1}{(3.210,1.400)}
\gppoint{gp mark 1}{(1.867,1.286)}
\gppoint{gp mark 1}{(3.446,1.415)}
\gppoint{gp mark 1}{(3.920,1.447)}
\gppoint{gp mark 1}{(3.289,1.392)}
\gppoint{gp mark 1}{(5.420,1.613)}
\gppoint{gp mark 1}{(5.341,1.607)}
\gppoint{gp mark 1}{(4.710,1.527)}
\gppoint{gp mark 1}{(5.104,1.571)}
\gppoint{gp mark 1}{(6.052,1.655)}
\gppoint{gp mark 1}{(4.394,1.499)}
\gppoint{gp mark 1}{(3.367,1.424)}
\gppoint{gp mark 1}{(5.736,1.616)}
\gppoint{gp mark 1}{(4.789,1.538)}
\gppoint{gp mark 1}{(3.604,1.432)}
\gppoint{gp mark 1}{(6.210,1.624)}
\gppoint{gp mark 1}{(4.868,1.536)}
\gppoint{gp mark 1}{(5.420,1.602)}
\gppoint{gp mark 1}{(4.946,1.549)}
\gppoint{gp mark 1}{(3.762,1.448)}
\gppoint{gp mark 1}{(3.446,1.412)}
\gppoint{gp mark 1}{(2.736,1.364)}
\gppoint{gp mark 1}{(3.446,1.397)}
\gppoint{gp mark 1}{(3.841,1.462)}
\gppoint{gp mark 1}{(3.841,1.464)}
\gppoint{gp mark 1}{(4.868,1.545)}
\gppoint{gp mark 1}{(7.157,1.644)}
\gppoint{gp mark 1}{(5.815,1.621)}
\gppoint{gp mark 1}{(4.710,1.522)}
\gppoint{gp mark 1}{(6.762,1.631)}
\gppoint{gp mark 1}{(6.525,1.656)}
\gppoint{gp mark 1}{(3.052,1.399)}
\gppoint{gp mark 1}{(3.920,1.467)}
\gppoint{gp mark 1}{(2.973,1.392)}
\gppoint{gp mark 1}{(2.499,1.353)}
\gppoint{gp mark 1}{(2.262,1.328)}
\gppoint{gp mark 1}{(1.946,1.313)}
\gppoint{gp mark 1}{(2.262,1.330)}
\gppoint{gp mark 1}{(3.289,1.400)}
\gppoint{gp mark 1}{(1.631,1.286)}
\gppoint{gp mark 1}{(3.999,1.466)}
\gppoint{gp mark 1}{(1.473,1.256)}
\gppoint{gp mark 1}{(5.104,1.554)}
\gppoint{gp mark 1}{(2.736,1.376)}
\gppoint{gp mark 1}{(2.894,1.383)}
\gppoint{gp mark 1}{(2.262,1.330)}
\gppoint{gp mark 1}{(1.946,1.310)}
\gppoint{gp mark 1}{(2.341,1.329)}
\gppoint{gp mark 1}{(1.552,1.276)}
\gppoint{gp mark 1}{(1.946,1.301)}
\gppoint{gp mark 1}{(3.367,1.416)}
\gppoint{gp mark 1}{(3.210,1.426)}
\gppoint{gp mark 1}{(4.552,1.527)}
\gppoint{gp mark 1}{(1.867,1.312)}
\gppoint{gp mark 1}{(1.315,1.236)}
\gppoint{gp mark 1}{(3.525,1.447)}
\gppoint{gp mark 1}{(2.894,1.371)}
\gppoint{gp mark 1}{(3.052,1.403)}
\gppoint{gp mark 1}{(2.262,1.329)}
\gppoint{gp mark 1}{(3.683,1.432)}
\gppoint{gp mark 1}{(3.920,1.461)}
\gppoint{gp mark 1}{(2.183,1.325)}
\gppoint{gp mark 1}{(3.920,1.478)}
\gppoint{gp mark 1}{(2.183,1.324)}
\gppoint{gp mark 1}{(2.341,1.337)}
\gppoint{gp mark 1}{(3.762,1.454)}
\gppoint{gp mark 1}{(3.525,1.440)}
\gppoint{gp mark 1}{(2.894,1.366)}
\gppoint{gp mark 1}{(2.657,1.332)}
\gppoint{gp mark 1}{(2.262,1.301)}
\gppoint{gp mark 1}{(1.946,1.280)}
\gppoint{gp mark 1}{(1.631,1.255)}
\gppoint{gp mark 1}{(3.210,1.372)}
\gppoint{gp mark 1}{(3.289,1.412)}
\gppoint{gp mark 1}{(2.894,1.398)}
\gppoint{gp mark 1}{(2.183,1.342)}
\gppoint{gp mark 1}{(3.841,1.457)}
\gppoint{gp mark 1}{(1.710,1.316)}
\gppoint{gp mark 1}{(3.683,1.450)}
\gppoint{gp mark 1}{(2.420,1.332)}
\gppoint{gp mark 1}{(2.183,1.343)}
\gppoint{gp mark 1}{(1.867,1.272)}
\gppoint{gp mark 1}{(2.341,1.317)}
\gppoint{gp mark 1}{(6.762,1.657)}
\gppoint{gp mark 1}{(1.788,1.289)}
\gppoint{gp mark 1}{(4.868,1.547)}
\gppoint{gp mark 1}{(3.841,1.488)}
\gppoint{gp mark 1}{(3.289,1.419)}
\gppoint{gp mark 1}{(3.604,1.432)}
\gppoint{gp mark 1}{(3.210,1.407)}
\gppoint{gp mark 1}{(2.341,1.319)}
\gppoint{gp mark 1}{(7.078,1.660)}
\gppoint{gp mark 1}{(3.289,1.438)}
\gppoint{gp mark 1}{(3.920,1.445)}
\gppoint{gp mark 1}{(2.262,1.311)}
\gppoint{gp mark 1}{(1.631,1.254)}
\gppoint{gp mark 1}{(2.025,1.304)}
\gppoint{gp mark 1}{(3.604,1.438)}
\gppoint{gp mark 1}{(3.762,1.462)}
\gppoint{gp mark 1}{(1.394,1.258)}
\gppoint{gp mark 1}{(1.788,1.285)}
\gppoint{gp mark 1}{(5.025,1.554)}
\gppoint{gp mark 1}{(6.210,1.857)}
\gppoint{gp mark 1}{(4.236,1.507)}
\gppoint{gp mark 1}{(3.289,1.409)}
\gppoint{gp mark 1}{(1.867,1.304)}
\gppoint{gp mark 1}{(1.710,1.278)}
\gppoint{gp mark 1}{(3.525,1.418)}
\gppoint{gp mark 1}{(3.210,1.391)}
\gppoint{gp mark 1}{(2.657,1.332)}
\gppoint{gp mark 1}{(3.841,1.438)}
\gppoint{gp mark 1}{(3.052,1.375)}
\gppoint{gp mark 1}{(2.420,1.329)}
\gppoint{gp mark 1}{(3.920,1.446)}
\gppoint{gp mark 1}{(4.315,1.483)}
\gppoint{gp mark 1}{(3.525,1.445)}
\gppoint{gp mark 1}{(3.920,1.485)}
\gppoint{gp mark 1}{(3.367,1.434)}
\gppoint{gp mark 1}{(5.025,1.542)}
\gppoint{gp mark 1}{(2.499,1.348)}
\gppoint{gp mark 1}{(2.025,1.312)}
\gppoint{gp mark 1}{(2.578,1.356)}
\gppoint{gp mark 1}{(2.894,1.384)}
\gppoint{gp mark 1}{(2.578,1.358)}
\gppoint{gp mark 1}{(2.815,1.357)}
\gppoint{gp mark 1}{(1.788,1.281)}
\gppoint{gp mark 1}{(1.710,1.274)}
\gppoint{gp mark 1}{(4.315,1.483)}
\gppoint{gp mark 1}{(3.525,1.440)}
\gppoint{gp mark 1}{(2.183,1.312)}
\gppoint{gp mark 1}{(2.894,1.373)}
\gppoint{gp mark 1}{(4.157,1.493)}
\gppoint{gp mark 1}{(3.762,1.443)}
\gppoint{gp mark 1}{(3.841,1.452)}
\gppoint{gp mark 1}{(5.499,1.583)}
\gppoint{gp mark 1}{(2.499,1.351)}
\gppoint{gp mark 1}{(1.631,1.262)}
\gppoint{gp mark 1}{(2.973,1.366)}
\gppoint{gp mark 1}{(2.578,1.339)}
\gppoint{gp mark 1}{(2.262,1.331)}
\gppoint{gp mark 1}{(3.999,1.464)}
\gppoint{gp mark 1}{(2.973,1.369)}
\gppoint{gp mark 1}{(3.052,1.397)}
\gppoint{gp mark 1}{(4.868,1.542)}
\gppoint{gp mark 1}{(1.631,1.287)}
\gppoint{gp mark 1}{(2.341,1.321)}
\gppoint{gp mark 1}{(4.394,1.488)}
\gppoint{gp mark 1}{(2.183,1.309)}
\gppoint{gp mark 1}{(2.657,1.340)}
\gppoint{gp mark 1}{(2.973,1.361)}
\gppoint{gp mark 1}{(2.736,1.347)}
\gppoint{gp mark 1}{(3.052,1.376)}
\gppoint{gp mark 1}{(3.446,1.404)}
\gppoint{gp mark 1}{(3.367,1.396)}
\gppoint{gp mark 1}{(3.446,1.398)}
\gppoint{gp mark 1}{(1.552,1.245)}
\gppoint{gp mark 1}{(2.183,1.305)}
\gppoint{gp mark 1}{(3.762,1.427)}
\gppoint{gp mark 1}{(2.499,1.329)}
\gppoint{gp mark 1}{(4.868,1.510)}
\gppoint{gp mark 1}{(1.867,1.280)}
\gppoint{gp mark 1}{(3.525,1.403)}
\gppoint{gp mark 1}{(3.525,1.406)}
\gppoint{gp mark 1}{(1.315,1.231)}
\gppoint{gp mark 1}{(3.525,1.416)}
\gppoint{gp mark 1}{(2.973,1.374)}
\gppoint{gp mark 1}{(3.131,1.419)}
\gppoint{gp mark 1}{(1.394,1.257)}
\gppoint{gp mark 1}{(1.631,1.263)}
\gppoint{gp mark 1}{(2.973,1.373)}
\gppoint{gp mark 1}{(1.315,1.244)}
\gppoint{gp mark 1}{(2.341,1.327)}
\gppoint{gp mark 1}{(3.052,1.396)}
\gppoint{gp mark 1}{(2.578,1.348)}
\gppoint{gp mark 1}{(1.631,1.252)}
\gppoint{gp mark 1}{(2.894,1.354)}
\gppoint{gp mark 1}{(3.920,1.435)}
\gppoint{gp mark 1}{(2.736,1.341)}
\gppoint{gp mark 1}{(3.210,1.386)}
\gppoint{gp mark 1}{(2.578,1.363)}
\gppoint{gp mark 1}{(1.157,1.245)}
\gppoint{gp mark 1}{(2.341,1.331)}
\gppoint{gp mark 1}{(5.183,1.575)}
\gppoint{gp mark 1}{(1.788,1.300)}
\gppoint{gp mark 1}{(2.420,1.337)}
\gppoint{gp mark 1}{(2.973,1.384)}
\gppoint{gp mark 1}{(2.815,1.355)}
\gppoint{gp mark 1}{(2.420,1.324)}
\gppoint{gp mark 1}{(2.420,1.326)}
\gppoint{gp mark 1}{(1.946,1.285)}
\gppoint{gp mark 1}{(2.657,1.344)}
\gppoint{gp mark 1}{(2.973,1.365)}
\gppoint{gp mark 1}{(1.552,1.248)}
\gppoint{gp mark 1}{(2.025,1.284)}
\gppoint{gp mark 1}{(2.578,1.334)}
\gppoint{gp mark 1}{(3.289,1.391)}
\gppoint{gp mark 1}{(4.789,1.514)}
\gppoint{gp mark 1}{(2.104,1.300)}
\gppoint{gp mark 1}{(2.578,1.336)}
\gppoint{gp mark 1}{(3.604,1.420)}
\gppoint{gp mark 1}{(3.052,1.361)}
\gppoint{gp mark 1}{(3.446,1.407)}
\gppoint{gp mark 1}{(3.210,1.394)}
\gppoint{gp mark 1}{(5.420,1.599)}
\gppoint{gp mark 1}{(3.367,1.422)}
\gppoint{gp mark 1}{(2.341,1.334)}
\gppoint{gp mark 1}{(4.157,1.503)}
\gppoint{gp mark 1}{(2.499,1.357)}
\gppoint{gp mark 1}{(3.683,1.458)}
\gppoint{gp mark 1}{(3.052,1.432)}
\gppoint{gp mark 1}{(2.657,1.339)}
\gppoint{gp mark 1}{(2.815,1.363)}
\gppoint{gp mark 1}{(4.552,1.506)}
\gppoint{gp mark 1}{(3.762,1.423)}
\gppoint{gp mark 1}{(2.578,1.340)}
\gppoint{gp mark 1}{(2.025,1.300)}
\gppoint{gp mark 1}{(2.420,1.330)}
\gppoint{gp mark 1}{(1.473,1.251)}
\gppoint{gp mark 1}{(2.420,1.333)}
\gppoint{gp mark 1}{(3.210,1.395)}
\gppoint{gp mark 1}{(4.315,1.478)}
\gppoint{gp mark 1}{(3.052,1.380)}
\gppoint{gp mark 1}{(3.920,1.452)}
\gppoint{gp mark 1}{(2.973,1.366)}
\gppoint{gp mark 1}{(4.078,1.452)}
\gppoint{gp mark 1}{(3.052,1.383)}
\gppoint{gp mark 1}{(4.315,1.472)}
\gppoint{gp mark 1}{(1.867,1.274)}
\gppoint{gp mark 1}{(3.762,1.434)}
\gppoint{gp mark 1}{(4.868,1.521)}
\gppoint{gp mark 1}{(5.183,1.562)}
\gppoint{gp mark 1}{(2.657,1.336)}
\gppoint{gp mark 1}{(3.446,1.402)}
\gppoint{gp mark 1}{(2.262,1.313)}
\gppoint{gp mark 1}{(3.999,1.455)}
\gppoint{gp mark 1}{(4.315,1.479)}
\gppoint{gp mark 1}{(1.788,1.278)}
\gppoint{gp mark 1}{(1.867,1.288)}
\gppoint{gp mark 1}{(3.367,1.419)}
\gppoint{gp mark 1}{(2.499,1.333)}
\gppoint{gp mark 1}{(3.131,1.391)}
\gppoint{gp mark 1}{(2.657,1.349)}
\gppoint{gp mark 1}{(2.973,1.366)}
\gppoint{gp mark 1}{(2.341,1.316)}
\gppoint{gp mark 1}{(2.578,1.343)}
\gppoint{gp mark 1}{(4.157,1.477)}
\gppoint{gp mark 1}{(4.315,1.508)}
\gppoint{gp mark 1}{(2.499,1.392)}
\gppoint{gp mark 1}{(2.104,1.304)}
\gppoint{gp mark 1}{(1.394,1.250)}
\gppoint{gp mark 1}{(4.394,1.495)}
\gppoint{gp mark 1}{(4.631,1.510)}
\gppoint{gp mark 1}{(6.368,1.630)}
\gppoint{gp mark 1}{(5.183,1.539)}
\gppoint{gp mark 1}{(3.999,1.451)}
\gppoint{gp mark 1}{(2.736,1.332)}
\gppoint{gp mark 1}{(5.104,1.557)}
\gppoint{gp mark 1}{(3.604,1.438)}
\gppoint{gp mark 1}{(5.499,1.589)}
\gppoint{gp mark 1}{(4.946,1.543)}
\gppoint{gp mark 1}{(3.052,1.379)}
\gppoint{gp mark 1}{(6.920,1.614)}
\gppoint{gp mark 1}{(3.367,1.405)}
\gppoint{gp mark 1}{(2.499,1.334)}
\gppoint{gp mark 1}{(2.183,1.350)}
\gppoint{gp mark 1}{(3.131,1.399)}
\gppoint{gp mark 1}{(2.499,1.351)}
\gppoint{gp mark 1}{(3.289,1.430)}
\gppoint{gp mark 1}{(4.236,1.505)}
\gppoint{gp mark 1}{(2.657,1.363)}
\gppoint{gp mark 1}{(3.446,1.432)}
\gppoint{gp mark 1}{(3.683,1.438)}
\gppoint{gp mark 1}{(3.999,1.475)}
\gppoint{gp mark 1}{(1.946,1.304)}
\gppoint{gp mark 1}{(2.420,1.335)}
\gppoint{gp mark 1}{(4.078,1.488)}
\gppoint{gp mark 1}{(3.841,1.468)}
\gppoint{gp mark 1}{(3.210,1.418)}
\gppoint{gp mark 1}{(8.357,3.761)}
\gpcolor{color=gp lt color border}
\gpsetlinewidth{1.00}
\draw[gp path] (0.920,6.999)--(0.920,0.986)--(8.815,0.986)--(8.815,6.999)--cycle;
%% coordinates of the plot area
\gpdefrectangularnode{gp plot 1}{\pgfpoint{0.920cm}{0.986cm}}{\pgfpoint{8.815cm}{6.999cm}}
\end{tikzpicture}
%% gnuplot variables


  \caption{The relationship between the latency and the number of source tokens
    using an \acl{ar} and a \acl{nar} model}%
  \label{fig:end-to-end:speed}
\end{figure}


% %%%%%%%%%%%%%%%%%%%%%%%%%%%%%%%%%%%%%%%%%%%%%%%%%%%%%%%%%%%%%%%%%%%%%%%%%%%%%
\chapter{Experiments}
\label{chap:experiments}
% %%%%%%%%%%%%%%%%%%%%%%%%%%%%%%%%%%%%%%%%%%%%%%%%%%%%%%%%%%%%%%%%%%%%%%%%%%%%%

The flaws in the evaluation methodology adopted by current research of \ac{nat}
models make it difficult to identify approaches suitable in production-ready
conditions.  In this chapter, we present further experiments with \ac{nat}
models trained with \ac{ctc}. We focus on a fair comparison with other
non-autoregressive approaches as well as state-of-the-art optimized
autoregressive methods.  In line with our previous experiments and also with
the rest of the related work, we conduct experiments on English-German
translation.

% ------------------------------------------------------------------------------
\section{Autoregressive Teacher Models}%
\label{sec:exp:teachers}
% ------------------------------------------------------------------------------

\paperdisclaim{This section is based on the paper \emph{``The University of
    Edinburgh's English-German and English-Hausa Submissions to the WMT21 News
    Translation Task''}, joint work with Pinzen Chen, Ulrich Germann, Laurie
  Burchell, Nikolay Bogoychev, Antonio Valerio Miceli Barone, Jonas Waldendorf,
  Alexandra Birch, and Kenneth Heafield, published at WMT 2021.}

\noindent
In our English-German experiments, we use autoregressive models from our
submission to the \ac{wmt} 2021 News Translation Shared Task
\citep{chen-etal-2021-university}. We use these models both as strong \ac{ar}
baselines and as teacher models for generating distilled data for the \ac{nar}
student models, described in Section \ref{sec:exp:students}. The models were
trained on cleaned parallel data augmented with backtranslated monolingual
data, using Marian, a C++ toolkit for training and decoding from \ac{nmt}
models \citep{junczys-dowmunt-etal-2018-marian}.


% ------------------------------------------------------------------------------
\paragraph{Data Cleaning.} We prepare the training dataset consisting of the
following parts: First, we use clean parallel data from the Europarl corpus
\citep{koehn-2005-europarl}, the Tilde MODEL -- RAPID corpus
\citep{rozis-skadins-2017-tilde}, and the News Commentary corpus from OPUS
\citep{tiedemann-2012-parallel}. Next, we include sources of crawled parallel
data from the web, which are considered noisy. These include Paracrawl
\citep{espla-etal-2019-paracrawl}, Common
Crawl\footnote{\url{https://commoncrawl.org/}}, WikiMatrix
\citep{schwenk2019wikimatrix}, and the Wikipedia Parallel Titles
Corpus\footnote{\url{https://linguatools.org/tools/corpora/wikipedia-parallel-titles-corpora/}}. Finally,
we use backtranslation \citep{sennrich-etal-2016-improving} of monolingual data
obtained from News Crawl.  We train our own models to generate the
backtranslations on cleaned parallel data, as described below.

We perform the following filtering techniques on the gathered data (both clean
and noisy) to improve the overall quality of the parallel data.

We start with deterministic rule-based filtering%
\footnote{\url{https://github.com/browsermt/students/blob/master/train-student/clean/}}
and deduplication. We remove all sentence pairs containing non-printing
characters, empty sentences, and sentences longer than 120 words; we also
remove all sentence pairs with length ratio of less than 0.6 (0.4 for
Wikititles), sentences in which over 40\% of characters do not constitute
tokens, and sentences in which more than 50\% were non-alphabetic characters. We
run language identification using fastText
\citep{joulin-etal-2017-bag,joulin2016fasttext} and remove all sentence pairs
classified as not English-German.

To further clean the data, we apply dual cross-entropy filtering, as proposed
by \citet{junczys-dowmunt-2018-dual}. We train two Transformer base models (one
for each translation direction) for dual cross-entropy filtering using the
clean part of the data after the rule-based cleaning step.

We score the crawled part of the parallel data using the trained models and we
sort the sentence pairs according to the score. To estimate how many of the
crawled sentence pairs can we consider clean, we train the big variant of the
Transformer translation models in both directions on different amounts of the
scored data (following the teacher model hyperparameter settings in Table
\ref{tab:trafo-big-hparams}). We use 25\%, 50\%, 75\% and 100\% of the data,
taking the highest-scoring sentence pairs for training. Based on the \acs{bleu}
scores achieved by the Transformer big models on the development data (the
\ac{wmt} 2019 test set, \citealp{barrault-etal-2019-findings}), we declare the
75\% as the ``clean'' portion of the crawled data.


% ------------------------------------------------------------------------------
\paragraph{Backtranslation.} We train three additional translation models (four
in total with the one from the data cleaning step) on the filtered parallel
dataset to create backtranslations \citep{sennrich-etal-2016-improving}. We use
the same hyperparameter settings as for the teacher models (see Table
\ref{tab:trafo-big-hparams}), except for different random seeds for parameter
initialization.

We translate the monolingual data using the four models in an ensemble. We
limit the target sentence length to 150 tokens and use beam search decoding
with 6 hypotheses in the beam, with the length normalization parameter set to
1.

As the source of the monolingual data, we use the News Crawl datasets from
years 2018, 2019, and 2020, as released by the \acs{wmt} organizers
\citep{bojar-etal-2018-findings,barrault-etal-2019-findings,
  barrault-etal-2020-findings}. In total, we gathered 91 million English
sentences for backtranslation into German and 146 million German sentences for
backtranslation into English.

We follow the approach of \citet{caswell-etal-2019-tagged} and we tag the
backtranslated sentences with a special token on a first position.


% ------------------------------------------------------------------------------
\paragraph{Teacher Model Training.}  We train the teacher models on shuffled
concatenation of the authentic parallel and tagged backtranslated data. As with
the backtranslation models, we train four models with different seeds for
random initialization in each direction. We show the hyperparameter values in
Table \ref{tab:trafo-big-hparams}.

After the training on mixed parallel and backtranslated data converged, we
continued the training of the models on parallel data only.

\begin{table}
  \centering
  \begin{tabular}{llr}
    \toprule
    Parameter & Marian config variable & \mcl{Value} \\
    \midrule
    No. of encoder layers & \texttt{enc-depth} & 6 \\
    No. of decoder layers & \texttt{dec-depth} & 6 \\
    Model dimension &  \texttt{dim-emb} & 1,024 \\
    Feed-forward state dimension & \texttt{transformer-dim-ffn} & 4,096 \\
    Attention heads & \texttt{transformer-heads} & 16 \\
    % & transformer-postprocess & dan \\
    % & transformer-postprocess-emb & d\\
    Vocabulary size & & 32,000 \\
    \midrule
    Optimizer method & \texttt{optimizer} & adam \\
    $\beta_1$ & \multirow{3}{*}{\texttt{optimizer-params}} & 0.9 \\
    $\beta_2$ & & 0.998 \\
    $\epsilon$ & & 10\textsuperscript{-9} \\
    No. of batches per update & \texttt{optimizer-delay} & 2 \\
    Fit batch to available memory & \texttt{mini-batch-fit} & true \\
    Learning rate & \texttt{learn-rate}  & 10\textsuperscript{-4} \\
    Learning rate warmup & \texttt{lr-warmup} & 8,000 \\
    Learning rate decay & \texttt{lr-decay-inv-sqrt} & 8,000 \\
    Gradient clipping\footnotemark & \texttt{clip-norm} & 0 \\
    \bottomrule
  \end{tabular}

  \caption{The hyperparameters of the teacher models. The same values were
    used for training the models for backtranslation.}%
  \label{tab:trafo-big-hparams}

\end{table}
% ------------------------------------------------------------------------------
\footnotetext{We did not use gradient clipping because of issues in the Marian
  toolkit implementation.}
% ------------------------------------------------------------------------------

% ------------------------------------------------------------------------------
\paragraph{Knowledge Distillation.} The teacher models are used to create
artificial targets for the student models \citep{kim-rush-2016-sequence}. For
each translation direction, we translate the source side of the parallel data
and the authentic monolingual data (in the source language) using the ensemble
of the teacher models. We do not create wholly synthetic datasets by
forward-translating backtranslated data.

% For generating the knowledge-distilled dataset, we use beam search with the
% beam size of 12 and the normalization parameter set to 0.8. Similarly to
% generating the backtranslated data, we limit the target sentence length to 150
% tokens.


% ------------------------------------------------------------------------------
\section{Student Models}%
\label{sec:exp:students}
% ------------------------------------------------------------------------------

In this section, we describe our non-autoregressive student models. We
implemented the \acs{ctc}-based Transformer architecture in the Marian toolkit
\citep{junczys-dowmunt-etal-2018-marian}. For the computation of the \ac{ctc}
loss, we use the warp-ctc
library,\footnote{\url{https://github.com/baidu-research/warp-ctc/}} an
efficient parallelized implementation for GPUs \citep{amodei-etal-2016-deep}.

We experiment with different hyperparameter settings that control the size of
the student model. Our baseline non-autoregressive model is a big Transformer
model with 6 encoder layers, followed by the state-splitting layer, and another
6 decoder layers. Apart from the \ac{ctc}-specific configuration, we use the
same hyperparameters as in the teacher models, shown in Table
\ref{tab:trafo-big-hparams}. In addition, we train four smaller models based on
the Transformer base hyperparameters -- using model dimension of 512,
feed-forward state dimension of 2,048, and 8 attention heads. Each smaller
model uses a different number of encoder and decoder layers. We show the number
of encoder and decoder layers in Table \ref{tab:student-model-hparams}. In all
settings, we use the splitting factor of 3 in the state splitting layer.

\begin{table}
  \centering

  \begin{tabular}{llrr}
    \toprule
    Model & Base architecture & \mcl{Encoder layers} & \mcl{Decoder layers} \\
    \midrule
    Teacher & Transformer big & 6 & 6 \\
    \midrule
    Large & Transformer big & 6 & 6 \\
    \addlinespace
    Base  & \multirow{4}{*}{Transformer base} & 6 & 6 \\
    Small & & 3 & 3\\
    Micro & & 2 & 2 \\
    Tiny  & & 1 & 1 \\
    \bottomrule
  \end{tabular}

  \caption{The hyperparameters of the student models.}%
  \label{tab:student-model-hparams}
\end{table}


% ------------------------------------------------------------------------------
\section{Results}%
\label{sec:exp:results}
% ------------------------------------------------------------------------------

In this section we analyze the results both in terms of translation quality in
Section \ref{subsec:results:quality} and decoding speed in Section
\ref{subsec:results:time}. We discuss the results from the perspectives of the
related literature on the \ac{nat} models and the submissions to the Efficiency
Shared Task in Section \ref{subsec:results:discussion}.


% ------------------------------------------------------------------------------
\subsection{Translation Quality}%
\label{subsec:results:quality}
% ------------------------------------------------------------------------------

We compare the results of our models to both related work on \ac{nat} and to
the results of the \ac{wmt}~21 Efficiency Shared Task which features highly
optimized \ac{ar} translation models \citep{heafield-etal-2021-findings}. In
the \ac{nar} literature, the \acs{wmt}~14 test set
\citep{bojar-etal-2014-findings} is used as a standard benchmark. On the other
hand, the efficiency task uses the recent test set from \acs{wmt}~21
\citep{akhbardeh-etal-2021-findings}. In this section, we present the results
on both datasets.

\paragraph{Results on the \acs{wmt}~21 test set.}
In line with the evaluation methodology of the efficiency task, we also measure
\acs{comet} \citep{rei-etal-2020-comet} and \acs{chrf} score
\citep{popovic-2015-chrf}. We do not perform human evaluation.

Table \ref{tab:wmt21-scores} shows the results of automatic evaluation
% with \acs{bleu}, \acs{chrf}, and \acs{comet}% scores
on the \acs{wmt}~21 news translation test set. In the English $\rightarrow$
German direction, the test set consists of 1,002 sentences along with three
different reference translations. In the German $\rightarrow$ English
direction, there are 1,000 sentences with two reference translations each. We
measure multi-reference \acs{bleu} score using Sacrebleu \citep{post-2018-call}
and we report confidence intervals computed with bootstrap resampling. We
report the Sacrebleu signatures for English $\rightarrow$ German%
\footnote{En$\rightarrow$De: {\scriptsize
    \texttt{nrefs:3|bs:1000|seed:12345|case:mixed|eff:no|tok:13a|smooth:exp|version:2.0.0}}}
and German $\rightarrow$ English%
\footnote{De$\rightarrow$En: {\scriptsize
    \texttt{nrefs:2|bs:1000|seed:12345|case:mixed|eff:no|tok:13a|smooth:exp|version:2.0.0}}}
directions. We compute the \acs{chrf} score separately on each reference
translation set using Sacrebleu, and we report the
average.\footnote{\acs{chrf}; En$\leftrightarrow$De: {\scriptsize
    \texttt{nrefs:1|case:mixed|eff:yes|nc:6|nw:0|space:no|version:2.0.0}}}
Finally, we compute the \acs{comet} scores with the \texttt{wmt20-comet-da}
model of the \acs{comet} version \texttt{dd2298} (1.0.0rc9).

We observe that in both translation directions, the \ac{ar} models outperform
the \ac{nar} models. The performance gap between the models grows further with
beam search and ensembling. We can also see that knowledge distillation has a
positive effect on both \ac{ar} and \ac{nar} models, with the student \ac{ar}
model matching the performance of the ensemble of four large teacher models.
We also note that the difference in the \acs{comet} score is bigger than in
\acs{bleu}, which might suggest that \acs{nar} models would rank poorly in
human evaluation, despite achieving reasonable \acs{bleu} scores.

The results on the test set confirm the ranking of the \acs{nar} models as seen
during training, including the interesting exception of the
English $\rightarrow$ German Micro model. Otherwise, the larger the student
model is, the better scores it achieves.

Finally, we see that using a lexical shortlist does not have an effect on the
translation quality in all model variants. However, we see that shortlisting
impairs the performance of the Transformer base model when decoding with beam
search.


\begin{table}
  \centering

  \begin{tabular}{lrrr@{}>{\small \enspace \textpm}lrrr@{}>{\small \enspace
        \textpm}l}
    \toprule
    \multirow{2}{*}{\bf En $\rightarrow$ De} %
    & \multicolumn{4}{c}{Full output projection} & \multicolumn{4}{c}{Shortlist} \\
    & \acs{chrf} & \acs{comet} & \multicolumn{2}{c}{\acs{bleu}}
    & \acs{chrf} & \acs{comet} & \multicolumn{2}{c}{\acs{bleu}} \\

    \midrule
    Single greedy \acs{ar} \\
    Large & 59.2 & 0.4110 & 50.5 & 1.3 & 59.2 & 0.4124 & 50.6 & 1.3\\
    Base  & 58.1 & 0.3881 & 47.9 & 1.3 & 58.1 & 0.3875 & 47.9 & 1.2\\

    \addlinespace
    Single beam \acs{ar} \\
    Large & 58.8 & 0.4053 & 50.8 & 1.3 & 58.8 & 0.4144 & 47.9 & 1.2\\
    Base  & 57.9 & 0.3873 & 48.0 & 1.3 & 55.1 & 0.2666 & 39.3 & 1.1\\

    \addlinespace
    Ensemble beam \acs{ar} \\
    Large & 59.5 & 0.4332 & 52.2 & 1.3 & 59.4 & 0.4303 & 52.2 & 1.3\\

    \midrule
    Student \acs{ar} \\
    Base  & 59.5 &  0.4550 & 51.6 & 1.2 & 59.6 &  0.4564 & 51.6 & 1.2 \\

    \addlinespace
    \Acs{nar} models \\
    Large & 58.6 &  0.1485 & 47.8 & 1.2 & 58.7 &  0.1442 & 47.7 & 1.2 \\
    Base  & 56.3 & -0.0521 & 41.8 & 1.1 & 56.3 & -0.0545 & 41.8 & 1.1 \\
    Small & 56.2 & -0.0752 & 41.9 & 1.1 & 56.2 & -0.0773 & 41.9 & 1.2 \\
    Micro & 57.3 & -0.0083 & 43.5 & 1.1 & 57.4 & -0.0085 & 43.6 & 1.1 \\
    Tiny  & 53.6 & -0.3333 & 34.7 & 1.0 & 53.8 & -0.3346 & 34.8 & 1.0 \\

    \bottomrule
  \end{tabular}

%   \caption{Quantitative results of the English $\rightarrow$ German translation
%     models on the \acs{wmt}~21 test set using three different automatic
%     evalutation metrics.}%
%   \label{tab:wmt21-scores-ende}
% \end{table}

% \begin{table}
%   \centering

  \vspace{1\baselineskip}

  \begin{tabular}{lrrr@{}>{\small \enspace \textpm}lrrr@{}>{\small \enspace
        \textpm}l}
    \toprule
    \multirow{2}{*}{\bf De $\rightarrow$ En} %
    & \multicolumn{4}{c}{Full output projection} & \multicolumn{4}{c}{Shortlist} \\
    & \acs{chrf} & \acs{comet} & \multicolumn{2}{c}{\acs{bleu}}
                 & \acs{chrf} & \acs{comet} & \multicolumn{2}{c}{\acs{bleu}} \\

    \midrule
    Single greedy \acs{ar} \\
    Large & 61.9 & 0.5868 & 48.4 & 1.3 & 61.9 & 0.5866 & 48.5 & 1.3\\
    Base  & 61.0 & 0.5532 & 47.0 & 1.3 & 60.5 & 0.5534 & 47.1 & 1.3\\

    \addlinespace
    Single beam \acs{ar} \\
    Large & 61.5 & 0.5885 & 49.2 & 1.2 & 61.2 & 0.5659 & 43.9 & 1.1\\
    Base  & 60.7 & 0.5534 & 47.4 & 1.3 & 58.0 & 0.4591 & 38.5 & 1.2\\

    \addlinespace
    Ensemble beam \acs{ar} \\
    Large & 62.0 & 0.5954 & 50.6 & 1.3 & 62.3 & 0.5970 & 50.8 & 1.3\\

    \midrule
    Student \acs{ar} \\
    Base & 63.3 & 0.6115 & 51.1 & 1.3 & 63.3 & 0.6112 & 51.1 & 1.3\\

    \addlinespace
    \Acs{nar} models \\
    Large & 61.6 &  0.3296 & 46.4 & 1.4 & 61.6 &  0.3288 & 46.4 & 1.4\\
    Base  & 61.4 &  0.2957 & 45.8 & 1.3 & 61.4 &  0.2663 & 45.7 & 1.3\\
    Small & 61.0 &  0.2462 & 44.6 & 1.3 & 61.0 &  0.2454 & 44.6 & 1.3\\
    Micro & 59.6 &  0.1475 & 42.3 & 1.4 & 59.6 &  0.1468 & 42.3 & 1.4\\
    Tiny  & 55.9 & -0.1558 & 34.4 & 1.3 & 55.9 & -0.1560 & 34.4 & 1.3\\

    \bottomrule
  \end{tabular}

  \caption{Quantitative results of the German $\leftrightarrow$ English
    translation models on the \acs{wmt}~21 test set using \acs{chrf},
    \acs{comet}, and \acs{bleu}.}%
  \label{tab:wmt21-scores}
\end{table}


\paragraph{Results on the \acs{wmt}~14 test set.}
We present automatic evaluation results measured on the \acs{wmt}~14 test set.
Since many of the related approaches stop the training early after 300 thousand
updates \citep{gu2017nonautoregressive, gu-kong-2021-fully}, we report the
scores of our models both at this point in training, and after the training
reached convergence.

Table \ref{tab:wmt14-weight-avg} shows the \acs{wmt}~14 \acs{bleu} scores with
checkpoint averaging. In each variant, we take the average parameters of the
five best scoring models as measured on the validation set (either before the
300,000th update or overall).

\begin{table}
  \centering

  \begin{tabular}{lrrrr}
    \toprule
    & \multicolumn{2}{c}{En $\rightarrow$ De}
    & \multicolumn{2}{c}{De $\rightarrow$ En} \\
    Model
    & 300k & Final & 300k &  Final \\
    \midrule

    Large & 27.7 & 28.4 & 30.0 & 31.3 \\
    Base  & 22.4 & 23.7 & 28.1 & 30.3 \\
    Small & 22.5 & 23.6 & 26.7 & 29.1 \\
    Micro & 23.7 & 25.0 & 25.1 & 27.5 \\
    Tiny  & 19.0 & 20.3 & 19.6 & 21.7 \\

    \bottomrule
  \end{tabular}

  \caption{The \acs{bleu} scores of the \emph{averaged} models on the
    \acs{wmt}~14 test set after 300k updates and at the end of the training.}%
  \label{tab:wmt14-weight-avg}

\end{table}



% ------------------------------------------------------------------------------
\subsection{Decoding Time}%
\label{subsec:results:time}
% ------------------------------------------------------------------------------

In this section, we analyze the decoding speed of the English $\rightarrow$
German \ac{nar} models. We aim at recreating the evaluation conditions
following the \acs{wmt}~21 Efficiency Shared Task
\citep{heafield-etal-2021-findings}. We measure the decoding latency and
throughput in different hardware environments.

The decoding time is measured on a dataset containing one million sentences to
minimize the effect of the model loading overhead.

For measuring the CPU times, we use the same environment as the \acs{wmt}~21
shared task organizers, which is a dual-socket Intel Xeon Gold 6354 from Oracle
Cloud, a 36-core CPU server. For GPU efficiency, we use an Nvidia A100 GPU.


\paragraph{GPU Latency and Throughput.}
Figure \ref{fig:throughput:a100} shows the relationship between the batch size
and the decoding time in seconds on the A100 GPU.  For \ac{ar} models,
increasing the batch size heavily reduces the overall decoding speed and
eventually surpasses the large \ac{nar} models. Increasing the batch size
speeds up the decoding in \ac{nar} models as well, but the effect is diminished
for larger batch sizes.

From this point of view, the optimal scenario for \ac{nar} models is a
situation in which the system runs in online mode, i.e. with a batch size of 1,
or with a batch size of a few sentences. On a faster GPU with more threads,
\ac{ar} models need a larger batch size to meet the speed of \ac{nar} models.


\begin{figure}
  \centering
  \begin{tikzpicture}[gnuplot]
%% generated with GNUPLOT 5.2p8 (Lua 5.3; terminal rev. Nov 2018, script rev. 108)
%% Mon 08 Nov 2021 06:15:48 PM GMT
\path (0.000,0.000) rectangle (13.500,8.000);
\gpcolor{color=gp lt color border}
\gpsetlinetype{gp lt border}
\gpsetdashtype{gp dt solid}
\gpsetlinewidth{1.00}
\draw[gp path] (1.872,0.985)--(2.052,0.985);
\draw[gp path] (12.947,0.985)--(12.767,0.985);
\node[gp node right] at (1.688,0.985) {0};
\draw[gp path] (1.872,2.103)--(2.052,2.103);
\draw[gp path] (12.947,2.103)--(12.767,2.103);
\node[gp node right] at (1.688,2.103) {5,000};
\draw[gp path] (1.872,3.220)--(2.052,3.220);
\draw[gp path] (12.947,3.220)--(12.767,3.220);
\node[gp node right] at (1.688,3.220) {10,000};
\draw[gp path] (1.872,4.338)--(2.052,4.338);
\draw[gp path] (12.947,4.338)--(12.767,4.338);
\node[gp node right] at (1.688,4.338) {15,000};
\draw[gp path] (1.872,5.456)--(2.052,5.456);
\draw[gp path] (12.947,5.456)--(12.767,5.456);
\node[gp node right] at (1.688,5.456) {20,000};
\draw[gp path] (1.872,6.573)--(2.052,6.573);
\draw[gp path] (12.947,6.573)--(12.767,6.573);
\node[gp node right] at (1.688,6.573) {25,000};
\draw[gp path] (1.872,7.691)--(2.052,7.691);
\draw[gp path] (12.947,7.691)--(12.767,7.691);
\node[gp node right] at (1.688,7.691) {30,000};
\draw[gp path] (1.872,0.985)--(1.872,1.165);
\draw[gp path] (1.872,7.691)--(1.872,7.511);
\node[gp node center] at (1.872,0.677) {1};
\draw[gp path] (3.454,0.985)--(3.454,1.165);
\draw[gp path] (3.454,7.691)--(3.454,7.511);
\node[gp node center] at (3.454,0.677) {2};
\draw[gp path] (5.036,0.985)--(5.036,1.165);
\draw[gp path] (5.036,7.691)--(5.036,7.511);
\node[gp node center] at (5.036,0.677) {4};
\draw[gp path] (6.618,0.985)--(6.618,1.165);
\draw[gp path] (6.618,7.691)--(6.618,7.511);
\node[gp node center] at (6.618,0.677) {8};
\draw[gp path] (8.201,0.985)--(8.201,1.165);
\draw[gp path] (8.201,7.691)--(8.201,7.511);
\node[gp node center] at (8.201,0.677) {16};
\draw[gp path] (9.783,0.985)--(9.783,1.165);
\draw[gp path] (9.783,7.691)--(9.783,7.511);
\node[gp node center] at (9.783,0.677) {32};
\draw[gp path] (11.365,0.985)--(11.365,1.165);
\draw[gp path] (11.365,7.691)--(11.365,7.511);
\node[gp node center] at (11.365,0.677) {64};
\draw[gp path] (12.947,0.985)--(12.947,1.165);
\draw[gp path] (12.947,7.691)--(12.947,7.511);
\node[gp node center] at (12.947,0.677) {128};
\draw[gp path] (1.872,7.691)--(1.872,0.985)--(12.947,0.985)--(12.947,7.691)--cycle;
\node[gp node center,rotate=-270] at (0.016,4.338) {Decoding time (seconds)};
\node[gp node center] at (7.409,0.215) {Batch size};
\node[gp node right] at (12.031,7.280) {Ar Large};
\gpcolor{rgb color={0.200,0.200,0.200}}
\gpsetlinewidth{2.50}
\draw[gp path] (12.215,7.280)--(12.579,7.280);
\draw[gp path] (11.365,1.676)--(12.947,1.414);
\gpsetpointsize{4.00}
\gppoint{gp mark 1}{(11.365,1.676)}
\gppoint{gp mark 1}{(12.947,1.414)}
\gppoint{gp mark 1}{(12.397,7.280)}
\gpcolor{color=gp lt color border}
\node[gp node right] at (12.031,6.818) {Ar Base (stdnt)};
\gpcolor{rgb color={0.800,0.800,0.800}}
\draw[gp path] (12.215,6.818)--(12.579,6.818);
\draw[gp path] (6.618,4.078)--(8.201,2.661)--(9.783,1.878)--(11.365,1.515)--(12.947,1.312);
\gppoint{gp mark 2}{(6.618,4.078)}
\gppoint{gp mark 2}{(8.201,2.661)}
\gppoint{gp mark 2}{(9.783,1.878)}
\gppoint{gp mark 2}{(11.365,1.515)}
\gppoint{gp mark 2}{(12.947,1.312)}
\gppoint{gp mark 2}{(12.397,6.818)}
\gpcolor{color=gp lt color border}
\node[gp node right] at (12.031,6.356) {Large};
\gpcolor{rgb color={0.800,0.000,0.000}}
\draw[gp path] (12.215,6.356)--(12.579,6.356);
\draw[gp path] (1.872,2.554)--(3.454,1.851)--(5.036,1.497)--(6.618,1.331)--(8.201,1.249)%
  --(9.783,1.202)--(11.365,1.175)--(12.947,1.160);
\gppoint{gp mark 3}{(1.872,2.554)}
\gppoint{gp mark 3}{(3.454,1.851)}
\gppoint{gp mark 3}{(5.036,1.497)}
\gppoint{gp mark 3}{(6.618,1.331)}
\gppoint{gp mark 3}{(8.201,1.249)}
\gppoint{gp mark 3}{(9.783,1.202)}
\gppoint{gp mark 3}{(11.365,1.175)}
\gppoint{gp mark 3}{(12.947,1.160)}
\gppoint{gp mark 3}{(12.397,6.356)}
\gpcolor{color=gp lt color border}
\node[gp node right] at (12.031,5.894) {Base};
\gpcolor{rgb color={1.000,0.800,0.000}}
\draw[gp path] (12.215,5.894)--(12.579,5.894);
\draw[gp path] (1.872,2.391)--(3.454,1.745)--(5.036,1.399)--(6.618,1.246)--(8.201,1.167)%
  --(9.783,1.127)--(11.365,1.106)--(12.947,1.093);
\gppoint{gp mark 4}{(1.872,2.391)}
\gppoint{gp mark 4}{(3.454,1.745)}
\gppoint{gp mark 4}{(5.036,1.399)}
\gppoint{gp mark 4}{(6.618,1.246)}
\gppoint{gp mark 4}{(8.201,1.167)}
\gppoint{gp mark 4}{(9.783,1.127)}
\gppoint{gp mark 4}{(11.365,1.106)}
\gppoint{gp mark 4}{(12.947,1.093)}
\gppoint{gp mark 4}{(12.397,5.894)}
\gpcolor{color=gp lt color border}
\node[gp node right] at (12.031,5.432) {Small};
\gpcolor{rgb color={0.800,0.000,1.000}}
\draw[gp path] (12.215,5.432)--(12.579,5.432);
\draw[gp path] (1.872,1.723)--(3.454,1.401)--(5.036,1.220)--(6.618,1.145)--(8.201,1.103)%
  --(9.783,1.082)--(11.365,1.070)--(12.947,1.065);
\gppoint{gp mark 5}{(1.872,1.723)}
\gppoint{gp mark 5}{(3.454,1.401)}
\gppoint{gp mark 5}{(5.036,1.220)}
\gppoint{gp mark 5}{(6.618,1.145)}
\gppoint{gp mark 5}{(8.201,1.103)}
\gppoint{gp mark 5}{(9.783,1.082)}
\gppoint{gp mark 5}{(11.365,1.070)}
\gppoint{gp mark 5}{(12.947,1.065)}
\gppoint{gp mark 5}{(12.397,5.432)}
\gpcolor{color=gp lt color border}
\node[gp node right] at (12.031,4.970) {Micro};
\gpcolor{rgb color={0.000,0.000,0.800}}
\draw[gp path] (12.215,4.970)--(12.579,4.970);
\draw[gp path] (1.872,1.504)--(3.454,1.286)--(5.036,1.171)--(6.618,1.107)--(8.201,1.082)%
  --(9.783,1.067)--(11.365,1.059)--(12.947,1.055);
\gppoint{gp mark 6}{(1.872,1.504)}
\gppoint{gp mark 6}{(3.454,1.286)}
\gppoint{gp mark 6}{(5.036,1.171)}
\gppoint{gp mark 6}{(6.618,1.107)}
\gppoint{gp mark 6}{(8.201,1.082)}
\gppoint{gp mark 6}{(9.783,1.067)}
\gppoint{gp mark 6}{(11.365,1.059)}
\gppoint{gp mark 6}{(12.947,1.055)}
\gppoint{gp mark 6}{(12.397,4.970)}
\gpcolor{color=gp lt color border}
\node[gp node right] at (12.031,4.508) {Tiny};
\gpcolor{rgb color={0.000,0.800,0.000}}
\draw[gp path] (12.215,4.508)--(12.579,4.508);
\draw[gp path] (1.872,1.289)--(3.454,1.159)--(5.036,1.097)--(6.618,1.067)--(8.201,1.052)%
  --(9.783,1.046)--(11.365,1.041)--(12.947,1.039);
\gppoint{gp mark 7}{(1.872,1.289)}
\gppoint{gp mark 7}{(3.454,1.159)}
\gppoint{gp mark 7}{(5.036,1.097)}
\gppoint{gp mark 7}{(6.618,1.067)}
\gppoint{gp mark 7}{(8.201,1.052)}
\gppoint{gp mark 7}{(9.783,1.046)}
\gppoint{gp mark 7}{(11.365,1.041)}
\gppoint{gp mark 7}{(12.947,1.039)}
\gppoint{gp mark 7}{(12.397,4.508)}
\gpcolor{color=gp lt color border}
\gpsetlinewidth{1.00}
\draw[gp path] (1.872,7.691)--(1.872,0.985)--(12.947,0.985)--(12.947,7.691)--cycle;
%% coordinates of the plot area
\gpdefrectangularnode{gp plot 1}{\pgfpoint{1.872cm}{0.985cm}}{\pgfpoint{12.947cm}{7.691cm}}
\end{tikzpicture}
%% gnuplot variables

  \vspace{1\baselineskip}

  \begin{tabular}{lrrrrrrrr}
    \toprule
    Batch size & 1 & 2 & 4 & 8 & 16 & 32 & 64 & 128 \\
    \midrule
    \acs{ar} -- Large && 53,902 & 29,369 & 15,351 & 8,907 & 5,216 & 3,090 & 1,918 \\
    \acs{ar} -- Base  && 47,145 & 25,745 & 13,836 & 7,498 & 3,997 & 2,371 & 1,465 \\
    \midrule
    Large & 7,020 & 3,874 & 2,292 & 1,547 & 1,179 & 973 & 850 & 782 \\
    Base  & 6,289 & 3,400 & 1,854 & 1,166 &   816 & 635 & 542 & 485 \\
    Small & 3,300 & 1,860 & 1,051 &   717 &   526 & 434 & 380 & 357 \\
    Micro & 2,322 & 1,345 &   833 &   544 &   433 & 367 & 332 & 311 \\
    Tiny  & 1,360 &   780 &   503 &   367 &   301 & 273 & 252 & 243 \\
    \bottomrule
  \end{tabular}

  \caption{Wall times to translate one million sentences (in seconds) on a
    single Nvidia \emph{Ampere} A100 GPU with different batch size settings.}%
  \label{fig:throughput:a100}
\end{figure}


% -----------------------------------------------------------------------------
\paragraph{CPU Latency and Throughput.}
In Figure \ref{fig:throughput:cpu36}, we show the CPU decoding times of the
trained models using 36 CPU cores with different batching settings.  We see
similar trends to GPU decoding -- the \ac{nar} models are faster with smaller
batch sizes. The \ac{ar} models eventually match the decoding speed of the
\ac{nar} models as the batch size increases. However, there is a considerable
difference between the large and base models in both \ac{ar} and \ac{nar}
variants.

We also notice that increasing the batch size can slow the decoding speed
down. There may be several reasons for this behavior. First, the size of the
shortlist grows proportionally with the batch size, as there are more possible
target words. Second, when the batch is too large, much of the computation is
wasted on the padded positions in shorter sentences. These issues are not
evident in GPU decoding due to a higher level of parallelism.

It is apparent from the data table in Figure \ref{fig:throughput:cpu36} that
the use of a lexical shortlist improves the decoding speed of both \ac{ar} and
\ac{nar} models. For clarity, we only plot the measured decoding times with
shortlisting. We see that changing the batch size has a similar effect in both
cases, perhaps with the exception of the micro and tiny models, which benefit
greatly from the shortlist in combination with a small batch size.

In case of the \ac{nar} model with batch size of 128, we ran across an
out-of-memory error due to a limitation in the Marian implementation. When
using multiple CPUs for the decoding, the program makes a copy of the whole
model for each CPU core. When the batch size is too large, this will eventually
fill the whole RAM, and the process is killed.

% We tried to measure single-core CPU latency, but the decoding speed was too
% slow to compete with efficiency task submissions. Based on the decoding time on
% a smaller sample of the data, we estimate the time to translate the whole
% dataset would take around 100 hours using the large \ac{nar} model and around
% 15 hours using the tiny model.

\begin{figure}
  \centering

  \begin{tikzpicture}[gnuplot]
%% generated with GNUPLOT 5.2p2 (Lua 5.3; terminal rev. 99, script rev. 102)
%% Wed 19 Jan 2022 03:44:52 PM CET
\path (0.000,0.000) rectangle (11.000,7.000);
\gpcolor{color=gp lt color border}
\gpsetlinetype{gp lt border}
\gpsetdashtype{gp dt solid}
\gpsetlinewidth{1.00}
\draw[gp path] (1.872,0.985)--(2.052,0.985);
\draw[gp path] (10.447,0.985)--(10.267,0.985);
\node[gp node right] at (1.688,0.985) {0};
\draw[gp path] (1.872,1.746)--(2.052,1.746);
\draw[gp path] (10.447,1.746)--(10.267,1.746);
\node[gp node right] at (1.688,1.746) {2,000};
\draw[gp path] (1.872,2.507)--(2.052,2.507);
\draw[gp path] (10.447,2.507)--(10.267,2.507);
\node[gp node right] at (1.688,2.507) {4,000};
\draw[gp path] (1.872,3.267)--(2.052,3.267);
\draw[gp path] (10.447,3.267)--(10.267,3.267);
\node[gp node right] at (1.688,3.267) {6,000};
\draw[gp path] (1.872,4.028)--(2.052,4.028);
\draw[gp path] (10.447,4.028)--(10.267,4.028);
\node[gp node right] at (1.688,4.028) {8,000};
\draw[gp path] (1.872,4.789)--(2.052,4.789);
\draw[gp path] (10.447,4.789)--(10.267,4.789);
\node[gp node right] at (1.688,4.789) {10,000};
\draw[gp path] (1.872,5.550)--(2.052,5.550);
\draw[gp path] (10.447,5.550)--(10.267,5.550);
\node[gp node right] at (1.688,5.550) {12,000};
\draw[gp path] (1.872,6.311)--(2.052,6.311);
\draw[gp path] (10.447,6.311)--(10.267,6.311);
\node[gp node right] at (1.688,6.311) {14,000};
\draw[gp path] (1.872,0.985)--(1.872,1.165);
\draw[gp path] (1.872,6.691)--(1.872,6.511);
\node[gp node center] at (1.872,0.677) {1};
\draw[gp path] (3.097,0.985)--(3.097,1.165);
\draw[gp path] (3.097,6.691)--(3.097,6.511);
\node[gp node center] at (3.097,0.677) {2};
\draw[gp path] (4.322,0.985)--(4.322,1.165);
\draw[gp path] (4.322,6.691)--(4.322,6.511);
\node[gp node center] at (4.322,0.677) {4};
\draw[gp path] (5.547,0.985)--(5.547,1.165);
\draw[gp path] (5.547,6.691)--(5.547,6.511);
\node[gp node center] at (5.547,0.677) {8};
\draw[gp path] (6.772,0.985)--(6.772,1.165);
\draw[gp path] (6.772,6.691)--(6.772,6.511);
\node[gp node center] at (6.772,0.677) {16};
\draw[gp path] (7.997,0.985)--(7.997,1.165);
\draw[gp path] (7.997,6.691)--(7.997,6.511);
\node[gp node center] at (7.997,0.677) {32};
\draw[gp path] (9.222,0.985)--(9.222,1.165);
\draw[gp path] (9.222,6.691)--(9.222,6.511);
\node[gp node center] at (9.222,0.677) {64};
\draw[gp path] (10.447,0.985)--(10.447,1.165);
\draw[gp path] (10.447,6.691)--(10.447,6.511);
\node[gp node center] at (10.447,0.677) {128};
\draw[gp path] (1.872,6.691)--(1.872,0.985)--(10.447,0.985)--(10.447,6.691)--cycle;
\node[gp node center,rotate=-270] at (0.000,3.838) {Decoding time (seconds)};
\node[gp node center] at (6.159,0.215) {Batch size (sentences)};
\node[gp node right] at (9.531,6.280) {\acs{ar} Big};
\gpcolor{rgb color={0.200,0.200,0.200}}
\gpsetlinewidth{2.50}
\draw[gp path] (9.715,6.280)--(10.079,6.280);
\draw[gp path] (5.297,6.691)--(5.547,6.157)--(6.772,5.149)--(7.997,4.653)--(9.222,4.467)%
  --(10.447,4.443);
\gpsetpointsize{4.00}
\gppoint{gp mark 1}{(5.547,6.157)}
\gppoint{gp mark 1}{(6.772,5.149)}
\gppoint{gp mark 1}{(7.997,4.653)}
\gppoint{gp mark 1}{(9.222,4.467)}
\gppoint{gp mark 1}{(10.447,4.443)}
\gppoint{gp mark 1}{(9.897,6.280)}
\gpcolor{color=gp lt color border}
\node[gp node right] at (9.531,5.818) {\acs{ar} Base};
\gpcolor{rgb color={0.800,0.800,0.800}}
\draw[gp path] (9.715,5.818)--(10.079,5.818);
\draw[gp path] (1.872,5.000)--(3.097,3.563)--(4.322,2.758)--(5.547,2.379)--(6.772,2.110)%
  --(7.997,1.990)--(9.222,1.960)--(10.447,1.969);
\gppoint{gp mark 2}{(1.872,5.000)}
\gppoint{gp mark 2}{(3.097,3.563)}
\gppoint{gp mark 2}{(4.322,2.758)}
\gppoint{gp mark 2}{(5.547,2.379)}
\gppoint{gp mark 2}{(6.772,2.110)}
\gppoint{gp mark 2}{(7.997,1.990)}
\gppoint{gp mark 2}{(9.222,1.960)}
\gppoint{gp mark 2}{(10.447,1.969)}
\gppoint{gp mark 2}{(9.897,5.818)}
\gpcolor{color=gp lt color border}
\node[gp node right] at (9.531,5.356) {Large};
\gpcolor{rgb color={0.800,0.000,0.000}}
\draw[gp path] (9.715,5.356)--(10.079,5.356);
\draw[gp path] (1.872,5.854)--(3.097,4.740)--(4.322,4.121)--(5.547,3.855)--(6.772,3.813)%
  --(7.997,3.842)--(9.222,3.891);
\gppoint{gp mark 3}{(1.872,5.854)}
\gppoint{gp mark 3}{(3.097,4.740)}
\gppoint{gp mark 3}{(4.322,4.121)}
\gppoint{gp mark 3}{(5.547,3.855)}
\gppoint{gp mark 3}{(6.772,3.813)}
\gppoint{gp mark 3}{(7.997,3.842)}
\gppoint{gp mark 3}{(9.222,3.891)}
\gppoint{gp mark 3}{(9.897,5.356)}
\gpcolor{color=gp lt color border}
\node[gp node right] at (9.531,4.894) {Base};
\gpcolor{rgb color={1.000,0.800,0.000}}
\draw[gp path] (9.715,4.894)--(10.079,4.894);
\draw[gp path] (1.872,1.955)--(3.097,1.859)--(4.322,1.846)--(5.547,1.862)--(6.772,1.898)%
  --(7.997,1.937)--(9.222,1.977)--(10.447,2.015);
\gppoint{gp mark 4}{(1.872,1.955)}
\gppoint{gp mark 4}{(3.097,1.859)}
\gppoint{gp mark 4}{(4.322,1.846)}
\gppoint{gp mark 4}{(5.547,1.862)}
\gppoint{gp mark 4}{(6.772,1.898)}
\gppoint{gp mark 4}{(7.997,1.937)}
\gppoint{gp mark 4}{(9.222,1.977)}
\gppoint{gp mark 4}{(10.447,2.015)}
\gppoint{gp mark 4}{(9.897,4.894)}
\gpcolor{color=gp lt color border}
\node[gp node right] at (9.531,4.432) {Small};
\gpcolor{rgb color={0.800,0.000,1.000}}
\draw[gp path] (9.715,4.432)--(10.079,4.432);
\draw[gp path] (1.872,1.497)--(3.097,1.459)--(4.322,1.461)--(5.547,1.482)--(6.772,1.517)%
  --(7.997,1.554)--(9.222,1.596)--(10.447,1.637);
\gppoint{gp mark 5}{(1.872,1.497)}
\gppoint{gp mark 5}{(3.097,1.459)}
\gppoint{gp mark 5}{(4.322,1.461)}
\gppoint{gp mark 5}{(5.547,1.482)}
\gppoint{gp mark 5}{(6.772,1.517)}
\gppoint{gp mark 5}{(7.997,1.554)}
\gppoint{gp mark 5}{(9.222,1.596)}
\gppoint{gp mark 5}{(10.447,1.637)}
\gppoint{gp mark 5}{(9.897,4.432)}
\gpcolor{color=gp lt color border}
\node[gp node right] at (9.531,3.970) {Micro};
\gpcolor{rgb color={0.000,0.000,0.800}}
\draw[gp path] (9.715,3.970)--(10.079,3.970);
\draw[gp path] (1.872,1.349)--(3.097,1.326)--(4.322,1.332)--(5.547,1.356)--(6.772,1.389)%
  --(7.997,1.427)--(9.222,1.468)--(10.447,1.510);
\gppoint{gp mark 6}{(1.872,1.349)}
\gppoint{gp mark 6}{(3.097,1.326)}
\gppoint{gp mark 6}{(4.322,1.332)}
\gppoint{gp mark 6}{(5.547,1.356)}
\gppoint{gp mark 6}{(6.772,1.389)}
\gppoint{gp mark 6}{(7.997,1.427)}
\gppoint{gp mark 6}{(9.222,1.468)}
\gppoint{gp mark 6}{(10.447,1.510)}
\gppoint{gp mark 6}{(9.897,3.970)}
\gpcolor{color=gp lt color border}
\node[gp node right] at (9.531,3.508) {Tiny};
\gpcolor{rgb color={0.000,0.800,0.000}}
\draw[gp path] (9.715,3.508)--(10.079,3.508);
\draw[gp path] (1.872,1.078)--(3.097,1.079)--(4.322,1.089)--(5.547,1.105)--(6.772,1.127)%
  --(7.997,1.151)--(9.222,1.177)--(10.447,1.206);
\gppoint{gp mark 7}{(1.872,1.078)}
\gppoint{gp mark 7}{(3.097,1.079)}
\gppoint{gp mark 7}{(4.322,1.089)}
\gppoint{gp mark 7}{(5.547,1.105)}
\gppoint{gp mark 7}{(6.772,1.127)}
\gppoint{gp mark 7}{(7.997,1.151)}
\gppoint{gp mark 7}{(9.222,1.177)}
\gppoint{gp mark 7}{(10.447,1.206)}
\gppoint{gp mark 7}{(9.897,3.508)}
\gpcolor{color=gp lt color border}
\gpsetlinewidth{1.00}
\draw[gp path] (1.872,6.691)--(1.872,0.985)--(10.447,0.985)--(10.447,6.691)--cycle;
%% coordinates of the plot area
\gpdefrectangularnode{gp plot 1}{\pgfpoint{1.872cm}{0.985cm}}{\pgfpoint{10.447cm}{6.691cm}}
\end{tikzpicture}
%% gnuplot variables

  \vspace{1\baselineskip}

  \begin{tabular}{lrrrrrrrr}
    \toprule
    Batch size & 1 & 2 & 4 & 8 & 16 & 32 & 64 & 128 \\


    \midrule
    Full output \\
    \acs{ar} -- Large & 52,087 & 33,189 & 23,478 & 16,731 & 12,166 & 10,151 & 9,370 & 9,244 \\
    \acs{ar} -- Base  & 16,293 &  9,727 &  7,191 &  5,420 &  3,635 &  2,925 & 2,707 & 2,664 \\
    \addlinespace
    Large & 14,542 & 11,320 & 9,534 & 8,657 & 8,357 & 8,247 & 8,238 & \\
    Base  &  3,508 &  3,126 & 2,979 & 2,927 & 2,921 & 2,920 & 2,934 & 2,948 \\
    Small &  2,346 &  2,079 & 1,975 & 1,933 & 1,921 & 1,921 & 1,936 & 1,950 \\
    Micro &  1,952 &  1,728 & 1,641 & 1,600 & 1,588 & 1,587 & 1,607 & 1,627 \\
    Tiny  &    784 &    719 &   697 &   687 &   685 &   694 &   701 &   719 \\


    \midrule
    Shortlist \\
    \acs{ar} -- Large & 41,168 & 27,977 & 18,914 & 13,597 & 10,946 & 9,643 & 9,154 & 9,090 \\
    \acs{ar} -- Base  & 10,555 &  6,776 &  4,660 &  3,664 &  2,957 & 2,643 & 2,564 & 2,587 \\
    \addlinespace
    Large & 12,799 & 9,870 & 8,245 & 7,545 & 7,434 & 7,511 & 7,639 & \\
    Base  &  2,549 & 2,298 & 2,263 & 2,306 & 2,399 & 2,503 & 2,609 & 2,707 \\
    Small &  1,346 & 1,246 & 1,250 & 1,306 & 1,399 & 1,497 & 1,606 & 1,714 \\
    Micro &    958 &   897 &   913 &   974 & 1,062 & 1,163 & 1,271 & 1,380 \\
    Tiny  &    244 &   246 &   273 &   316 &   373 &   437 &   506 &   582 \\


    \bottomrule
  \end{tabular}

  \caption{Wall times to translate one million sentences (in seconds) on 36 CPU
    cores with different batch size settings. The graph shows the decoding
    times with shortlisting.}%
  \label{fig:throughput:cpu36}
\end{figure}


% ------------------------------------------------------------------------------
\subsection{Discussion}%
\label{subsec:results:discussion}
% ------------------------------------------------------------------------------

Table \ref{tab:wmt14-weight-avg} in Section \ref{subsec:results:quality} shows
that we achieve state-of-the-art scores in non-autoreg\-ressive translation on the
\acs{wmt}~14 test set, outperforming both single-step and iterative
methods. However, the results on the \acs{wmt}~21 test set in Table
\ref{tab:wmt21-scores} show that there is still a great deal of room for
improvement, especially when looking at the \acs{comet} scores. We believe that
these findings should motivate future research not to evaluate translation
quality exclusively on the \acs{wmt}~14 test set.

We compare our models with one of the best performing submissions in the
efficiency task -- the University of Edinburgh's ``base'' model
\citep{Behnke-wmt21-speed} -- in Table \ref{tab:efficiency:comparison}. It is
clear from the comparison that the Edinburgh \acl{ar} model is superior to our
\ac{nar} models in most regards. One exception is the GPU decoding latency. As
we discuss in Section \ref{sec:nat:discussion}, these conditions are the only
scenario considered in most of the related studies.

\begin{table}
  \centering

  \begin{tabular}{lrrrrrr}
    \toprule
    & \multicolumn{3}{c}{Translation quality} & \multicolumn{3}{c}{Decoding time (seconds)} \\
    & {\small \acs{chrf}} & {\small \acs{comet}} & {\small \acs{bleu}} & {\small GPU, b>1} & {\small GPU, b=1} & {\small CPU, b>1} \\
    \midrule
    Edinburgh base & 61.5 & 0.527 & 55.3 & 140 & 16,851 & 500 \\
    \midrule
    \acs{ar} -- Large (teacher) & 59.2 & 0.411 & 50.5 & 1,918 & {\it > 24h} & 9,090 \\
    \acs{ar} -- Base (student) & 59.5 & 0.455 & 51.6 & 1,465 & {\it > 24h} & 2,587 \\
    \addlinespace
    \acs{nar} -- Large & 58.6 & 0.149 & 47.8 & 782 & 7,020 & 7,434 \\
    \acs{nar} -- Micro & 57.3 & -0.008 & 43.5 & 311 & 2,322 & 897 \\
    \bottomrule
  \end{tabular}

  \caption{Comparison of our models with the Edinburgh ``base'' model submitted
    to the \acs{wmt} Efficiency Shared Task \citep{Behnke-wmt21-speed}. Columns
    denoted \emph{b>1} show the best result using batching, \emph{b=1} is
    measured with a single sentence in the batch. CPU times were measured using
    36 CPU cores.} %
  \label{tab:efficiency:comparison}
\end{table}

% %%%%%%%%%%%%%%%%%%%%%%%%%%%%%%%%%%%%%%%%%%%%%%%%%%%%%%%%%%%%%%%%%%%%%%%%%%%%%
\chapter{Conclusions}
% %%%%%%%%%%%%%%%%%%%%%%%%%%%%%%%%%%%%%%%%%%%%%%%%%%%%%%%%%%%%%%%%%%%%%%%%%%%%%


In this thesis, we explored the possibilities of \ac{nar} models for
\ac{nmt}. We described the commonly used \ac{nmt} models and showed techniques
to effectively train them and to use them for translation.

We presented a comprehensive survey of the related research efforts that have
been made so far in this rapidly developing field. We described \ac{nar}
methods based on latent variables, iterative refinement, or on the relaxed
alignment between the output predictions and the ground-truth labels.

We pointed out some of the drawbacks in related \ac{nar} approaches, such as
using weaker baselines both in terms of translation quality and decoding speed,
and identified some of the flaws in the evaluation methodology.

In the first of the two experimental chapters (Chapter \ref{chap:nar-nmt-ctc}),
we proposed a novel method for \ac{nar} \ac{nmt} based on \ac{ctc} which has
been adopted by the \ac{nar} research community as one of the basic approaches
since its original publication.

In Chapter \ref{chap:experiments}, we aim at improving our \acs{ctc}-based
approach by using knowledge distillation using a strong \ac{ar} teacher model.
We compare our results to both the \acs{wmt}~14 benchmark and the results of
the \acs{wmt}~21 Efficiency Shared Task \citep{heafield-etal-2021-findings}. We
find that even though our models are among the best performing on \acs{wmt}~14,
there is a large room for improvement when we compare them to highly optimized
\acl{ar} models submitted to the efficiency task.

To conclude, over the past few years, the research community has set its hopes
on the \ac{nar} models because, in theory, the decoding has lower time
complexity. Moreover, the literature on this topic often claims that \ac{nar}
models already match the performance of their \ac{ar} counterparts. However,
our findings suggest that, in fair comparison with strong efficient baselines,
many research studies concluding that \ac{nar} models are superior over \ac{ar}
models may be overclaiming. To avoid these problems in the future, research of
\ac{nar} models should take into account findings of the efficiency task,
evaluate the translation quality also on newer test sets, and measure the
decoding speed under various conditions, such as batched GPU and CPU decoding.



%%% Local Variables:
%%% mode: latex
%%% TeX-master: "main"
%%% End:
