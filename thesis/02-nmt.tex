% %%%%%%%%%%%%%%%%%%%%%%%%%%%%%%%%%%%%%%%%%%%%%%%%%%%%%%%%%%%%%%%%%%%%%%%%%%%%%
\chapter{Neural Machine Translation}
\label{chap:nmt}
% %%%%%%%%%%%%%%%%%%%%%%%%%%%%%%%%%%%%%%%%%%%%%%%%%%%%%%%%%%%%%%%%%%%%%%%%%%%%%

\begin{markdown}
V týhle kapitole bude:

* basics:
    - segmentation
    - word2id
    - model
    - softmax
    - training objective
    - decoding algorithms

* models
    - rnn based
    - transformer

sources:
bahdanau+luong a cho, vaswani, BPE, sentencepiece, lstm+gru

víceméně náplň kompendia.


### Pohádka ###

**preface**
\gls{NMT} je to jak se dělá překlad;


\end{markdown}


\gls{nmt} is the current state-of-the-art approach to \gls{mt}. As the name
suggests, the underlying machine learning concept in \gls{nmt} systems are
neural networks.

Most of the \gls{nmt} architectures have two parts: an \emph{encoder} and a
\emph{decoder}. The encoder processes the input sentence and creates a hidden
representation. The decoder then accesses this hidden representation and
generates the output sentence.
