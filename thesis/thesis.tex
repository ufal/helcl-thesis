%%% The main file. It contains definitions of basic parameters and includes all other parts.

%% Settings for single-side (simplex) printing
% Margins: left 40mm, right 25mm, top and bottom 25mm
% (but beware, LaTeX adds 1in implicitly)
\documentclass[12pt,notitlepage,a4paper,openright]{report}
\pagestyle{plain}

%\begin{filecontents*}{\jobname.xmpdata}
%  \Author{Jindřich Helcl}
%  \Title{Non-Autoregressive Neural Machine Translation}
%  \Keywords{non-autoregressive neural machine translation, neural machine translation, deep learning}
%  \Subject{TODO}
%  \Publisher{Charles University}
%\end{filecontents*}

% fix pdfx
\usepackage{etoolbox}
\makeatletter
\@ifl@t@r\fmtversion{2021-06-01}%
 {\AddToHook{package/after/xmpincl}
   {\patchcmd\mcs@xmpincl@patchFile{\if\par}{\ifx\par}{}{\fail}}}{}
\makeatother


\usepackage[usenames,dvipsnames,svgnames,table,rgb]{xcolor}
\usepackage[a-2u]{pdfx}
\usepackage{fontspec}
% TODO zapnout až bude potřeba
%\usepackage{microtype}
\usepackage[czech,english]{babel}
\usepackage{lmodern}
\usepackage{textcomp}
\usepackage[defaultlines=4,all]{nowidow}

\usepackage[hybrid]{markdown}

\usepackage{graphicx} % nezbytné pro standardní vkládání obrázků do dokumentu
\usepackage[twoside, inner=3.7cm, outer=2.9cm, top=2.6cm, bottom=3.4cm]{geometry} % nastavení dané velikosti okrajů
\usepackage{thesis}
\usepackage[round]{natbib}
\usepackage{multirow}
\usepackage{arydshln} % dashed lines in tables
\usepackage{array}
\usepackage{amssymb,latexsym,pifont}
\usepackage{amsmath}
%\usepackage{longtable}
\usepackage{enumitem} % custom lists
\usepackage[normalem]{ulem} % underlining
\usepackage{setspace} % radkovani
\usepackage{varioref} % nice references (above/below)
\usepackage[above,section]{placeins} % avoid figures pushed at end of chapters
\usepackage{listings}


\usepackage{tabularx}
\usepackage{booktabs} % nicer lines in table
\usepackage{multicol}
\usepackage{tikz}
\usepackage{pgfplots}
\usepackage{gnuplot-lua-tikz}
\usetikzlibrary{shapes.geometric}
\usepackage{epstopdf}
\usepackage{algorithmicx}
\usepackage{algorithm}
\usepackage{algpseudocode}

\usepackage[acronym]{glossaries}
\usepackage[shortcuts=ac,automake]{glossaries-extra}
\preto\chapter{\glsresetall}

\setabbreviationstyle[acronym]{long-short}

\usepackage{subcaption} % sub figures in a fiture
\usepackage{standalone} % include standoalone tikz images
\usepackage{bibentry}

\hypersetup{
    colorlinks=false,
    pdfborder={0 0 0},
}

\makeglossaries{}
\newacronym{nmt}{NMT}{neural machine translation}
\newacronym{nlp}{NLP}{natural language processing}
\newacronym{rnn}{RNN}{Recurrent neural network}
\newacronym{cnn}{CNN}{convolutional neural network}
\newacronym{san}{SAN}{Self-attentive network}
\newacronym{ai}{AI}{Artificial intelligence}
\newacronym{mmt}{MMT}{Multimodal machine translation}
\newacronym{wmt}{WMT}{Conference on Machine Translation}
\newacronym{ml}{ML}{Machine learning}
\newacronym{lstm}{LSTM}{Long Short-Term Memory}
\newacronym{gru}{GRU}{Gated Recurrent Unit}
\newacronym{relu}{ReLU}{Rectified Linear Unit}
\newacronym{mt}{MT}{Machine translation}
\newacronym{lm}{LM}{Language model}
\newacronym{oov}{OOV}{out-of-vocabulary}
\newacronym{bpe}{BPE}{byte-pair encoding}
\newacronym{nat}{NAT}{non-autoregressive NMT}
\newacronym{nar}{NAR}{non-autoregressive}
\newacronym{ctc}{CTC}{connectionist temporal classification}
\newacronym{npd}{NPD}{noisy parallel decoding}
\newacronym{bleu}{BLEU}{bilingual evaluation understudy}

% Czech babel conflicts with cline, hacky fix (http://tex.stackexchange.com/questions/111999/slovak-and-czech-babel-gives-problems-with-cmidrule-and-cline):
% - basically disables hyphenation in tables, but it's not used anyway so it doesn't matter
\preto\tabular{\shorthandoff{-}}
%
%
\hyphenation{%
da-ta-sets
da-ta-set
} % -- custom hyphenation

\setmainfont[Ligatures=Common]{Linux Libertine}
\setsansfont[Scale=MatchLowercase]{DejaVu Sans}
\setmonofont[Scale=MatchLowercase]{DejaVu Sans Mono}


\setstretch{1.1} % radkovani

\expandafter\def\expandafter\quote\expandafter{\quote\small} % mensi pismo u quotations

% orphan & widow control
%\clubpenalty 10000
%\widowpenalty 10000

% odstup poznamek od textu
\def\footnoteskip#1{
  \renewcommand\footnoterule{
     \vspace{#1}
     \hrule width 0.4\columnwidth%
     \vspace{3pt}
}
}
\footnoteskip{0.8em}

% v obsahu budou jen sections
\setcounter{tocdepth}{2}
% necisluju subsections
\setcounter{secnumdepth}{2}

%% cutting down warnings
%\hfuzz=2pt
%\hbadness=10000

% force-ordering citations according to dummy keys
\newcommand{\dummybiborderkey}[1]{}

\newcommand{\eos}{\texttt{</s>}}

\DeclareMathOperator*{\softmax}{softmax}
\DeclareMathOperator*{\argmax}{argmax}
\DeclareMathOperator{\relu}{ReLu}
\def\Pr{\ensuremath\mathsf{P}}
\let\emptyset\varnothing{}
\DeclareMathOperator*{\sgn}{sgn}
\DeclareMathOperator{\attn}{Attn}
\DeclareMathOperator{\multihead}{Multihead}
\DeclareMathOperator{\concat}{concat}

\newcommand{\veryshortarrow}[1][3pt]{\mathrel{%
     \vcenter{\hbox{\rule[-.5\fontdimen8\textfont3]{#1}{\fontdimen8\textfont3}}}%
     \mkern-4mu\hbox{\usefont{U}{lasy}{m}{n}\symbol{41}}}}

\newcommand{\paperdisclaim}[1]{%
\begin{center}\begin{minipage}{0.9\textwidth}
\footnotesize\it #1
\end{minipage}\end{center}
}
\newcommand{\R}[2]{#1 \tiny $\pm$ \small #2}

\def\JH#1{{\color{blue}JH: \it #1}}


\title{Non-Autoregressive Neural Machine Translation}

\def\fulldate{}
\author{Jindřich Helcl}
\date{2021}
\dept{Institute of Formal and Applied Linguistics}
\supervisor{prof. RNDr. Jan Hajič, Dr.}
\studyprogram{Computer Science}
\studyfield{Computational Linguistics}



\begin{document}

%
%
%
\renewcommand{\thepage}{\roman{page}}
\selectlanguage{english}
\maketitle

\pagestyle{plain}
\normalsize % nastavení normální velikosti fontu
\setcounter{page}{2} % nastavení číslování stránek

\cleardoublepage{}
\ \vspace{10mm}

\noindent \it

\vspace{\fill}
\noindent \rm
I declare that I carried out this doctoral thesis independently,
  and only with the cited sources, literature and other professional sources.

I understand that my work relates to the rights and obligations
  under the Act No.~121/2000 Coll., the Copyright Act, as amended,
  in particular the fact that Charles University has the right
  to conclude a license agreement on the use of this work as a school work
  pursuant to Section~60 paragraph~1 of the Copyright Act.

\vspace{2cm}
\noindent Prague, \JH{datum}, 2021 \hspace{\fill}\theauthor\\ % doplňte patřičné datum, jméno a příjmení

%%%   Výtisk pak na tomto míste nezapomeňte PODEPSAT!
%%%                                         *********

\cleardoublepage{} % přechod na novou stránku
\pagestyle{plain}
%\singlespacing
\addcontentsline{toc}{chapter}{English Abstract}
%%% Následuje strana s abstrakty. Doplňte vlastní údaje.
%\selectlanguage{english}
\begin{description}[leftmargin=7.5em,labelwidth=7em,labelindent=0em,labelsep=0.5em]
\item[Title:] \thetitle{}
\item[Author:] \theauthor{}
\item[Department:] \thedept{}
\item[Supervisor:] \thesupervisor{},\\ \thedept{}
\end{description}
\subsubsection{Abstract:}

In recent years, a number of mehtods for improving the decoding speed of neural
machine translation systems have emerged.
%
One of the approaches that proposes fundamental changes to the model
architecture are non-autoregressive models.
%
In standard autoregressive models, the output token distributions are
conditioned on the previously decoded outputs.
% Autoregressive models impose conditional dependence of the generated words on
% the previously decoded outputs.
%
The conditional dependence allows the model to keep track of the state of the
decoding process, which improves the fluency of the output.
%
On the other hand, it requires the neural network computation to be run
sequentially, and thus it cannot be parallelized.
%
Non-autoregressive models impose conditional independence on the output
distributions, which means that the decoding process is parallelizable and
hence the decoding speed improves.
%
A major drawback of this approach is lower translation quality compared to the
autoregressive models.
%
The goal of non-autoregressive machine translation is to improve the
translation quality, while retaining high decoding speed.
%
In this thesis, we explore the research progress so far, and bring together a
number of techniques to advance towards the stated goal.
%
We experiement with non-autoregressive models trained with connectionist
temporal classification.
%
\JH{sentence or two about the methods and findings}




%%% Local Variables:
%%% mode: latex
%%% TeX-master: "thesis"
%%% End:


\begin{description}[leftmargin=7.5em,labelwidth=7em,labelindent=0em,labelsep=0.5em]
        %
\item[Keywords:] machine translation, deep learning, natural language processing
    %
\end{description}


\cleardoublepage{}
\addcontentsline{toc}{chapter}{Czech Abstract}
\selectlanguage{czech}
\begin{description}[leftmargin=7.5em,labelwidth=7em,labelindent=0em,labelsep=0.5em]
\item[Název práce:] Neautoregresivní neuronový strojový překlad
\item[Autor:] \theauthor{}
\item[Katedra:] Ústav formální a aplikované lingvistiky
\item[Vedoucí práce:] \thesupervisor,\\ Ústav formální a aplikované lingvistiky
\end{description}

\subsubsection{Abstrakt:}

V poslední době nabídl výzkum strojového překladu nové metody pro zrychlení
generování.
%
Jedním z navrhovaných metod je takzvaný neautoregresivní neuronový strojový
překlad.
%
V klasických autoregresivních překladových systémech jsou výstupní
pravděpodobnostní rozdělení modelována podmíněně na předchozích výstupech.
%
Tato závislost umožňuje modelům sledovat stav překládání a obvykle vede ke
generování velmi plynulých textů.
%
Autoregresivní postup je však ze své podstaty sekvenční a nelze jej
paralelizovat.
%
Neautoregresivní systémy modelují pravděpodobnosti jednotlivých cílových slov
jako navzájem podmíněně nezávislé, což znamená, že dekódování lze paralelizovat
snadno.
%
Nevýhodou je ovšem nízká kvalita překladu ve srovnání s modely
autoregresivními.
%
Cíl výzkumu neautoregresivních metod strojového překladu je zlepšit kvalitu
překladu a zároveň uchovat vysokou rychlost dekódování.
%
Naše práce předkládá rešerši publikovaných metod a poukazuje na některé
nedostatky plynoucí z obecně přijímané evaluační metodologie.
%
Popisujeme experimenty s neautoregresivními modely trénovaných pomocí takzvané
\uv{connectionist temporal classification}.
%
Z našich výsledků plyne, že i když dosahujeme nejlepších výsledků mezi
neautoregresivními modely na datech z WMT z roku 2014, při porovnání s
nejnovějšími optimalizovanými autoregresivními systémy tyto modely pořád
zaostávají.



%%% Local Variables:
%%% mode: latex
%%% TeX-master: "thesis"
%%% End:


\begin{description}[leftmargin=7.5em,labelwidth=7em,labelindent=0em,labelsep=0.5em]
%
\item[Klíčová slova:] strojový překlad, hluboké učení, zpracování přirozených jazyků
%
\end{description}

\selectlanguage{english}




\cleardoublepage{}
\ \vspace{10mm}

\addcontentsline{toc}{chapter}{Acknowledgements}
\subsection*{Acknowledgements}

{
  děkuju. tos řek hezky.
}

\vfill


{\noindent\footnotesize TODO akcknowledgementy}

\cleardoublepage{}
\addcontentsline{toc}{chapter}{Table of Contents}
\tableofcontents % vkládá automaticky generovaný obsah dokumentu

\cleardoublepage{}
\renewcommand{\chapterheadstartvskip}{\vspace*{-10mm}} % mezera u kapitoly
\setstretch{1.2} % radkovani

%
% TEXT START
%
\renewcommand{\thepage}{\arabic{page}}
\setcounter{page}{1}

\sloppy
% %%%%%%%%%%%%%%%%%%%%%%%%%%%%%%%%%%%%%%%%%%%%%%%%%%%%%%%%%%%%%%%%%%%%%%%%%%%%
\chapter{Introduction}
\label{chap:intro}
% %%%%%%%%%%%%%%%%%%%%%%%%%%%%%%%%%%%%%%%%%%%%%%%%%%%%%%%%%%%%%%%%%%%%%%%%%%%%


\JH{rewrite celý}

% It is universally agreed that context plays a vital role in human communication.

% při rozhovoru je kontext kromě toho, co bylo řečeno i to, kde, kdo s kým a za
% jakých okolností komunikuje. V případě korespondence je kontext omezenější

% In this thesis, we present two case studies aimed on this phenomenon. In the
% first study, we provide additional context to a \gls{nmt} system and observe the
% changes in translation quality. In the other case study, we go the other way: we
% limit the context available to the decoder and examine the model behavior.

This thesis is structured as follows. In Chapter \ref{chap:nmt}, we describe the
basics of \gls{nmt} as well as recent advancements in the field. % Chapter
% \ref{chap:context} summarizes the scientific contributions exploring the
% influence of context in \gls{nmt} from different perspectives. \JH{rewrite, NAT
%   chapter: \ref{chap:nat}}
%   We present two
% case studies focused on the importance of context in \gls{nmt} in Chapters
% \ref{chap:mmmt} and \ref{chap:nat}.
%The discussion of the results observed in the case studies is given in Chapter
%\ref{chap:discuss}.


%%% Local Variables:
%%% mode: latex
%%% TeX-master: "thesis"
%%% End:

% %%%%%%%%%%%%%%%%%%%%%%%%%%%%%%%%%%%%%%%%%%%%%%%%%%%%%%%%%%%%%%%%%%%%%%%%%%%%%
\chapter{Neural Machine Translation}
\label{chap:nmt}
% %%%%%%%%%%%%%%%%%%%%%%%%%%%%%%%%%%%%%%%%%%%%%%%%%%%%%%%%%%%%%%%%%%%%%%%%%%%%%

\begin{markdown}
V týhle kapitole bude:

* basics:
    - segmentation
    - word2id
    - model
    - softmax
    - training objective
    - decoding algorithms

* models
    - rnn based
    - transformer

sources:
bahdanau+luong a cho, vaswani, BPE, sentencepiece, lstm+gru

víceméně náplň kompendia.


### Pohádka ###

**preface**

* \gls{nmt} je to jak se dělá překlad
* je to sequence-to-sequence a architektury jsou encoder-decoder
* protože zpracováváme sekvence, používáme speciální typy neuronek
* dva hlavní pro NLP - rnn a transformer
* v týhle kapitole si to postupně vysvětlíme

**zpracování textu neuronovýma sítěma**

* neuronky jsou matematický funkce a jakožto takový musí žrát číselnej vstup
* myšlenka je převést tokeny na čísla
* rozdělování textu na tokeny je netriviální problém. v kapitolce o tokenizaci popíšem hlavní metody


\end{markdown}


\gls{nmt} is the current state-of-the-art approach to \gls{mt}. As the name
suggests, the underlying machine learning concept in \gls{nmt} systems are
neural networks.

Most of the \gls{nmt} architectures have two parts: an \emph{encoder} and a
\emph{decoder}. The encoder processes the input sentence and creates a hidden
representation. The decoder then accesses this hidden representation and
generates the output sentence.


\section{Processing Text with Neural Networks}

%Since sentences come in various lenghts,

\subsection{Tokenization}


\paragraph{Open Vocabulary Problem.}



\paragraph{BPE.}
Byte-pair encoding \citep{sennrich2016bpe} is an approach which tackles the
open vocabulary problem by splitting words to so-called subword units.  The
idea is to devise a vocabulary of a pre-defined size, such that nearly every
word can be composed using the units from the vocabulary. An additional
requirement is for the vocabulary to contain as many frequent words as
possible, so only rare words need to be split to more subwords.


\paragraph{SentencePiece.}


\section{Recurrent Neural Networks}

\section{Transformer Model}


\section{Training vs. Inference}










\begin{markdown}

\end{markdown}
%\input{03-context.md}



%\include{04-mmt}
% %%%%%%%%%%%%%%%%%%%%%%%%%%%%%%%%%%%%%%%%%%%%%%%%%%%%%%%%%%%%%%%%%%%%%%%%%%%%%
\chapter{Non-Autoregressive NMT}
\label{chap:nat}
% %%%%%%%%%%%%%%%%%%%%%%%%%%%%%%%%%%%%%%%%%%%%%%%%%%%%%%%%%%%%%%%%%%%%%%%%%%%%%

So far we discussed the influence of combining additional sources of information
on translation. This chapter presents a case study of \gls{nat} in which the
contextual information is limited.

\gls{nat} is a recent subtask of \gls{nmt} in which the decoding process cannot
access the previously decoded outputs, imposing conditional independence on the
output token probability distributions. This assumption allows parallelization
of the decoding which can significantly reduce the latency of the translation
system. On the other hand, it also presents a challenge to the language model,
which usually leads to poorer translation quality.

This chapter begins with the overview of the non-autoregressive methods (Section
\ref{sec:nat-methods}). In Section \ref{sec:nat-ctc}, we introduce
non-autoregressive \gls{nmt} with \gls{ctc}. We present our experiments with
\gls{ctc} in Section \ref{sec:ctc-experiments}. We summarize our contributions
and outline the possible future efforts in Section \ref{sec:nat-future}.


\section{Related Work}
\label{sec:nat-methods}

\citep{lee2018deterministic} \citep{gu2017nonautoregressive}
\citep{ghazvininejad2019mask} \citep{mansimov2019generalized}

\section{Non-Autoregressive NMT with Connectionist Temporal Classification}
\label{sec:nat-ctc}

\JH{base this on the End-to-End paper}


\section{Improving Fluency using N-gram Language Models}
\label{sec:nat-lm}


\section{Non-Autoregressive Model Optimizations}
\label{sec:nat-opt}



% %%%%%%%%%%%%%%%%%%%%%%%%%%%%%%%%%%%%%%%%%%%%%%%%%%%%%%%%%%%%%%%%%%%%%%%%%%%%%
\chapter{Non-autoregressive NMT with Connectionist Temporal Classification}
\label{chap:nar-nmt-ctc}
% %%%%%%%%%%%%%%%%%%%%%%%%%%%%%%%%%%%%%%%%%%%%%%%%%%%%%%%%%%%%%%%%%%%%%%%%%%%%%

In this chapter, we lay grounds for the \gls{nar} approaches studied in this
thesis. We describe our experiments with an architecture based on the \gls{ctc}
loss \citep{libovicky-helcl-2018-end}. \JH{ok to self-cite when sections will
  be based on this?}

% -----------------------------------------------------------------------------
\section{Connectionist Temporal Classification}
\label{sec:ctc}
% -----------------------------------------------------------------------------

\Gls{ctc} \citep{graves2006connectionist} is a method for training neural
networks on sequential data. Originally applied to the phonetic labelling task,
but later successfully adapted in related areas, including \gls{asr} or
handwriting recognition \citep{liwicki2007novel, eyben2009speech,
  graves2014towards}.

The main strength of \gls{ctc} becomes evident in tasks where the input and
output labels are weakly or not at all aligned, for example in situations where
the observed input sequence is considerably longer than the target output
sequence -- hence the application to \gls{asr}, where the number of extracted
features per second is higher than the number of phonemes uttered per second.
\JH{Confirm this.}

Training neural networks with \gls{ctc} is independent on the actual neural
network architecture. The \gls{ctc} loss function can be applied on any network
with sequential outputs. Thus, this method is applicable to both \glspl{rnn}
and the Transformer model.

Models trained with the \gls{ctc} assume that the alignment between the input
(e.g. a group of frames in an audio signal) and the output (e.g. a phoneme)
states is unknown. A variable number of frames in a row can encode a single
phoneme. Similarly, in translation, multiple words in the source language may
correspond to any number of (even non-consecutive) words in the target
language.

The idea behind \gls{ctc} is to allow some states to produce no output. This is
realized by introducing a special blank token in the target vocabulary.
Optionally, identical outputs produced by multiple consecutive states may be
merged and considered a single output. Because of these properties, there are
groups of equivalent output sequences, which all represent the same target, as
illustrated in Figure~\ref{fig:ctc-equivalent-sequences}.

\begin{figure}
  \centering
  \begin{minipage}{\textwidth}
    \begin{equation*}
        \text{a cat sat on a mat} =
        \begin{cases}
          & \text{a <blank> cat sat on a <blank> mat} \\
          & \text{a a cat cat sat on a mat} \\
          & \text{a <blank> cat cat sat on a mat} \\
        \end{cases}
    \end{equation*}
  \end{minipage}
  \caption{A group of output sequences of equal length which all represent the
    same target in CTC.} %
  \label{fig:ctc-equivalent-sequences}
\end{figure}

In the standard sequence-to-sequence architectures, the value of the loss
function is defined as the sum of the cross entropies of each output state with
respect to the target sequence (see Equation \ref{eq:loss}). In \gls{ctc}, the
loss is defined as the sum of cross-entropy losses of all of the output
sequences equivalent to the given target sequence:
%
\begin{equation}
  J_{\theta}^{\text{CTC}} = - \sum_{(x, y) \in D} \sum_{y' \sim y}  \log p(y' | x, \theta)
  \label{eq:ctc-loss}
\end{equation}
%
where $\sim$ denotes the equivalence relation.  \JH{$J_{\theta}$ should perhaps
  be $J(\theta)$. Also, consider the $\sim$ sign.}

The inner summation in Equation \ref{eq:ctc-loss} is computed over all possible
sequences equivalent to the label sequence. For technical purposes, the label
sequences are limited to a fixed length, which greatly reduces the number of
acceptable hypotheses. However, the number of equivalent hypotheses of a given
length still grows exponentially with the sequence length -- in \gls{ctc}, the
fixed length is always set to be longer than the label sequence. \JH{confirm
  the exponential claim}

The summation over the large set of equivalent sequences can be
implemented using dynamic programming. When both the length of the output and
the length of the target sequences are known, there is a constant number of the
blank tokens to be generated. The process of computing the loss of the whole
output sequence is divided into computing the partial losses with respect to
the possible label prefixes.

\begin{figure}
  \centering

  \includegraphics[width=13cm]{img/ctc_schema.png}

  \caption{An illustration of the algorihm for the CTC loss computation. Each
    node denotes producing either a token from the label sequence, or the blank
    token. Each path from one of the two top-left nodes to one of the two
    bottom-right nodes corresponds to one of the equivalent sequences.  }
  \label{fig:ctc-dynamic-programming}
\end{figure}

The \gls{ctc} loss computation is illustrated in Figure
\ref{fig:ctc-dynamic-programming}. The rows represent tokens from the label
sequence plus the optional blank tokens. The columns represent the output state
sequence.  Each node in the graph denotes generating a label from an output
state. The arrows show valid transitions between the generation steps. An arrow
can only go down one or two rows, or horizontally.  The horizontal arrows
denote repeated generation of the same label. These labels are later merged to
form a single output. An arrow can only go two rows down when the skipped row
corresponds to the blank token, so no target tokens are left out. Each path in
the diagram therefore shows one of the equivalent sequences that lead to
generating the given label sequence.

Using the idea that many of the paths from left to right in the diagram share
segments, we can apply dynamic programming to compute the sum of losses across
all paths without the need to enumerate each of them. A node on coordinates
$(i,j)$ stores the accumulated losses for the all path prefixes that lead to
the node, added with the negative log likelihood of the label on the $i$-th row
being generated by the $j$-th output state. The two bottom-right nodes then
store the sum of losses of all the paths.


\JH{doplnit}
The training of the network with \gls{ctc} is done
by minimizing the \gls{ctc} loss function, which is defined as follows.

% -----------------------------------------------------------------------------
\section{Model Architecture}
\label{sec:ctc:arch}
% -----------------------------------------------------------------------------

As said in the previous chapter, training models with \gls{ctc} does not impose
any requirements on the model architecture. In our experiments, we aim for a
reasonable comparison between our proposed approach and the state-of-the-art
autoregressive models. We adapt the Transformer model and use similar
hyper-parameters where applicable.

Non-autoregressive models generate the outputs in parallel, which requires that
the output length is known beforehand. In autoregressive models, the end of
sequence is indicated by a special end symbol, and the constraint on maximum
length is merely a technical aspect.

To leverage the ability to output empty tokens to the full extent, the output
length should be set to a higher number than the length of the target sequence.
Since the length estimation does not need to be accurate, we select a number
$k$ and we set the target sequence length to be $k$ times longer than the
source length. Note that in case the selected length is shorter than the label
sequence, the model will not be able to generate the whole target sequence.


\begin{figure}
  \centering

  \def\inputsize{7}

\begin{tikzpicture}[]

\draw (\inputsize / 2 + 0.1, -0.1) node {Input token embeddings};

\foreach \i in {0,...,\inputsize} {
	\draw (\i,-0.5) rectangle (\i+0.2,-1);
    \draw [->] (\i+0.1,-1) -- (\i+0.1, -1.25);
};

\draw (0, -1.25) rectangle (\inputsize + 0.2, -2.25);
\draw (\inputsize / 2 + 0.1, -1.75) node {Encoder};

\foreach \i in {0,...,\inputsize} {
	\draw [->] (\i+0.1,-2.25) -- (\i+0.1, -2.5);
    \draw[fill=yellow!40] (\i,-2.5) rectangle (\i+0.2,-3);

    \draw [->] (\i+0.1,-3) -- (\i+0.1, -3.25);
	\draw[fill=blue!40] (\i,-3.25) rectangle (\i+0.2,-3.75);
	\draw[fill=red!40] (\i,-3.75) rectangle (\i+0.2,-4.25);

    \draw [dashed,->] (\i+0.1,-4.25) -  - (\i+0.1, -4.75);
    \draw [dashed,->] (\i+0.2,-3.5) .. controls (\i + 0.6, -3.65) .. (\i+0.6, -4.75);

	\draw[fill=red!40] (\i,-4.75) rectangle (\i+0.2,-5.25);
	\draw[fill=blue!40] (\i + 0.5,-4.75) rectangle (\i+0.7,-5.25);

    \draw [->] (\i+0.1,-5.25) - - (\i+0.1, -5.5);
    \draw [->] (\i+0.6,-5.25) - - (\i+0.6, -5.5);
};

\draw (\inputsize + 1.2, -2.75) node {$\mathbf{h}$};
\draw (\inputsize + 1.2, -3.75) node {$W_{\text{spl}}\mathbf{h}$};
\draw (\inputsize + 1.2, -5.00) node {$\mathbf{s}$};

\draw (0, -5.5) rectangle (\inputsize + 0.7, -6.5);
\draw (\inputsize / 2 + 0.5 + 0.1, -6.0) node {Decoder};

\draw [fill=green!80!black!60] (\inputsize / 2 + 0.4,-7.2) circle [x radius=\inputsize / 2 + 0.4, y radius=0.5];
\draw (\inputsize / 2 + 0.6, -7.2) node {Connectionist Temporal Classification};

\foreach \i in {0,...,\inputsize} {
   \draw [->] (\i+0.1,-7.9) - - (\i+0.1, -8.15);
   \draw [->] (\i+0.6,-7.9) - - (\i+0.6, -8.15);
}

\draw  (0+0.1,-8.4) node {$w_1$};
\draw  (0+0.6,-8.4) node {$w_2$};
\draw  (1+0.1,-8.4) node {$w_3$};
\draw  (1+0.6,-8.4) node {$\varnothing$};
\draw  (2+0.1,-8.4) node {$w_4$};
\draw  (2+0.6,-8.4) node {$\varnothing$};
\draw  (3+0.1,-8.4) node {$w_5$};
\draw  (3+0.6,-8.4) node {$w_6$};
\draw  (4+0.1,-8.4) node {$\varnothing$};
\draw  (4+0.6,-8.4) node {$\varnothing$};
\draw  (5+0.1,-8.4) node {$\varnothing$};
\draw  (5+0.6,-8.4) node {$w_7$};
\draw  (6+0.1,-8.4) node {$w_8$};
\draw  (6+0.6,-8.4) node {$\varnothing$};
\draw  (7+0.1,-8.4) node {$w_9$};
\draw  (7+0.6,-8.4) node {$\varnothing$};

\draw (\inputsize / 2 + 0.3, -8.95) node {Output tokens / null symbols};

\end{tikzpicture}

  \caption{The scheme of the non-autoregressive architecture with
    state-splitting and CTC. The image is taken from
    \citet{libovicky-helcl-2018-end}.}%
  \label{fig:state-splitting}
\end{figure}


We implement the source-to-target length expansion by linear projections and
state splitting. This mechanism is illustrated in Figure
\ref{fig:state-splitting}. After a given Transformer layer completes its
computation, we linearly project the states
$h_1, \ldots, h_{T_x} \in \mathbb{R}^d$ into $\mathbb{R}^{kd}$. Then, we split
each of these projections into $k$ parts, which results to a $k$-times longer
sequence of states $s_1, \ldots, s_{k \cdot T_x}$ of the original dimension
$d$:
%
\begin{equation}
  s_{ck+b} = \left( W_{\text{spl}} h_c + b_{\text{spl}} \right)_{bd:(b+1)d}
\end{equation}
%
for $b=0 \ldots k-1$ and $c=1 \ldots T_x$ where
$W_{\text{spl}} \in \mathbb{R}^{d \times kd}$ and
$b_{\text{spl}} \in \mathbb{R}^{kd}$ are the trainable projection parameters.

We experiment with two placement options of the state splitting layer. First,
we try placing the state splitting at the end of the Transformer layer
stack. In this scenario, there are 12 Transformer encoder layers, followed by
the state splitting layer, whose outputs are used in the output
projection. Second, we place the state splitting layer in the middle of the
Transformer layer stack, mimicking the 6-layer encoder-decoder architecture of
the autoregressive Transformer model. In the second variant, cross-attention
can be included in the second half of the layers, which attends to the states
right after state splitting.


\section{Baseline Experiments}

\JH{Section with experiments from the 2018 paper. Should show that it works,
  but it's still far from perfect.}


%%% Local Variables:
%%% mode: latex
%%% TeX-master: "thesis"
%%% End:

% %%%%%%%%%%%%%%%%%%%%%%%%%%%%%%%%%%%%%%%%%%%%%%%%%%%%%%%%%%%%%%%%%%%%%%%%%%%%%
\chapter{Experiments}
\label{chap:experiments}
% %%%%%%%%%%%%%%%%%%%%%%%%%%%%%%%%%%%%%%%%%%%%%%%%%%%%%%%%%%%%%%%%%%%%%%%%%%%%%

In this chapter, we describe the performed experiments. \JH{co dodat?}

\section{Autoregressive Baseline Models}
\label{sec:exp:autoregressive}

\paragraph{English -- German.} For our English-to-German models, we take
dataset created and published by \citet{germann2020university}. The dataset
consists of three parts. First, there is clean parallel data from the Europarl
corpus \citep{koehn2005europarl}, the Tilde MODEL -- RAPID corpus
\citep{rozis-skadins-2017-tilde}, and News Commentary corpus from OPUS
\citep{tiedemann2012opus}. Second, the dataset contains crawled parallel data
from the Web, which are considered noisy. These sources include Paracrawl
\citep{espla-etal-2019-paracrawl}, Common
Crawl\footnote{\url{http://commoncrawl.org/}}, and WikiMatrix
\citep{schwenk2019wikimatrix}. Third, the dataset contains backtranslated
sentences of originally monolingual News Crawl datasets from years 2008 to
2017, in both translation directions. The details of the data sizes are given
in Table \ref{tab:ende-data-sizes}.

\begin{table}
  \centering
  \begin{tabular}{cc}
    \toprule
    haha & beta \\
    \midrule
    bravo & delta \\
    \bottomrule
  \end{tabular}

  \caption{The sizes of the individual data sources in the EN-DE training
    data.}%
  \label{tab:ende-data-sizes}
\end{table}

We trained the first models on the clean parallel data alone, for comparison
purposes and for the sake of completeness.


We train our autoregressive models on a mix of the clean data and the
backtranslations. We use tagged-backtranslation and we oversample the clean
dataset to form roughly 25\% of the mix. We select the top 100 million sentence
pairs from the backtranslated data according to Moore-Lewis language modeling
score. \JH{tento odstavec rozvest aspon na stranku a kus}. We also try using
the top 150 million. \JH{napsat jinak:} As the size of the clean data is 3.6
million sentence pairs, we oversample the dataset by a factor of 10 for the 100M
backtranslation version or 14 for the 150M backtranslation version.


%\input{end-end.tex}

%\input{improving.tex}

%%% Local Variables:
%%% mode: latex
%%% TeX-master: "thesis"
%%% End:

%\include{05-architectures}
%\include{06-analysis}
%\include{07-conclusions}

%
% TEXT END
%

\renewcommand{\chapterheadstartvskip}{\vspace*{0mm}} % mezera u kapitoly

\cleardoublepage{}
\bibliographystyle{csplainnat}
\addcontentsline{toc}{chapter}{Bibliography}
{\small \bibliography{references,anthology}}

\cleardoublepage{}
\addcontentsline{toc}{chapter}{List of Publications}
\chapter*{List of Publications}

TODO tohle se musí přepsat

\phantom{\nobibliography*{references}}

% % % % % % % % % % % % % % % % % % % % % % % % % % % % % % % % % % % % % % % %

\noindent\bibentry{libovicky2016cuni}
\begin{itemize}[noitemsep,topsep=0pt]

    \item The paper describes a submission to the WMT16 which describes our early
        experiments with \gls{mmt}. This paper is partially discussed in
        Sections~\ref{sec:mmtsota} and~\ref{sec:aclattstrat}.

    \item Citations (without self-citations): 28

\end{itemize}\vspace{3mm}

% % % % % % % % % % % % % % % % % % % % % % % % % % % % % % % % % % % % % % % %

\noindent\bibentry{libovicky2017attention}
\begin{itemize}[noitemsep,topsep=0pt]

  \item The paper introduces techniques for combining multiple different inputs in
      sequence-to-sequence learning using \glspl{rnn}. Content of this paper is
        discussed mainly in Section~\ref{sec:aclattstrat}.

  \item Awarded as an Outstanding Paper at ACL~2017 and ÚFAL Best Paper 2017.

  \item Citations (without self-citations): 19

\end{itemize}\vspace{3mm}

% % % % % % % % % % % % % % % % % % % % % % % % % % % % % % % % % % % % % % % %

\noindent\bibentry{helcl2017neural}
\begin{itemize}[noitemsep,topsep=0pt]

  \item This paper introduces a software tool \emph{Neural Monkey} which was
      used for all the experiments in this thesis.

  \item Citations (without self-citations): 22

\end{itemize}\vspace{3mm}

% % % % % % % % % % % % % % % % % % % % % % % % % % % % % % % % % % % % % % % %

\newpage
\noindent\bibentry{helcl2017cuni}
\begin{itemize}[noitemsep,topsep=0pt]

  \item A submission to WMT17 \gls{mmt} task. The submission tests
      architectures proposed in the previous paper in a more competitive setup
        and discusses techniques for acquiring additional training data which
        are discussed in Section~\ref{sec:technical}.

  \item Citations (without self-citations): 7

\end{itemize}\vspace{3mm}

% % % % % % % % % % % % % % % % % % % % % % % % % % % % % % % % % % % % % % % %

\noindent\bibentry{helcl2018neural}
\begin{itemize}[noitemsep,topsep=0pt]

  \item A paper summarizing the Neural Monkey software at the beginning of
      2018.

  \item Citations (without self-citations): 1

\end{itemize}\vspace{3mm}

% % % % % % % % % % % % % % % % % % % % % % % % % % % % % % % % % % % % % % % %

\noindent\bibentry{libovicky2018input}
\begin{itemize}[noitemsep,topsep=0pt]

  \item The paper introduces techniques for input combinations in
      sequence-to-sequence models with self-attentive encoder and decoder.

  \item Citations (without self-citations): 0

\end{itemize}\vspace{3mm}

% % % % % % % % % % % % % % % % % % % % % % % % % % % % % % % % % % % % % % % %

\noindent\bibentry{helcl2018cuni}
\begin{itemize}[noitemsep,topsep=0pt]

  \item A paper summarizing the Neural Monkey software at the beginning of
      2018.

  \item Citations (without self-citations): 0

\end{itemize}\vspace{3mm}

\vspace{1cm}

\noindent Only publication relevant to this thesis are included. The number of
citations was computed using Google Scholar. Total number of citations of
publication related to the topic of the thesis (without self-citations):
{\large 77} (by the thesis submission on March 21, 2019).



\cleardoublepage{}
\addcontentsline{toc}{chapter}{List of Abbreviations}
\renewcommand*{\acronymname}{List of Abbreviations}
\printglossary[type=\acronymtype,style=index]

\addcontentsline{toc}{chapter}{List of Tables}
{\small \listoftables\par}

\addcontentsline{toc}{chapter}{List of Figures}
{\small \listoffigures\par}

\end{document}
