% %%%%%%%%%%%%%%%%%%%%%%%%%%%%%%%%%%%%%%%%%%%%%%%%%%%%%%%%%%%%%%%%%%%%%%%%%%%%%
\chapter{Non-autoregressive NMT with Connectionist Temporal Classification}
\label{chap:nar-nmt-ctc}
% %%%%%%%%%%%%%%%%%%%%%%%%%%%%%%%%%%%%%%%%%%%%%%%%%%%%%%%%%%%%%%%%%%%%%%%%%%%%%



% -----------------------------------------------------------------------------
\section{Connectionist Temporal Classification}
\label{sec:ctc}
% -----------------------------------------------------------------------------

\Gls{ctc} \citep{graves2006connectionist} is a method for training neural
networks on sequential data. Originally applied to the phonetic labelling task,
but later successfully adapted in related areas, including \gls{asr} or
handwriting recognition \citep{liwicki2007novel, eyben2009speech,
  graves2014towards}.

The main strength of \gls{ctc} becomes evident in tasks where the input and
output labels are weakly or not at all aligned, for example in situations where
the observed input sequence is considerably longer than the target output
sequence -- hence the application to \gls{asr}, where the number of extracted
features per second is much higher than the number of words uttered per second.



%%% Local Variables:
%%% mode: latex
%%% TeX-master: "thesis"
%%% End:
