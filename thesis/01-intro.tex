% %%%%%%%%%%%%%%%%%%%%%%%%%%%%%%%%%%%%%%%%%%%%%%%%%%%%%%%%%%%%%%%%%%%%%%%%%%%%%
\chapter{Introduction}
\label{chap:intro}
% %%%%%%%%%%%%%%%%%%%%%%%%%%%%%%%%%%%%%%%%%%%%%%%%%%%%%%%%%%%%%%%%%%%%%%%%%%%%%

\begin{markdown}

* ahoj
* beta
    - hovno
    - hovado

\end{markdown}


In this thesis, we present two case studies aimed on this phenomenon. In the
first study, we provide additional context to the \gls{nmt} system and observe
the changes in translation quality. In the other case study, we go the other
way: we limit the context available to the decoder and examine the model
behavior.


This thesis is structured as follows. In Chapter \ref{chap:nmt}, we describe
the basics of \gls{nmt} as well as recent advancements in the field. Chapter
\ref{chap:context} summarizes the scientific contributions exploring the
influence of context in \gls{nmt} from different perspectives. We present two
case studies focused on the importance of context in \gls{nmt} in Chapters
\ref{chap:mmmt} and \ref{chap:nar}. The discussion of the results observed in
the case studies is given in Chapter \ref{chap:discuss}.
