V poslední době nabídl výzkum strojového překladu nové metody pro zrychlení
generování.
%
Jedním z navrhovaných metod je takzvaný neautoregresivní neuronový strojový
překlad.
%
V klasických autoregresivních překladových systémech jsou výstupní
pravděpodobnostní rozdělení modelována podmíněně na předchozích výstupech.
%
Tato závislost umožňuje modelům sledovat stav překládání a obvykle vede ke
generování velmi plynulých textů.
%
Autoregresivní postup je však ze své podstaty sekvenční a nelze jej
paralelizovat.
%
Neautoregresivní systémy modelují pravděpodobnosti jednotlivých cílových slov
jako navzájem podmíněně nezávislé, což znamená, že dekódování lze paralelizovat
snadno.
%
Nevýhodou je ovšem nízká kvalita překladu ve srovnání s modely
autoregresivními.
%
Cíl výzkumu neautoregresivních metod strojového překladu je zlepšit kvalitu
překladu a zároveň uchovat vysokou rychlost dekódování.
%
Naše práce předkládá rešerši publikovaných metod a poukazuje na některé
nedostatky plynoucí z obecně přijímané evaluační metodologie.
%
Popisujeme experimenty s neautoregresivními modely trénovaných pomocí takzvané
\uv{connectionist temporal classification}.
%
Z našich výsledků plyne, že i když dosahujeme nejlepších výsledků mezi
neautoregresivními modely na datech z WMT z roku 2014, při porovnání s
nejnovějšími optimalizovanými autoregresivními systémy tyto modely pořád
zaostávají.



%%% Local Variables:
%%% mode: latex
%%% TeX-master: "thesis"
%%% End:
