V poslední době nabídl výzkum strojového překladu nové metody pro zrychlení
generování.
%
Jedním z navrhovaných metod je takzvaný neautoregresivní neuronový strojový
překlad.
%
V klasických autoregresivních překladových systémech jsou výstupní
pravděpodobnostní rozdělení modelována podmíněně na předchozích výstupech.
%
Tato závislost umožňuje modelům sledovat stav překládání a obvykle vede ke
generování velmi plynulých textů.
%
Autoregresivní postup je však ze své podstaty sekvenční a nelze jej
paralelizovat.
%
Neautoregresivní systémy modelují pravděpodobnosti jednotlivých cílových slov
jako navzájem podmíněně nezávislé, což znamená, že dekódování lze paralelizovat
snadno.
%
Nevýhodou je ovšem nízká kvalita překladu ve srovnání s modely
autoregresivními.
%
Cíl výzkumu neautoregresivních metod strojového překladu je zlepšit kvalitu
překladu a zároveň uchovat vysokou rychlost dekódování.
%
V naší práci se věnujeme výzkumu neautoregresivních modelů trénovaných pomocí
takzvané \uv{connectionist temporal classification}.
%





%%% Local Variables:
%%% mode: latex
%%% TeX-master: "thesis"
%%% End:
