In recent years, a number of mehtods for improving the decoding speed of neural
machine translation systems have emerged.
%
One of the approaches that proposes fundamental changes to the model
architecture are non-autoregressive models.
%
In standard autoregressive models, the output token distributions are
conditioned on the previously decoded outputs.
% Autoregressive models impose conditional dependence of the generated words on
% the previously decoded outputs.
%
The conditional dependence allows the model to keep track of the state of the
decoding process, which improves the fluency of the output.
%
On the other hand, it requires the neural network computation to be run
sequentially, and thus it cannot be parallelized.
%
Non-autoregressive models impose conditional independence on the output
distributions, which means that the decoding process is parallelizable and
hence the decoding speed improves.
%
A major drawback of this approach is lower translation quality compared to the
autoregressive models.
%
The goal of the non-autoregressive translation research is to find methods that
improve the translation quality, while retaining high decoding speed.
%
In this thesis, we explore the research progress so far and identify flaws in
the generally accepted evaluation methodology.
%
We experiement with non-autoregressive models trained with connectionist
temporal classification.
%
We find that even though our models achieve state-of-the-art performance on the
standard WMT~14 benchmark, there is a large room for improvement when we
compare non-autoregressive methods to highly optimized autoregressive models.


%%% Local Variables:
%%% mode: latex
%%% TeX-master: "thesis"
%%% End:
