In recent years, a number of mehtods for improving the decoding speed of neural
machine translation systems have emerged.
%
One of the approaches that proposes fundamental changes to the model
architecture are non-autoregressive models.
%
In standard autoregressive models, the output token distributions are
conditioned on the previously decoded outputs.
% Autoregressive models impose conditional dependence of the generated words on
% the previously decoded outputs.
%
The conditional dependence allows the model to keep track of the state of the
decoding process, which improves the fluency of the output.
%
On the other hand, it requires the neural network computation to be run
sequentially, and thus it cannot be parallelized.
%
Non-autoregressive models impose conditional independence on the output
distributions, which means that the decoding process is parallelizable and
hence the decoding speed improves.
%
A major drawback of this approach is lower translation quality compared to the
autoregressive models.
%
The goal of non-autoregressive machine translation is to improve the
translation quality, while retaining high decoding speed.
%
In this thesis, we explore the research progress so far, and bring together a
number of techniques to advance towards the stated goal.
%
We experiement with non-autoregressive models trained with connectionist
temporal classification.
%
\JH{sentence or two about the methods and findings}




%%% Local Variables:
%%% mode: latex
%%% TeX-master: "thesis"
%%% End:
