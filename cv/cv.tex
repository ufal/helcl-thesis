\documentclass[10pt,a4paper, sans]{moderncv}

%% ModernCV themes
\moderncvstyle{classic}
\moderncvcolor{black}
%\renewcommand{\familydefault}{\sfdefault}
\nopagenumbers{}

%% Character encoding
\usepackage[utf8]{inputenc}

%% Adjust the page margins
\usepackage[scale=0.75]{geometry}

%% add google scholar stuff
% \newcommand*{\scholarsocialsymbol}{\includegraphics[height=.7\baselineskip]{google-scholar}}
% \collectionadd[scholar]{socials}{\href{https://scholar.google.com/citations?user=YkypSXgAAAAJ}{https://scholar.google.com/citations?user=YkypSXgAAAAJ}}


%% Personal data
\firstname{Jindřich}
\familyname{Helcl}
\title{Curriculum Vitae}
\address{Pod Marjánkou 41}{169 00 Prague}{Czech Republic}% {Česká republika}
\mobile{+420~602~551~727}
\email{jindra.helcl@gmail.com}
\homepage{ufal.mff.cuni.cz/jindrich-helcl}


% \extrainfo{Datum narození: 9. února, 1988}
% \extrainfo{Date of birth: 9 February, 1988}

\photo[64pt][0.4pt]{jindra.jpg}

%%------------------------------------------------------------------------------
%% Content
%%------------------------------------------------------------------------------
\begin{document}
\makecvtitle

% \section{Vzdělání}
\section{Education}

% \cventry{2014--současnost}{PhD.}{Ústav formální a aplikované lingvistiky, Univerzita Karlova}{Praha}{}{Téma disertace: Neautoregresivní neuronový strojový překlad}
% \cventry{2011--2014}{Mgr.}{Matematická lingvistika, Univerzita Karlova}{Praha}{}{}
% \cventry{2007--2011}{Bc.}{Informatika, Univerzita Karlova}{Praha}{}{}

\cventry{2014--2022}{PhD.}{Institute of Formal and Applied Linguistics, Charles University}{Prague}{}{Thesis topic: Non-Autoregressive Neural Machine Translation}
\cventry{2011--2014}{Mgr.}{Computational Linguistics, Charles University}{Prague}{}{}
\cventry{2007--2011}{Bc.}{Computer Science, Charles University}{Prague}{}{}

% \section{Pracovní zkušenosti}
\section{Work Experience}

% \cventry{2017--současnost}{Výzkumný pracovník}{Ústav formální a aplikované lingvistiky, Univerzita Karlova}{Praha}{}{Výzkum zaměřen na multimodální překlad a využití lingvistické informace v neuronovém strojovém překladu.}
% \cventry{2016}{Výzkumný pracovník}{Deutsche Forschungszentrum für Künstliche Intelligenz (DFKI)}{Berlín}{}{Pětiměsíční výzkumná pozice; Neuronový strojový překlad}
% \cventry{2012--2016}{Java/C++ Developer}{IBM Praha R\&D Lab}{Praha}{}{Studentská pozice}
% \cventry{2014--2015}{Expert datových analýz}{Technologická agentura České republiky (TAČR)}{Praha}{}{Práce na projektu zaměřená na shlukovou analýzu velkého množství dokumentů.}
% \cventry{2011--2012}{PHP programátor}{Intya, s.r.o}{Prague}{}{Vývoj e-shopů a dalších webových stránek.}

\cventry{2017--present}{Research Assistant}{Institute of Formal and Applied Linguistics, Charles University}{Prague}{}{Multimodal MT, non-autoregressive decoding, lexically grounded tokenization.}
\cventry{2020--2022}{Research Associate}{University of Edinburgh}{Edinburgh}{}{Working on non-autoregressive neural machine translation and on machine translation under low-resource settings.}
\cventry{2016}{Research Fellow}{Deutsche Forschungszentrum für Künstliche Intelligenz (DFKI)}{Berlin}{}{Five-month research fellowship; neural machine translation}
\cventry{2012--2016}{Java/C++ Developer}{IBM Prague R\&D Lab}{Prague}{}{Student position}
\cventry{2014--2015}{Data Analysis Expert}{Technological agency of the Czech Republic (TAČR)}{Prague}{}{Document-level clustering on large data collections.}
\cventry{2011--2012}{PHP developer}{Intya, s.r.o}{Prague}{}{Development of e-shop applications and other web pages.}


% \section{Stáže a výzkumné pobyty}
\section{Research Internships}

% \cventry{2019}{Microsoft}{}{}{}{Čtyřměsíční výzkumná stáž; neautoregresivní modely pro neuronový strojový překlad}
% \cventry{2017--2018}{Google Inc.}{}{}{}{Čtyřměsíční výzkumná stáž; deep learning v NLP}
% \cventry{2016--2017}{University of Edinburgh}{}{}{}{Čtyřměsíční studijní pobyt; neuronový strojový překlad}

\cventry{2019}{Microsoft}{}{}{}{Four-month research internship; non-autoregressive models for neural machine translation}
\cventry{2017--2018}{Google}{}{}{}{Four-month research internship; deep learning for NLP}
\cventry{2016--2017}{University of Edinburgh}{}{}{}{Four-month research internship; neural machine translation}



% \section{Ocenění}
\section{Awards}

\cventry{2022}{Steven Krauwer Award}{}{awarded at the CLARIN Annual Conference to
  the ÚFAL for Ukraine team for the Czech--Ukrainian translation tool built in
  response to the Russian invasion to Ukraine in 2022 and the subsequent
  refugee crisis in the Czech Republic}{Prague}{}

\cventry{2017}{Outstanding paper}{}{for paper ``Attention Strategies for
  Multi-Source Sequence-to-Sequence Learning'' on the ACL 2017
  conference}{Vancouver}{}


\newpage

\section{Academic Community Service}

\cvitem{2022, 2024}{Main organizer of the \emph{MT Marathon} in Prague, Czech Republic}

\cvitem{2024}{Co-organizer of the \emph{Mu-SHROOM SemEval-2025 Task 3}}

\cvitem{2017--present}{Rewieving for \emph{ARR (*CL, EMNLP), COLING, LREC}}

\cvitem{2017}{Co-organizer of the \emph{WMT17 Neural MT Training Task}}

\cvitem{2017}{Presented the Neural Machine Translation tutorial at \emph{RANLP 2017} in Varna, Bulgaria}

% \section{Publikace}
\section{Selected Bibliography}
\cvline{}{\small See \url{https://scholar.google.cz/citations?user=YkypSXgAAAAJ} for full list}

\subsection{\textbf{CORE A/A* ranked venues and impacted journals}}

\cvitem{2024}{Jindřich Libovický, Jindřich Helcl: Lexically Grounded Subword
  Segmentation. In \emph{Proceedings of the 2024 Conference on Empirical
  Methods in Natural Language Processing (EMNLP)}}

\cvitem{2022}{Jindřich Helcl, Barry Haddow, Alexandra Birch: Non-Autoregressive
  Machine Translation: It's Not as Fast as it Seems. In \emph{Proceedings of
  the 2022 Conference of the North American Chapter of the Association for
  Computational Linguistics: Human Language Technologies (NAACL).}}

\cvitem{2022}{Barry Haddow, Rachel Bawden, Antonio Valerio Miceli Barone,
  Jindřich Helcl, Alexandra Birch: Survey of Low-Resource Machine
  Translation. In \emph{Computational Linguistics, Vol. 48, No. 3}}

\cvitem{2018}{Jindřich Libovický, Jindřich Helcl: End-to-End Non-Autoregressive
  Neural Machine Translation with Connectionist Temporal Classification.  In
  \emph{Proceedings of the 2018 Conference on Empirical Methods in Natural Language
  Processing (EMNLP)}}

\cvitem{2017}{Jindřich Libovický, Jindřich Helcl: Attention Strategies for
  Multi-Source Sequence-to-Sequence Learning. In \emph{Proceedings of the 55th
  Annual Meeting of the Association for Computational Linguistics (ACL)}}


\subsection{\textbf{Other peer-reviewed venues}}

\cvitem{2023}{Jindřich Helcl, Zdeněk Kasner, Ondřej Dušek, Tomasz Limisiewicz,
  Dominik Macháček, Tomáš Musil, Jindřich Libovický: Teaching LLMs at Charles
  University: Assignments and Activities. In \emph{Proceedings of the Sixth
   Workshop on Teaching NLP (TeachNLP)}}

\cvitem{2021}{Pinzen Chen, Jindřich Helcl, Ulrich Germann, Laurie Burchell,
  Nikolay Bogoychev, Antonio Valerio Miceli Barone, Jonas Waldendorf, Alexandra
  Birch, Kenneth Heafield: The University of Edinburgh’s English-German and
  English-Hausa submissions to the WMT21 news translation task. In
  \emph{Proceedings of the Sixth Conference on Machine Translation (WMT)}}

\cvitem{2018}{Jindřich Libovický, Jindřich Helcl, David Mareček: Input
  Combination Strategies for Multi-Source Transformer Decoder. In
  \emph{Proceedings of the Third Conference on Machine Translation (WMT)}}

\cvitem{2017}{Antonio Valerio Miceli Barone, Jindřich Helcl, Rico Sennrich,
  Barry Haddow and Alexandra Birch: Deep Architectures for Neural Machine
  Translation. In \emph{Proceedings of the Second Conference on Machine
  Translation (WMT)}}

\cvitem{2017}{Jindřich Helcl, Jindřich Libovický: CUNI System for the WMT17
  Multimodal Translation Task. In \emph{Proceedings of the Second Conference on
  Machine Translation (WMT)}}

\cvitem{2017}{Jindřich Helcl, Jindřich Libovický: Neural Monkey: An Open-Source
Toolkit for Sequence Learning. In \emph{The Prague Bulletin of Mathematical
Linguistics}}

\cvitem{2016}{Jindřich Libovický, Jindřich Helcl, Marek Tlustý, Pavel Pecina
  and Ondřej Bojar: CUNI System for WMT16 Automatic Post-Editing and Multimodal
  Translation Tasks. In \emph{Proceedings of the First Conference on Machine
  Translation (WMT)}}

% \subsection{\textbf{Preprints}}

% \cvitem{2023}{Nikolay Bogoychev, Jelmer van der Linde, Graeme Nail, Barry
%   Haddow, Jaume Zaragoza-Bernabeu, Gema Ramírez-Sánchez, Lukas Weymann, Tudor
%   Nicolae Mateiu, Jindřich Helcl, Mikko Aulamo: OpusCleaner and OpusTrainer,
%   open source toolkits for training Machine Translation and Large language
%   models. \url{https://arxiv.org/abs/2311.14838}}

% \cvitem{2020}{Zdeněk Kasner, Jindřich Libovický, Jindřich Helcl: Improving
%   Fluency of Non-Autoregressive Neural Machine Translation. \url{https://arxiv.org/abs/2004.03227}}

% \section{Computer Skills}
% \section{Počítačové dovednosti}

% \cvitem{programování}{\textbf{Python, Java, Bash}, Perl, PHP}
% \cvitem{jiné}{Unix, Emacs, Git}

% \cvitem{programming}{\textbf{Python, Bash}, Pytorch, Tensorflow, C/C++, Java}
% \cvitem{other}{Unix, Emacs, Git}

% \subsection{Strojové učení a NLP}

% \cvitem{}{Zkušenosti s řízeným i neřízeným učením, neuronovými sítěmi, zpracováním velkých kolekcí dat.
%   Aplikováno na úlohách neuronového strojového překladu, clusterování dokumentů, detekce kolokací, jazykového modelování.}

% \subsection{Machine learning and NLP}
% \cvitem{}{Experience with both supervised and unsupervised learning, neural networks (using TensorFlow), processing of large data collections.
%   Applied on neural machine translation, document clustering, language modeling.}


% \section{Jazykové znalosti}
% \section{Language Skills}

% % \cvitemwithcomment{}{Angličtina}{Velmi dobrá znalost}
% % \cvitemwithcomment{}{Němčina}{Pasivní znalost}

% \cvitemwithcomment{}{English}{Professional working proficiency}
% \cvitemwithcomment{}{German}{Elementary proficiency}

\end{document}
