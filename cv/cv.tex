\documentclass[10pt,a4paper, sans]{moderncv}

%% ModernCV themes
\moderncvstyle{classic}
\moderncvcolor{black}
%\renewcommand{\familydefault}{\sfdefault}
\nopagenumbers{}

%% Character encoding
\usepackage[utf8]{inputenc}

%% Adjust the page margins
\usepackage[scale=0.75]{geometry}

%% Personal data
\firstname{Jindřich}
\familyname{Helcl}
\title{Curriculum Vitae}
\address{Lípová 103/25}{417 02 Dubí u Teplic}{Czech Republic}% {Česká republika}
\mobile{+420~602~551~727}
%\phone{+2~(345)~678~901}
%\fax{+3~(456)~789~012}
\email{jindra.helcl@gmail.com}
\homepage{ufal.mff.cuni.cz/jindrich-helcl}

% \extrainfo{Datum narození: 9. února, 1988}
% \extrainfo{Datum narození: 9. února, 1988}

\photo[64pt][0.4pt]{jindra.jpg}

%%------------------------------------------------------------------------------
%% Content
%%------------------------------------------------------------------------------
\begin{document}
\makecvtitle

% \section{Vzdělání}
\section{Education}

% \cventry{2014--současnost}{PhD.}{Ústav formální a aplikované lingvistiky, Univerzita Karlova}{Praha}{}{Téma disertace: Využití externí informace v neuronovém strojovém překladu}
% \cventry{2011--2014}{Mgr.}{Matematická lingvistika, Univerzita Karlova}{Praha}{}{}
% \cventry{2007--2011}{Bc.}{Informatika, Univerzita Karlova}{Praha}{}{}

\cventry{2014--present}{PhD.}{Institute of Formal and Applied Linguistics, Charles University}{Prague}{}{Thesis topic: On the Importance of Context in Neural Machine Translation}
\cventry{2011--2014}{Mgr.}{Computational Linguistics, Charles University}{Prague}{}{}
\cventry{2007--2011}{Bc.}{Computer Science, Charles University}{Prague}{}{}

% \section{Diplomová práce}
\section{Master Thesis}

% \cvitem{název}{\emph{Vícejazyčná databáze kolokací}}
% \cvitem{vedoucí}{prof. RNDr. Jan Hajič, Dr.}
% \cvitem{popis}{Databáze českých a anglických kolokací vytvořená aplikací statistických metod na velké kolekce dat.}

\cvitem{name}{\emph{Multilingual Collocation Database}}
\cvitem{supervisor}{prof. RNDr. Jan Hajič, Dr.}
\cvitem{description}{A database of Czech and English collocation retrieved from big data collections using statistical methods.}

% \section{Stáže a výzkumné pobyty}
\section{Research Internships}

% \cventry{2017--2018}{Google Inc.}{}{}{}{Čtyřměsíční výzkumná stáž; deep learning v NLP}
% \cventry{2016--2017}{University of Edinburgh}{}{}{}{Čtyřměsíční studijní pobyt; neuronový strojový překlad}

\cventry{2019}{Microsoft}{}{}{}{Four-month research internship; non-autoregressive models for neural machine translation}
\cventry{2017--2018}{Google}{}{}{}{Four-month research internship; deep learning for NLP}
\cventry{2016--2017}{University of Edinburgh}{}{}{}{Four-month research internship; neural machine translation}

% \section{Pracovní zkušenosti}
\section{Work Experience}

% \cventry{2017--současnost}{Výzkumný pracovník}{Ústav formální a aplikované lingvistiky, Univerzita Karlova}{Praha}{}{Výzkum zaměřen na multimodální překlad a využití lingvistické informace v neuronovém strojovém překladu.}
% \cventry{2016}{Výzkumný pracovník}{Deutsche Forschungszentrum für Künstliche Intelligenz (DFKI)}{Berlín}{}{Pětiměsíční výzkumná pozice; Neuronový strojový překlad}
% \cventry{2012--2016}{Java/C++ Developer}{IBM Praha R\&D Lab}{Praha}{}{Studentská pozice}
% \cventry{2014--2015}{Expert datových analýz}{Technologická agentura České republiky (TAČR)}{Praha}{}{Práce na projektu zaměřená na shlukovou analýzu velkého množství dokumentů.}
% \cventry{2011--2012}{PHP programátor}{Intya, s.r.o}{Prague}{}{Vývoj e-shopů a dalších webových stránek.}

\cventry{2017--present}{Research Assistant}{Institute of Formal and Applied Linguistics, Charles University}{Prague}{}{Research on multimodal translation and improving neural MT with linguistic information.}
\cventry{2016}{Research Fellow}{Deutsche Forschungszentrum für Künstliche Intelligenz (DFKI)}{Berlin}{}{Five-month research fellowship; neural machine translation}
\cventry{2012--2016}{Java/C++ Developer}{IBM Prague R\&D Lab}{Prague}{}{Student position}
\cventry{2014--2015}{Data Analysis Expert}{Technological agency of the Czech Republic (TAČR)}{Prague}{}{Document-level clustering on large data collections.}
\cventry{2011--2012}{PHP developer}{Intya, s.r.o}{Prague}{}{Development of e-shop applications and other web pages.}


% \section{Publikace}
\section{Selected Bibliography}


\cvitem{2018}{Jindřich Libovický, Jindřich Helcl: End-to-End Non-Autoregressive Neural Machine Translation with Connectionist Temporal Classification.
  In \emph{Proceedings of the Conference on Empirical Methods in Natural Language Processing EMNLP 2018}}

\cvitem{2018}{Jindřich Libovický, Jindřich Helcl, David Mareček: Input Combination Strategies for Multi-Source Transformer Decoder. In \emph{Proceedings of the Third Conference on Machine Translation (WMT)}}

\cvitem{2018}{Jindřich Helcl, Jindřich Libovický, Dušan Variš: 	CUNI System for WMT18 Multimodal Tranlsation Task. In \emph{Proceedings of the Third Conference on Machine Translation (WMT)}}

\cvitem{2017}{Antonio Valerio Miceli Barone, Jindřich Helcl, Rico Sennrich, Barry Haddow and Alexandra Birch: Deep Architectures for Neural Machine Translation. In \emph{Proceedings of the Second Conference on Machine Translation (WMT)}}

\cvitem{2017}{Ondřej Bojar, Jindřich Helcl, Tom Kocmi, Jindřich Libovický, Tomáš Musil: Results of the WMT17 Neural MT Training Task. In \emph{Proceedings of the Second Conference on Machine Translation}}

\cvitem{2017}{Jindřich Helcl, Jindřich Libovický: CUNI System for the WMT17 Multimodal Translation Task. In \emph{Proceedings of the Second Conference on Machine Translation (WMT)}}

\cvitem{2017}{Jindřich Libovický, Jindřich Helcl: Attention Strategies for Multi-Source Sequence-to-Sequence Learning. In \emph{Proceedings of the 55th Annual Meeting of the Association for Computational Linguistics (ACL)}}

\cvitem{2017}{Jindřich Helcl, Jindřich Libovický: Neural Monkey: An Open-Source Toolkit for Sequence Learning. In \emph{The Prague Bulletin of Mathematical Linguistics}}

\cvitem{2016}{Eleftherios Avramidis, Vivien Macketanz, Aljoscha Burchardt, Jindřich Helcl, Hans Uszkoreit: Deeper Machine Translation and Evaluation for German. In \emph{Proceedings of the 2nd Deep Machine Translation Workshop}}

\cvitem{2016}{Ondřej Bojar, Ondřej Cífka, Jindřich Helcl, Tom Kocmi, Roman Sudarikov: UFAL Submissions to the IWSLT 2016 MT Track. In \emph{Proceedings of the ninth International Workshop on Spoken Language Translation (IWSLT)}}

\cvitem{2016}{Jindřich Libovický, Jindřich Helcl, Marek Tlustý, Pavel Pecina and Ondřej Bojar: CUNI System for WMT16 Automatic Post-Editing and Multimodal Translation Tasks. In \emph{Proceedings of the First Conference on Machine Translation (WMT)}}

% \section{Ocenění}
\section{Awards}

\cventry{2017}{Outstanding paper}{for paper ``Attention Strategies for Multi-Source Sequence-to-Sequence Learning'' on the ACL 2017 conference}{Vancouver}{}{}

\section{Computer Skills}
% \section{Počítačové dovednosti}

% \cvitem{programování}{\textbf{Python, Java, Bash}, Perl, PHP}
% \cvitem{jiné}{Unix, Emacs, Git}

\cvitem{programming}{\textbf{TensorFlow, Python, Bash}, Pytorch, C/C++, Java}
\cvitem{other}{Unix, Emacs, Git}

% \subsection{Strojové učení a NLP}

% \cvitem{}{Zkušenosti s řízeným i neřízeným učením, neuronovými sítěmi, zpracováním velkých kolekcí dat.
%   Aplikováno na úlohách neuronového strojového překladu, clusterování dokumentů, detekce kolokací, jazykového modelování.}

\subsection{Machine learning and NLP}
\cvitem{}{Experience with both supervised and unsupervised learning, neural networks (using TensorFlow), processing of large data collections.
  Applied on neural machine translation, document clustering, language modeling.}


% \section{Jazykové znalosti}
\section{Language Skills}

% \cvitemwithcomment{}{Angličtina}{Velmi dobrá znalost}
% \cvitemwithcomment{}{Němčina}{Pasivní znalost}

\cvitemwithcomment{}{English}{Professional working proficiency}
\cvitemwithcomment{}{German}{Elementary proficiency}

\end{document}
