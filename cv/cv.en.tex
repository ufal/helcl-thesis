\documentclass[10pt,a4paper, sans]{moderncv}

%% ModernCV themes
\moderncvstyle{classic}
\moderncvcolor{black}
%\renewcommand{\familydefault}{\sfdefault}
\nopagenumbers{}

%% Character encoding
\usepackage[utf8]{inputenc}

%% Adjust the page margins
\usepackage[scale=0.75]{geometry}

%% Personal data
\firstname{Jindřich}
\familyname{Helcl}
\title{Curriculum Vitae}
\address{Lípová 103/25}{417 02 Dubí u Teplic}{Česká republika}
\mobile{+420~602~551~727}
%\phone{+2~(345)~678~901}
%\fax{+3~(456)~789~012}
\email{jindra.helcl@gmail.com}
\homepage{ufal.mff.cuni.cz/jindrich-helcl}
\extrainfo{Date of birth: Feb 9, 1988}
\photo[64pt][0.4pt]{jindra.jpg}

%%------------------------------------------------------------------------------
%% Content
%%------------------------------------------------------------------------------
\begin{document}
\makecvtitle

\section{Education}
\cventry{2014--2022}{PhD.}{Institute of Formal and Applied Linguistics, Charles University}{Prague}{}{Non-Autoregressive Neural Machine Translation}

\cventry{2011--2014}{Master's Degree.}{Computational Linguistics, Charles University}{Prague}{}{}
\cventry{2007--2011}{Bachelor's Degree.}{Computer Science, Charles University}{Prague}{}{}

% \section{Master Thesis}
% \cvitem{title}{\emph{Multilingual Collocation Database}}
% \cvitem{supervisor}{prof. RNDr. Jan Hajič, Dr.}
% \cvitem{description}{A database of Czech and English collocations created using statistical methods on large data collections}

%\section{Research Interests}
%\cvitem{}{Statistical NLP: Collocations, Machine Translation, Language Modeling}

\section{Internships}

\cventry{2017--2018}{Google Inc.}{}{}{}{Four-month research internship; deep learning for NLP}

\cventry{2016--2017}{University of Edinburgh}{}{}{}{Four-month visiting student position; neural machine translation}

\section{Work Experience}

\cventry{2017--present}{Researcher}{Institute of Formal and Applied Linguistics, Charles University}{Prague}{}{Research focused on using linguistic annotation to improve neural machine translation.}

\cventry{2016}{Researcher}{German Research Center for Artificial Intelligence (DFKI)}{Berlin}{}{Five-month full-time research fellowship on neural machine translation}

\cventry{2012--2016}{Java/C++ Developer}{IBM Prague R\&D Lab}{Prague}{}{Part-time, Student programmer}

\cventry{2014--2015}{Data Analysis Expert}{Technology Agency of the Czech Republic}{Prague}{}{Part-time, work focused on large-scale document clustering.}

\cventry{2011--2012}{PHP Programmer}{Intya, s.r.o}{Prague}{}{Part-time, E-shops and Web pages development.}

\section{Selected Bibliography}

\cvitem{2017}{Antonio Valerio Miceli Barone, Jindřich Helcl, Rico Sennrich, Barry Haddow and Alexandra Birch: Deep Architectures for Neural Machine Translation. In \emph{Proceedings of the Second Conference on Machine Translation (WMT)}}

\cvitem{2017}{Jindřich Helcl, Jindřich Libovický: CUNI System for the WMT17 Multimodal Translation Task. In \emph{Proceedings of the Second Conference on Machine Translation (WMT)}}

\cvitem{2017}{Jindřich Libovický, Jindřich Helcl: Attention Strategies for Multi-Source Sequence-to-Sequence Learning. In \emph{Proceedings of the 55th Annual Meeting of the Association for Computational Linguistics (ACL)}}

\cvitem{2017}{Jindřich Helcl, Jindřich Libovický: Neural Monkey: An Open-Source Toolkit for Sequence Learning, \emph{The Prague Bulletin of Mathematical Linguistics}}

\cvitem{2016}{Jindřich Libovický, Jindřich Helcl, Marek Tlustý, Pavel Pecina and Ondřej Bojar: CUNI System for WMT16 Automatic Post-Editing and Multimodal Translation Tasks, \emph{Proceedings of the First Conference on Machine Translation (WMT)}}

\section{Computer Skills}
%\subsection{Expert Knowledge}

\cvitem{programming}{\textbf{Python, Java, Bash}, Perl, PHP}
\cvitem{other}{Unix, Emacs, Git}

\subsection{Machine Learning and NLP}

\cvitem{}{Experience with both supervised and unsupervised learning, neural networks, processing
of large datasets. Applied to neural machine translation, document clustering,
collocation identification, language modeling.}

\section{Language Skills}
\cvitemwithcomment{}{English}{Professional working proficiency}
\cvitemwithcomment{}{German}{Elementary proficiency}

\end{document}
